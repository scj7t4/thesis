
%% bare_conf.tex
%% V1.4b
%% 2015/08/26
%% by Michael Shell
%% See:
%% http://www.michaelshell.org/
%% for current contact information.
%%
%% This is a skeleton file demonstrating the use of IEEEtran.cls
%% (requires IEEEtran.cls version 1.8b or later) with an IEEE
%% conference paper.
%%
%% Support sites:
%% http://www.michaelshell.org/tex/ieeetran/
%% http://www.ctan.org/pkg/ieeetran
%% and
%% http://www.ieee.org/

%%*************************************************************************
%% Legal Notice:
%% This code is offered as-is without any warranty either expressed or
%% implied; without even the implied warranty of MERCHANTABILITY or
%% FITNESS FOR A PARTICULAR PURPOSE! 
%% User assumes all risk.
%% In no event shall the IEEE or any contributor to this code be liable for
%% any damages or losses, including, but not limited to, incidental,
%% consequential, or any other damages, resulting from the use or misuse
%% of any information contained here.
%%
%% All comments are the opinions of their respective authors and are not
%% necessarily endorsed by the IEEE.
%%
%% This work is distributed under the LaTeX Project Public License (LPPL)
%% ( http://www.latex-project.org/ ) version 1.3, and may be freely used,
%% distributed and modified. A copy of the LPPL, version 1.3, is included
%% in the base LaTeX documentation of all distributions of LaTeX released
%% 2003/12/01 or later.
%% Retain all contribution notices and credits.
%% ** Modified files should be clearly indicated as such, including  **
%% ** renaming them and changing author support contact information. **
%%*************************************************************************


% *** Authors should verify (and, if needed, correct) their LaTeX system  ***
% *** with the testflow diagnostic prior to trusting their LaTeX platform ***
% *** with production work. The IEEE's font choices and paper sizes can   ***
% *** trigger bugs that do not appear when using other class files.       ***                          ***
% The testflow support page is at:
% http://www.michaelshell.org/tex/testflow/



\documentclass[conference]{IEEEtran}
% Some Computer Society conferences also require the compsoc mode option,
% but others use the standard conference format.
%
% If IEEEtran.cls has not been installed into the LaTeX system files,
% manually specify the path to it like:
% \documentclass[conference]{../sty/IEEEtran}



\usepackage{comment} % multi-line comments
\usepackage{graphicx} % necessary for inclusion of .eps for figures
\usepackage{algpseudocode}
\usepackage{amsmath}
\usepackage{amssymb}
\usepackage{amsthm}
\usepackage{setspace}
\usepackage{subcaption}
\usepackage{caption}
\usepackage{acronym}
\usepackage[T1]{fontenc}
\usepackage{float}

\newtheorem{thm}{Theorem}
\newtheorem{pdef}{Definition}
\newtheorem{case}{Case}[thm]
\newtheorem{lem}[thm]{Lemma}
\newtheorem{cor}[thm]{Corollary}
\newtheorem{prop}[thm]{Proposition}
\newtheorem{rem}{Remark}[thm]
\newtheorem{ill}[thm]{Illustration}

\newacro{FREEDM}{Future Renewable Electric Energy Delivery and Management}
\newacro{DGI}{Distributed Grid Intelligence}
\newacro{CPS}{cyber-physical system} 
\newacro{AYC}{Are You Coordinator}
\newacro{AYT}{Are You There}
\newacro{RED}{Random Early Detection}
\newacro{NS3}[NS-3]{Network Simulator 3}
\newacro{ECN}{Explicit Congestion Notification}
\newacro{VANET}{Vehicular Ad Hoc Networks}
\newacro{EWMA}{Exponentially Weighted Moving Average}
\newacro{CDF}{Cumulative Distribution Function}

\graphicspath{ {./drawings/} {./plots/} }



% Some very useful LaTeX packages include:
% (uncomment the ones you want to load)


% *** MISC UTILITY PACKAGES ***
%
%\usepackage{ifpdf}
% Heiko Oberdiek's ifpdf.sty is very useful if you need conditional
% compilation based on whether the output is pdf or dvi.
% usage:
% \ifpdf
%   % pdf code
% \else
%   % dvi code
% \fi
% The latest version of ifpdf.sty can be obtained from:
% http://www.ctan.org/pkg/ifpdf
% Also, note that IEEEtran.cls V1.7 and later provides a builtin
% \ifCLASSINFOpdf conditional that works the same way.
% When switching from latex to pdflatex and vice-versa, the compiler may
% have to be run twice to clear warning/error messages.






% *** CITATION PACKAGES ***
%
%\usepackage{cite}
% cite.sty was written by Donald Arseneau
% V1.6 and later of IEEEtran pre-defines the format of the cite.sty package
% \cite{} output to follow that of the IEEE. Loading the cite package will
% result in citation numbers being automatically sorted and properly
% "compressed/ranged". e.g., [1], [9], [2], [7], [5], [6] without using
% cite.sty will become [1], [2], [5]--[7], [9] using cite.sty. cite.sty's
% \cite will automatically add leading space, if needed. Use cite.sty's
% noadjust option (cite.sty V3.8 and later) if you want to turn this off
% such as if a citation ever needs to be enclosed in parenthesis.
% cite.sty is already installed on most LaTeX systems. Be sure and use
% version 5.0 (2009-03-20) and later if using hyperref.sty.
% The latest version can be obtained at:
% http://www.ctan.org/pkg/cite
% The documentation is contained in the cite.sty file itself.






% *** GRAPHICS RELATED PACKAGES ***
%
\ifCLASSINFOpdf
  % \usepackage[pdftex]{graphicx}
  % declare the path(s) where your graphic files are
  % \graphicspath{{../pdf/}{../jpeg/}}
  % and their extensions so you won't have to specify these with
  % every instance of \includegraphics
  % \DeclareGraphicsExtensions{.pdf,.jpeg,.png}
\else
  % or other class option (dvipsone, dvipdf, if not using dvips). graphicx
  % will default to the driver specified in the system graphics.cfg if no
  % driver is specified.
  % \usepackage[dvips]{graphicx}
  % declare the path(s) where your graphic files are
  % \graphicspath{{../eps/}}
  % and their extensions so you won't have to specify these with
  % every instance of \includegraphics
  % \DeclareGraphicsExtensions{.eps}
\fi
% graphicx was written by David Carlisle and Sebastian Rahtz. It is
% required if you want graphics, photos, etc. graphicx.sty is already
% installed on most LaTeX systems. The latest version and documentation
% can be obtained at: 
% http://www.ctan.org/pkg/graphicx
% Another good source of documentation is "Using Imported Graphics in
% LaTeX2e" by Keith Reckdahl which can be found at:
% http://www.ctan.org/pkg/epslatex
%
% latex, and pdflatex in dvi mode, support graphics in encapsulated
% postscript (.eps) format. pdflatex in pdf mode supports graphics
% in .pdf, .jpeg, .png and .mps (metapost) formats. Users should ensure
% that all non-photo figures use a vector format (.eps, .pdf, .mps) and
% not a bitmapped formats (.jpeg, .png). The IEEE frowns on bitmapped formats
% which can result in "jaggedy"/blurry rendering of lines and letters as
% well as large increases in file sizes.
%
% You can find documentation about the pdfTeX application at:
% http://www.tug.org/applications/pdftex





% *** MATH PACKAGES ***
%
%\usepackage{amsmath}
% A popular package from the American Mathematical Society that provides
% many useful and powerful commands for dealing with mathematics.
%
% Note that the amsmath package sets \interdisplaylinepenalty to 10000
% thus preventing page breaks from occurring within multiline equations. Use:
%\interdisplaylinepenalty=2500
% after loading amsmath to restore such page breaks as IEEEtran.cls normally
% does. amsmath.sty is already installed on most LaTeX systems. The latest
% version and documentation can be obtained at:
% http://www.ctan.org/pkg/amsmath





% *** SPECIALIZED LIST PACKAGES ***
%
%\usepackage{algorithmic}
% algorithmic.sty was written by Peter Williams and Rogerio Brito.
% This package provides an algorithmic environment fo describing algorithms.
% You can use the algorithmic environment in-text or within a figure
% environment to provide for a floating algorithm. Do NOT use the algorithm
% floating environment provided by algorithm.sty (by the same authors) or
% algorithm2e.sty (by Christophe Fiorio) as the IEEE does not use dedicated
% algorithm float types and packages that provide these will not provide
% correct IEEE style captions. The latest version and documentation of
% algorithmic.sty can be obtained at:
% http://www.ctan.org/pkg/algorithms
% Also of interest may be the (relatively newer and more customizable)
% algorithmicx.sty package by Szasz Janos:
% http://www.ctan.org/pkg/algorithmicx




% *** ALIGNMENT PACKAGES ***
%
%\usepackage{array}
% Frank Mittelbach's and David Carlisle's array.sty patches and improves
% the standard LaTeX2e array and tabular environments to provide better
% appearance and additional user controls. As the default LaTeX2e table
% generation code is lacking to the point of almost being broken with
% respect to the quality of the end results, all users are strongly
% advised to use an enhanced (at the very least that provided by array.sty)
% set of table tools. array.sty is already installed on most systems. The
% latest version and documentation can be obtained at:
% http://www.ctan.org/pkg/array


% IEEEtran contains the IEEEeqnarray family of commands that can be used to
% generate multiline equations as well as matrices, tables, etc., of high
% quality.




% *** SUBFIGURE PACKAGES ***
%\ifCLASSOPTIONcompsoc
%  \usepackage[caption=false,font=normalsize,labelfont=sf,textfont=sf]{subfig}
%\else
%  \usepackage[caption=false,font=footnotesize]{subfig}
%\fi
% subfig.sty, written by Steven Douglas Cochran, is the modern replacement
% for subfigure.sty, the latter of which is no longer maintained and is
% incompatible with some LaTeX packages including fixltx2e. However,
% subfig.sty requires and automatically loads Axel Sommerfeldt's caption.sty
% which will override IEEEtran.cls' handling of captions and this will result
% in non-IEEE style figure/table captions. To prevent this problem, be sure
% and invoke subfig.sty's "caption=false" package option (available since
% subfig.sty version 1.3, 2005/06/28) as this is will preserve IEEEtran.cls
% handling of captions.
% Note that the Computer Society format requires a larger sans serif font
% than the serif footnote size font used in traditional IEEE formatting
% and thus the need to invoke different subfig.sty package options depending
% on whether compsoc mode has been enabled.
%
% The latest version and documentation of subfig.sty can be obtained at:
% http://www.ctan.org/pkg/subfig




% *** FLOAT PACKAGES ***
%
%\usepackage{fixltx2e}
% fixltx2e, the successor to the earlier fix2col.sty, was written by
% Frank Mittelbach and David Carlisle. This package corrects a few problems
% in the LaTeX2e kernel, the most notable of which is that in current
% LaTeX2e releases, the ordering of single and double column floats is not
% guaranteed to be preserved. Thus, an unpatched LaTeX2e can allow a
% single column figure to be placed prior to an earlier double column
% figure.
% Be aware that LaTeX2e kernels dated 2015 and later have fixltx2e.sty's
% corrections already built into the system in which case a warning will
% be issued if an attempt is made to load fixltx2e.sty as it is no longer
% needed.
% The latest version and documentation can be found at:
% http://www.ctan.org/pkg/fixltx2e


%\usepackage{stfloats}
% stfloats.sty was written by Sigitas Tolusis. This package gives LaTeX2e
% the ability to do double column floats at the bottom of the page as well
% as the top. (e.g., "\begin{figure*}[!b]" is not normally possible in
% LaTeX2e). It also provides a command:
%\fnbelowfloat
% to enable the placement of footnotes below bottom floats (the standard
% LaTeX2e kernel puts them above bottom floats). This is an invasive package
% which rewrites many portions of the LaTeX2e float routines. It may not work
% with other packages that modify the LaTeX2e float routines. The latest
% version and documentation can be obtained at:
% http://www.ctan.org/pkg/stfloats
% Do not use the stfloats baselinefloat ability as the IEEE does not allow
% \baselineskip to stretch. Authors submitting work to the IEEE should note
% that the IEEE rarely uses double column equations and that authors should try
% to avoid such use. Do not be tempted to use the cuted.sty or midfloat.sty
% packages (also by Sigitas Tolusis) as the IEEE does not format its papers in
% such ways.
% Do not attempt to use stfloats with fixltx2e as they are incompatible.
% Instead, use Morten Hogholm'a dblfloatfix which combines the features
% of both fixltx2e and stfloats:
%
% \usepackage{dblfloatfix}
% The latest version can be found at:
% http://www.ctan.org/pkg/dblfloatfix




% *** PDF, URL AND HYPERLINK PACKAGES ***
%
%\usepackage{url}
% url.sty was written by Donald Arseneau. It provides better support for
% handling and breaking URLs. url.sty is already installed on most LaTeX
% systems. The latest version and documentation can be obtained at:
% http://www.ctan.org/pkg/url
% Basically, \url{my_url_here}.




% *** Do not adjust lengths that control margins, column widths, etc. ***
% *** Do not use packages that alter fonts (such as pslatex).         ***
% There should be no need to do such things with IEEEtran.cls V1.6 and later.
% (Unless specifically asked to do so by the journal or conference you plan
% to submit to, of course. )


% correct bad hyphenation here
\hyphenation{op-tical net-works semi-conduc-tor}


\begin{document}
%
% paper title
% Titles are generally capitalized except for words such as a, an, and, as,
% at, but, by, for, in, nor, of, on, or, the, to and up, which are usually
% not capitalized unless they are the first or last word of the title.
% Linebreaks \\ can be used within to get better formatting as desired.
% Do not put math or special symbols in the title.
\title{Bare Demo of IEEEtran.cls\\ for IEEE Conferences}


% author names and affiliations
% use a multiple column layout for up to three different
% affiliations
\author{\IEEEauthorblockN{Michael Shell}
\IEEEauthorblockA{School of Electrical and\\Computer Engineering\\
Georgia Institute of Technology\\
Atlanta, Georgia 30332--0250\\
Email: http://www.michaelshell.org/contact.html}
\and
\IEEEauthorblockN{Homer Simpson}
\IEEEauthorblockA{Twentieth Century Fox\\
Springfield, USA\\
Email: homer@thesimpsons.com}
\and
\IEEEauthorblockN{James Kirk\\ and Montgomery Scott}
\IEEEauthorblockA{Starfleet Academy\\
San Francisco, California 96678--2391\\
Telephone: (800) 555--1212\\
Fax: (888) 555--1212}}

% conference papers do not typically use \thanks and this command
% is locked out in conference mode. If really needed, such as for
% the acknowledgment of grants, issue a \IEEEoverridecommandlockouts
% after \documentclass

% for over three affiliations, or if they all won't fit within the width
% of the page, use this alternative format:
% 
%\author{\IEEEauthorblockN{Michael Shell\IEEEauthorrefmark{1},
%Homer Simpson\IEEEauthorrefmark{2},
%James Kirk\IEEEauthorrefmark{3}, 
%Montgomery Scott\IEEEauthorrefmark{3} and
%Eldon Tyrell\IEEEauthorrefmark{4}}
%\IEEEauthorblockA{\IEEEauthorrefmark{1}School of Electrical and Computer Engineering\\
%Georgia Institute of Technology,
%Atlanta, Georgia 30332--0250\\ Email: see http://www.michaelshell.org/contact.html}
%\IEEEauthorblockA{\IEEEauthorrefmark{2}Twentieth Century Fox, Springfield, USA\\
%Email: homer@thesimpsons.com}
%\IEEEauthorblockA{\IEEEauthorrefmark{3}Starfleet Academy, San Francisco, California 96678-2391\\
%Telephone: (800) 555--1212, Fax: (888) 555--1212}
%\IEEEauthorblockA{\IEEEauthorrefmark{4}Tyrell Inc., 123 Replicant Street, Los Angeles, California 90210--4321}}




% use for special paper notices
%\IEEEspecialpapernotice{(Invited Paper)}




% make the title area
\maketitle

% As a general rule, do not put math, special symbols or citations
% in the abstract
\begin{abstract}
The abstract goes here.
\end{abstract}

% no keywords




% For peer review papers, you can put extra information on the cover
% page as needed:
% \ifCLASSOPTIONpeerreview
% \begin{center} \bfseries EDICS Category: 3-BBND \end{center}
% \fi
%
% For peerreview papers, this IEEEtran command inserts a page break and
% creates the second title. It will be ignored for other modes.
\IEEEpeerreviewmaketitle

\newacro{FREEDM}{Future Renewable Electric Energy Delivery and Management}
\newacro{DGI}{Distributed Grid Intelligence}
\newacro{CPS}{cyber-physical system} 
\newacro{AYC}{Are You Coordinator}
\newacro{AYT}{Are You There}
\newacro{RED}{Random Early Detection}
\newacro{NS3}[NS-3]{Network Simulator 3}
\newacro{ECN}{Explicit Congestion Notification}
\newacro{VANET}{Vehicular Ad Hoc Networks}
\newacro{EWMA}{Exponentially Weighted Moving Average}
\newacro{CDF}{Cumulative Distribution Function}

\section{Introduction}
% Computer Society journal papers do something a tad strange with the very
% first section heading (almost always called "Introduction"). They place it
% ABOVE the main text! IEEEtran.cls currently does not do this for you.
% However, You can achieve this effect by making LaTeX jump through some
% hoops via something like:
%
%\ifCLASSOPTIONcompsoc
%  \noindent\raisebox{2\baselineskip}[0pt][0pt]%
%  {\parbox{\columnwidth}{\section{Introduction}\label{sec:introduction}%
%  \global\everypar=\everypar}}%
%  \vspace{-1\baselineskip}\vspace{-\parskip}\par
%\else
%  \section{Introduction}\label{sec:introduction}\par
%\fi
%
% Admittedly, this is a hack and may well be fragile, but seems to do the
% trick for me. Note the need to keep any \label that may be used right
% after \section in the above as the hack puts \section within a raised box.



% The very first letter is a 2 line initial drop letter followed
% by the rest of the first word in caps (small caps for compsoc).
% 
% form to use if the first word consists of a single letter:
% \IEEEPARstart{A}{demo} file is ....
% 
% form to use if you need the single drop letter followed by
% normal text (unknown if ever used by IEEE):
% \IEEEPARstart{A}{}demo file is ....
% 
% Some journals put the first two words in caps:
% \IEEEPARstart{T}{his demo} file is ....
% 
% Here we have the typical use of a "T" for an initial drop letter
% and "HIS" in caps to complete the first word.
\IEEEPARstart{F}{REEDM} (Future Renewable Electric Energy Delivery and Management) is a smart grid project focused on the future of the electrical grid.
Major proposed features of the FREEDM network include the solid state transformer, distributed local energy storage, and distributed local energy generation \cite{FREEDMMIGRATION}.
This vein of research emphasizes decentralizing the power grid, making it more reliable by distributing energy production resources.
Part of this design requires the system to operate in islanded mode, where portions of the distribution network are partitioned from each other.
The effects of these partitions are still not well understood.
This is particularly true in a distributed cyber-physical system, in which partitions may occur in both the cyber and physical domains.
Related work\cite{HARINI}\cite{TSG} has indicated that cyber faults can cause a physical system to apply unstable settings.

This work presents a distributed leader election algorithm and Markov model of that algorithm.
The presented algorithm maintains the Markov property for the observations of the leader despite omission failures
This approach to considering how a distributed system interacts during a fault condition allows for the creation of new techniques for managing a fault scenario in cyber-physical systems. 
This discussion presents an approach that utilizes Markov chain to model a system's grouping behavior.
These chains produce expectations of how long a system can be expected to stay in a particular state as well as how much time it will be able to spend coordinating and doing useful work over a period of time. 
Using these measures, the behavior of the control system for the physical devices can be adjusted to prevent faults.

PUT IN A LIST OF SECTIONS

%\IEEEpubidadjcol
% needed in second column of first page if using \IEEEpubid
\section{FREEDM DGI}

This study models the group management module of the FREEDM DGI.
The DGI is a smart grid operating system that organizes and coordinates power electronics.
It also negotiates contracts to deliver power to devices and regions that cannot effectively facilitate their own needs.
DGI leverages common distributed algorithms to control the power grid, making it an attractive target for modeling a distributed system.

The DGI software consists of a central component, known as the broker.
This broker is responsible for presenting a communication interface.
It also furnishes any common functionality the system's algorithms may need.
These algorithms, grouped into modules, work in concert to move power from areas of excess supply to excess demand.

DGI utilizes several modules to manage a distributed smart-grid system.
Group management, the focus of this work, implements a leader election algorithm to discover which processes are reachable within the cyber domain.
Other modules provide additional functionality, such as collecting global snapshots, negotiating the migrations, and giving commands to physical components.

DGI is a real-time system; certain actions (and reactions) involving power system components need to be completed within a pre-specified time-frame to keep the system stable.
It uses a round robin scheduler in which each module is given a predetermined window of execution which it may use to perform its duties.
When a module's time period expires, the next module in the line is allowed to execute. 
The start of each round is synchronized between systems.
Each DGI process will execute the same module at the same time.

\section{Group Management}
The DGI uses the leader election algorithm, ``Invitation Election Algorithm,'' written by Garcia-Molina \cite{INVITATIONELECTION}.
Originally published in 1982, this algorithm provides a robust election  procedure that allows for transient partitions.
Transient partitions are formed when a faulty link between two or more clusters of DGIs causes the groups to divide temporarily.
These transient partitions merge when the link becomes more reliable.
The election algorithm allows for failures that disconnect two distinct sub-networks.
These sub networks are fully connected, but connectivity between the two sub-networks is limited by an unreliable link.

Many election algorithms have been created. 
Each algorithm is designed to be well-suited to the circumstances it will deployed in.
Specialized algorithms exist for wireless sensor networks \cite{LE-WSN-1}\cite{LE-WSN-2}, and other special circumstances \cite{LE-SPECIALCIRCUMSTANCES-1}\cite{LE-SPECIALCIRCUMSTANCES-2}.
Work on leader elections has been incorporated into a variety of distributed frameworks: Isis \cite{ISISTOOLKIT}, Horus \cite{HORUSTOOLKIT}, Totem \cite{TOTEMTOOLKIT}, Transis \cite{TRANSISTOOLKIT}, and Spread \cite{SPREADTOOLKIT} all have methods for creating groups.
Despite this wide array of work, the fundamentals of leader election are similar across all implementations.
Processes arrive at a consensus of a single peer that coordinates the group.
Processes that fail are detected and removed from the group. 

The elected leader is responsible for making work assignments, and identifying and merging with other coordinators when they are found, as well as maintaining an up-to-date list of peers for the members of his group. 
Group members monitor the group leader by periodically checking if the group leader is still alive by sending a message. 
If the leader fails to respond, the querying nodes will enter a recovery state and operate alone until
they can identify another coordinator.
Therefore, a leader and each of the members maintain a set of processes which are currently reachable, a subset of all known processes in the system.

Leader election can also be classified as a failure detector \cite{LEADERELECTIONEVAL}.
Failure detectors are algorithms which detect the failure of processes within a system; they maintain a list of processes that they suspect have crashed.
This informal description gives the failure detector strong ties to the leader election process. 
The group management module maintains a list of suspected processes which can be determined from the set of all processes and the current membership.

The leader and the members have separate roles to play in the failure detection process.
Leaders use a periodic search to locate other leaders in order to merge groups.
This serves as a ping / response query for detecting failures within the system.
The member sends a query to its leader.
The member will only suspect the leader, and not the other processes in their group.

\section {Experimental Setup}

\subsection{Broker Architecture}
The DGI software used in this designed around a broker architectural specification.
Each core functionality of the system was implemented within a module that was provided access to core interfaces.
These interfaces provided functionality such as scheduling requests, message passing, and a framework to manipulate physical devices.
The Broker provided a common message passing interface that all modules could access.
This interface was then used to pass information between modules. 
For this purpose of this work, messages are sent as single UDP datagrams.
If a datagram is lost, it is not resent.
DGI expect an increasing sequence number on datagrams, which ensures message ordering.

\subsection{Algorithmic Changes}

The original Garcia-Molina algorithm has been modified so the observations of the coordinator process have the Markov property.
The full algorithm is presented below.
The rest of this section describes the changes that were made to the algorithm and shows how the combination of those changes allow the observations of the Coordinator to follow the Markov process.
The execution model of this algorithm assumes a real-time system using a round-robin scheduler.
All processes have their clock synchronized to a reasonable tolerance.
At a predetermined time and following predetermined intervals the algorithm executes at each process.
Using the synchronized clocks, all processes execute either $Check()$ or $Timeout()$ at the same time.
Processes can only form groups if their clocks are sufficiently in sync with another process' clock.

\begin{algorithmic}

\State $AllNodes \gets \{ 1, 2, ..., N \}$
\State $Coordinators \gets \emptyset$
\State $UpNodes \gets { Me }$
\State $State \gets Normal$
\State $Coordinator \gets Me$
\State $Responses \gets \emptyset$
\State $Counter \gets$ A random initial identifier
\State $GroupID \gets (Me,Counter)$

\State

\Function{Check}{}
    \State This function is called at the start of a round by the leader
    \If {$State = Normal$ and $Coordinator \gets Me$}
        \State $Responses \gets \emptyset$
        \State $TempSet \gets \emptyset$
        \For {$j = (AllNodes - \{Me\})$}
            \State $AreYouCoordinator(j)$
            \State $TempSet \gets TempSet \cup j$
        \EndFor
        \State Nodes which respond "Yes" to $AreYouCoordinator$ are put into the $Responses$ set.
        \State When an $AreYouThere$ response is "No" and this process is a coordinator, the querying process is put in the $Responses$ set.
        \State Wait for $Timeout(CheckTimeout)$, Nodes that do not respond are removed from UpNodes.
        \State $UpNodes \gets (TempSet-Responses) \cup {Me}$
        \If {$Responses = \emptyset$}
            \Return
        \EndIf
        \State $p \gets \max(Responses)$
        \If $Me > p$
            \State Wait time proportional to me-p
        \EndIf
        \Call{Merge}{Responses}
    \EndIf
    \State The next call to this is after Timeout(CheckTimeout)
\EndFunction

\State

\Function{Timeout}{}
    \State This function is called at the start of a round by the group members
    \If {$Coordinator = Me$}
        \Return
    \Else
        \Call{AreYouThere}{Coordinator,GroupID,Me}
        \If{Response is No}
            \Call{Recovery}{}
        \EndIf
    \EndIf
    \State The next call to this is after Timeout(TimeoutTimeout)
\EndFunction

\State

\Function{Merge}{Coordinators}
    \State This function invites all coordinators in Coordinators to join a group led by Me
    \State $State \gets Election$
    \State Stop work
    \State $Counter \gets Counter+1$
    \State $GroupID \gets (Me,Counter)$
    \State $Coordinator \gets Me$
    \State $TempSet \gets UpNodes - {Me}$
    \State $UpNodes \gets \emptyset$
    \For {$j \in Coordinators$}
        \Call{Invite}{j,Coordinator,GroupID}
    \EndFor
    \For {$j \in TempSet$}
        \Call{Invite}{j,Coordinator,GroupID}
    \EndFor
    \State Wait for $Timeout(InviteTimeout)$, Nodes that accept the invite are added to UpNodes
    \State $State \gets Reorganization$
    \For {$j \in UpNodes$}
        \Call{Ready}{j,Coordinator,GroupID,UpNodes}
    \EndFor
    \State $Acknowledge \gets UpNodes$
    \State Wait for $Timeout(ReadyTimeout)$, Nodes that do not acknowledge are removed from UpNodes
    \State $UpNodes \gets UpNodes - Acknowledge$
    \State $State \gets Normal$
\EndFunction

\State

\Function{ReceiveReady}{Sender,Leader, Identifier, Peers}
    \If {$State = Reorganization$ and $GroupID = Identifier$}
        \State $UpNodes \gets Peers$
        \State $State \gets Normal$
        \State Respond Ready Acknowledge 
    \EndIf
\EndFunction

\State

\Function{ReceiveAreYouCoordinator}{Sender}
    \If {$State = Normal$ and $Coordinator = Me$}
        \State Respond Yes
    \Else
        \State Respond No
    \EndIf
\EndFunction

\State

\Function{ReceiveAreYouThere}{Sender, Identifier}
    \If {$GroupID = Identifier$ and $Coordinator = Me$ and $Sender \in UpNodes$}
        \State Respond Yes
    \Else
        \State Respond No
        \State Add sender to $Responses$ set for $Check()$ if this process is a coordinator.
    \EndIf
\EndFunction

\State

\Function{ReceiveInvitation}{Sender,Leader,Identifier}
    \If {$State \neq Normal$}
        \Return
    \EndIf
    \If {$Sender \neq 0$}
        \Return
    \EndIf
    \State Stop Work
    \State $Temp \gets Coordinator$
    \State $TempSet \gets UpNodes$
    \State $State \gets Election$
    \State $Coordinator \gets Leader$
    \State $GroupID \gets Identifier$
    \If {$Temp = Me$}
        \State Forward invite to old group members
        \For $j \in TempSet$
            \State $Invite(j,Coordinator,GroupID)$
        \EndFor
    \EndIf
    \State $Accept(Coordinator,GroupID)$
    \State $State \gets Reorganization$
    \If {$Timeout(ReadyTimeout)$ expires before $Ready$ is received}
        \State $Recovery()$
    \EndIf
\EndFunction

\State

\Function{ReceiveAccept}{Sender,Leader,Identifier}
    \If {$State \gets Election$ and $GroupID = Identifier$ and $Coordinator = Leader$}
        \State $UpNodes \gets UpNodes \cup {Sender}$
    \EndIf
\EndFunction

\Function{ReceiveReadyAcknowledge}{Sender}
    \State $Sender$ is removed from $Acknowledge$ in $Merge()$
\EndFunction

\State

\Function{Recovery}{}
    \State $State \gets Election$
    \State Stop Work
    \State $Counter \gets Counter + 1$
    \State $GroupID \gets (Me,Counter)$
    \State $Coordinator \gets Me$
    \State $UpNodes \gets {Me}$
    \State $State \gets Reorganization$
    \State $State \gets Normal$
\EndFunction

\end{algorithmic}

In a distributed system information cannot be instantaneously spread throughout the system.
A process can only make local observations.
In this work, we attempt to model what a process will observe as a result of omission failure.
Therefore, it is important that observations that a process makes hold to the Markov property.
In the original algorithm, there are several portions that, when projected to the leader's observation, do not meet the Markov property.
The following sections state the portions of the algorithm where the observation of the leader process does not yield the probability of next transition.

Leader selection is performed a priori-- only process 0 may become a group coordinator.
Only process 0 may become the leader of a multiprocess group.
This simplification was applied because the configuration of the system with a larger number of processes depended on the configuration of the other processes.
Without this simplification, the state of the rest of the system would not have the memoryless property.
The state of the processes that are not in the observers group would change each round.
As a consequence the state of the rest of the system and the likelihood of forming a specific group size would change each step if other processes could become leader.

DIAGRAM HERE

The changes added a third message to completing an election -- a ready acknowledge message.
This message is sent by a member after receiving the ready message from the coordinator.
This allows the coordinator to be certain of the member's status before the next round.
Without the ready acknowledgment, the member may not receive the ready message and the coordinator will observe the member is a part of the group.
As a consequence of that uncertainty, the probability that a member remains in a group in the first round after an election has a different probability than each subsequent round.
By adding the extra message, the observation of the coordinator of the state, must be the state of the member of the group.
The sequence presented in figure XXX is not possible and as a consequence, the probability a member remains in a group in the first round after an election is a fixed value.

DIAGRAM HERE

Members cannot leave a group without the leader's permission.
Members do not suspect the coordinator has failed, only the coordinator may suspect the members.
For the purpose of starting an election, an Are You There message and it's negative response are considered equivalent to a Are You Coordinator message and a positive response.
On receipt of the negative response, the member will immediately recover and become a leader.
This assumption relies on Are You Coordinator and Are You There messages being sent at roughly the same time.

DIAGRAM HERE

This change leads to a live-lock situation in a crash failure, where the group's leader crashes and does not return and as a consequence the remaining members are trapped in a group without a leader.
For the purpose of this work, we have disregarded these live lock scenarios.
However, the live-lock could be avoided if the member can detect that it has not received an "Are You Coordinator" message in a round.
When the process fails to receive the message, the coordinator must have also removed the member from their group since they could not have received a "Are You Coordinator" response message.
Integrating this component is an area of future work.

\section{Formal Modeling}

\subsection{Assumptions}

All participating peers were assumed to be on the same schedule; all peers began executing the model simultaneously.
Synchronization was accomplished using Choi's work \cite{DCS}.
It was also assumed the clocks are synchronized. 
If the network has faulted, process clocks would not drift noticeably from their last synchronization.
As an alternative, a production system would likely use GPS time synchronization to obtain certain power system readings \cite{PHASORREADINGS}.

\subsection{Constructing The Markov Chain}


\section{Conclusion}





\section{Conclusion}
The conclusion goes here.




% conference papers do not normally have an appendix


% use section* for acknowledgment
\section*{Acknowledgment}


The authors would like to thank...





% trigger a \newpage just before the given reference
% number - used to balance the columns on the last page
% adjust value as needed - may need to be readjusted if
% the document is modified later
%\IEEEtriggeratref{8}
% The "triggered" command can be changed if desired:
%\IEEEtriggercmd{\enlargethispage{-5in}}

% references section

% can use a bibliography generated by BibTeX as a .bbl file
% BibTeX documentation can be easily obtained at:
% http://mirror.ctan.org/biblio/bibtex/contrib/doc/
% The IEEEtran BibTeX style support page is at:
% http://www.michaelshell.org/tex/ieeetran/bibtex/
%\bibliographystyle{IEEEtran}
% argument is your BibTeX string definitions and bibliography database(s)
%\bibliography{IEEEabrv,../bib/paper}
%
% <OR> manually copy in the resultant .bbl file
% set second argument of \begin to the number of references
% (used to reserve space for the reference number labels box)
\begin{thebibliography}{1}

\bibitem{IEEEhowto:kopka}
H.~Kopka and P.~W. Daly, \emph{A Guide to \LaTeX}, 3rd~ed.\hskip 1em plus
  0.5em minus 0.4em\relax Harlow, England: Addison-Wesley, 1999.

\end{thebibliography}




% that's all folks
\end{document}


