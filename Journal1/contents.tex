\section{Introduction}
% Computer Society journal papers do something a tad strange with the very
% first section heading (almost always called "Introduction"). They place it
% ABOVE the main text! IEEEtran.cls currently does not do this for you.
% However, You can achieve this effect by making LaTeX jump through some
% hoops via something like:
%
%\ifCLASSOPTIONcompsoc
%  \noindent\raisebox{2\baselineskip}[0pt][0pt]%
%  {\parbox{\columnwidth}{\section{Introduction}\label{sec:introduction}%
%  \global\everypar=\everypar}}%
%  \vspace{-1\baselineskip}\vspace{-\parskip}\par
%\else
%  \section{Introduction}\label{sec:introduction}\par
%\fi
%
% Admittedly, this is a hack and may well be fragile, but seems to do the
% trick for me. Note the need to keep any \label that may be used right
% after \section in the above as the hack puts \section within a raised box.



% The very first letter is a 2 line initial drop letter followed
% by the rest of the first word in caps (small caps for compsoc).
% 
% form to use if the first word consists of a single letter:
% \IEEEPARstart{A}{demo} file is ....
% 
% form to use if you need the single drop letter followed by
% normal text (unknown if ever used by IEEE):
% \IEEEPARstart{A}{}demo file is ....
% 
% Some journals put the first two words in caps:
% \IEEEPARstart{T}{his demo} file is ....
% 
% Here we have the typical use of a "T" for an initial drop letter
% and "HIS" in caps to complete the first word.
\IEEEPARstart{F}{REEDM} (Future Renewable Electric Energy Delivery and Management) is a smart grid project focused on the future of the electrical grid.
Major proposed features of the FREEDM network include the solid state transformer, distributed local energy storage, and distributed local energy generation \cite{FREEDMMIGRATION}.
This vein of research emphasizes decentralizing the power grid, making it more reliable by distributing energy production resources.
Part of this design requires the system to operate in islanded mode, where portions of the distribution network are partitioned from each other.
The effects of these partitions are still not well understood.
This is particularly true in a distributed cyber-physical system, in which partitions may occur in both the cyber and physical domains.
Related work\cite{HARINI}\cite{TSG} has indicated that cyber faults can cause a physical system to apply unstable settings.

This work presents a distributed leader election algorithm and Markov model of that algorithm.
The presented algorithm maintains the Markov property for the observations of the leader despite omission failures
This approach to considering how a distributed system interacts during a fault condition allows for the creation of new techniques for managing a fault scenario in cyber-physical systems. 
This discussion presents an approach that utilizes Markov chain to model a system's grouping behavior.
These chains produce expectations of how long a system can be expected to stay in a particular state as well as how much time it will be able to spend coordinating and doing useful work over a period of time. 
Using these measures, the behavior of the control system for the physical devices can be adjusted to prevent faults.

PUT IN A LIST OF SECTIONS

%\IEEEpubidadjcol
% needed in second column of first page if using \IEEEpubid
\section{FREEDM DGI}

This study models the group management module of the FREEDM DGI.
The DGI is a smart grid operating system that organizes and coordinates power electronics.
It also negotiates contracts to deliver power to devices and regions that cannot effectively facilitate their own needs.
DGI leverages common distributed algorithms to control the power grid, making it an attractive target for modeling a distributed system.

The DGI software consists of a central component, known as the broker.
This broker is responsible for presenting a communication interface.
It also furnishes any common functionality the system's algorithms may need.
These algorithms, grouped into modules, work in concert to move power from areas of excess supply to excess demand.

DGI utilizes several modules to manage a distributed smart-grid system.
Group management, the focus of this work, implements a leader election algorithm to discover which processes are reachable within the cyber domain.
Other modules provide additional functionality, such as collecting global snapshots, negotiating the migrations, and giving commands to physical components.

DGI is a real-time system; certain actions (and reactions) involving power system components need to be completed within a pre-specified time-frame to keep the system stable.
It uses a round robin scheduler in which each module is given a predetermined window of execution which it may use to perform its duties.
When a module's time period expires, the next module in the line is allowed to execute. 
The start of each round is synchronized between systems.
Each DGI process will execute the same module at the same time.

\section{Group Management}
The DGI uses the leader election algorithm, ``Invitation Election Algorithm,'' written by Garcia-Molina \cite{INVITATIONELECTION}.
Originally published in 1982, this algorithm provides a robust election  procedure that allows for transient partitions.
Transient partitions are formed when a faulty link between two or more clusters of DGIs causes the groups to divide temporarily.
These transient partitions merge when the link becomes more reliable.
The election algorithm allows for failures that disconnect two distinct sub-networks.
These sub networks are fully connected, but connectivity between the two sub-networks is limited by an unreliable link.

Many election algorithms have been created. 
Each algorithm is designed to be well-suited to the circumstances it will deployed in.
Specialized algorithms exist for wireless sensor networks \cite{LE-WSN-1}\cite{LE-WSN-2}, and other special circumstances \cite{LE-SPECIALCIRCUMSTANCES-1}\cite{LE-SPECIALCIRCUMSTANCES-2}.
Work on leader elections has been incorporated into a variety of distributed frameworks: Isis \cite{ISISTOOLKIT}, Horus \cite{HORUSTOOLKIT}, Totem \cite{TOTEMTOOLKIT}, Transis \cite{TRANSISTOOLKIT}, and Spread \cite{SPREADTOOLKIT} all have methods for creating groups.
Despite this wide array of work, the fundamentals of leader election are similar across all implementations.
Processes arrive at a consensus of a single peer that coordinates the group.
Processes that fail are detected and removed from the group. 

The elected leader is responsible for making work assignments, and identifying and merging with other coordinators when they are found, as well as maintaining an up-to-date list of peers for the members of his group. 
Group members monitor the group leader by periodically checking if the group leader is still alive by sending a message. 
If the leader fails to respond, the querying nodes will enter a recovery state and operate alone until
they can identify another coordinator.
Therefore, a leader and each of the members maintain a set of processes which are currently reachable, a subset of all known processes in the system.

Leader election can also be classified as a failure detector \cite{LEADERELECTIONEVAL}.
Failure detectors are algorithms which detect the failure of processes within a system; they maintain a list of processes that they suspect have crashed.
This informal description gives the failure detector strong ties to the leader election process. 
The group management module maintains a list of suspected processes which can be determined from the set of all processes and the current membership.

The leader and the members have separate roles to play in the failure detection process.
Leaders use a periodic search to locate other leaders in order to merge groups.
This serves as a ping / response query for detecting failures within the system.
The member sends a query to its leader.
The member will only suspect the leader, and not the other processes in their group.

\section {Experimental Setup}

\subsection{Broker Architecture}
The DGI software used in this designed around a broker architectural specification.
Each core functionality of the system was implemented within a module that was provided access to core interfaces.
These interfaces provided functionality such as scheduling requests, message passing, and a framework to manipulate physical devices.
The Broker provided a common message passing interface that all modules could access.
This interface was then used to pass information between modules. 
For this purpose of this work, messages are sent as single UDP datagrams.
If a datagram is lost, it is not resent.
DGI expect an increasing sequence number on datagrams, which ensures message ordering.

\subsection{Algorithmic Changes}

The original Garcia-Molina algorithm has been modified so the observations of the coordinator process have the Markov property.
The full algorithm is presented below.
The rest of this section describes the changes that were made to the algorithm and shows how the combination of those changes allow the observations of the Coordinator to follow the Markov process.
The execution model of this algorithm assumes a real-time system using a round-robin scheduler.
All processes have their clock synchronized to a reasonable tolerance.
At a predetermined time and following predetermined intervals the algorithm executes at each process.
Using the synchronized clocks, all processes execute either $Check()$ or $Timeout()$ at the same time.
Processes can only form groups if their clocks are sufficiently in sync with another process' clock.

\begin{algorithmic}

\State $AllNodes \gets \{ 1, 2, ..., N \}$
\State $Coordinators \gets \emptyset$
\State $UpNodes \gets { Me }$
\State $State \gets Normal$
\State $Coordinator \gets Me$
\State $Responses \gets \emptyset$
\State $Counter \gets$ A random initial identifier
\State $GroupID \gets (Me,Counter)$

\State

\Function{Check}{}
    \State This function is called at the start of a round by the leader
    \If {$State = Normal$ and $Coordinator \gets Me$}
        \State $Responses \gets \emptyset$
        \State $TempSet \gets \emptyset$
        \For {$j = (AllNodes - \{Me\})$}
            \State $AreYouCoordinator(j)$
            \State $TempSet \gets TempSet \cup j$
        \EndFor
        \State Nodes which respond "Yes" to $AreYouCoordinator$ are put into the $Responses$ set.
        \State When an $AreYouThere$ response is "No" and this process is a coordinator, the querying process is put in the $Responses$ set.
        \State Wait for $Timeout(CheckTimeout)$, Nodes that do not respond are removed from UpNodes.
        \State $UpNodes \gets (TempSet-Responses) \cup {Me}$
        \If {$Responses = \emptyset$}
            \Return
        \EndIf
        \State $p \gets \max(Responses)$
        \If $Me > p$
            \State Wait time proportional to me-p
        \EndIf
        \Call{Merge}{Responses}
    \EndIf
    \State The next call to this is after Timeout(CheckTimeout)
\EndFunction

\State

\Function{Timeout}{}
    \State This function is called at the start of a round by the group members
    \If {$Coordinator = Me$}
        \Return
    \Else
        \Call{AreYouThere}{Coordinator,GroupID,Me}
        \If{Response is No}
            \Call{Recovery}{}
        \EndIf
    \EndIf
    \State The next call to this is after Timeout(TimeoutTimeout)
\EndFunction

\State

\Function{Merge}{Coordinators}
    \State This function invites all coordinators in Coordinators to join a group led by Me
    \State $State \gets Election$
    \State Stop work
    \State $Counter \gets Counter+1$
    \State $GroupID \gets (Me,Counter)$
    \State $Coordinator \gets Me$
    \State $TempSet \gets UpNodes - {Me}$
    \State $UpNodes \gets \emptyset$
    \For {$j \in Coordinators$}
        \Call{Invite}{j,Coordinator,GroupID}
    \EndFor
    \For {$j \in TempSet$}
        \Call{Invite}{j,Coordinator,GroupID}
    \EndFor
    \State Wait for $Timeout(InviteTimeout)$, Nodes that accept the invite are added to UpNodes
    \State $State \gets Reorganization$
    \For {$j \in UpNodes$}
        \Call{Ready}{j,Coordinator,GroupID,UpNodes}
    \EndFor
    \State $Acknowledge \gets UpNodes$
    \State Wait for $Timeout(ReadyTimeout)$, Nodes that do not acknowledge are removed from UpNodes
    \State $UpNodes \gets UpNodes - Acknowledge$
    \State $State \gets Normal$
\EndFunction

\State

\Function{ReceiveReady}{Sender,Leader, Identifier, Peers}
    \If {$State = Reorganization$ and $GroupID = Identifier$}
        \State $UpNodes \gets Peers$
        \State $State \gets Normal$
        \State Respond Ready Acknowledge 
    \EndIf
\EndFunction

\State

\Function{ReceiveAreYouCoordinator}{Sender}
    \If {$State = Normal$ and $Coordinator = Me$}
        \State Respond Yes
    \Else
        \State Respond No
    \EndIf
\EndFunction

\State

\Function{ReceiveAreYouThere}{Sender, Identifier}
    \If {$GroupID = Identifier$ and $Coordinator = Me$ and $Sender \in UpNodes$}
        \State Respond Yes
    \Else
        \State Respond No
        \State Add sender to $Responses$ set for $Check()$ if this process is a coordinator.
    \EndIf
\EndFunction

\State

\Function{ReceiveInvitation}{Sender,Leader,Identifier}
    \If {$State \neq Normal$}
        \Return
    \EndIf
    \If {$Sender \neq 0$}
        \Return
    \EndIf
    \State Stop Work
    \State $Temp \gets Coordinator$
    \State $TempSet \gets UpNodes$
    \State $State \gets Election$
    \State $Coordinator \gets Leader$
    \State $GroupID \gets Identifier$
    \If {$Temp = Me$}
        \State Forward invite to old group members
        \For $j \in TempSet$
            \State $Invite(j,Coordinator,GroupID)$
        \EndFor
    \EndIf
    \State $Accept(Coordinator,GroupID)$
    \State $State \gets Reorganization$
    \If {$Timeout(ReadyTimeout)$ expires before $Ready$ is received}
        \State $Recovery()$
    \EndIf
\EndFunction

\State

\Function{ReceiveAccept}{Sender,Leader,Identifier}
    \If {$State \gets Election$ and $GroupID = Identifier$ and $Coordinator = Leader$}
        \State $UpNodes \gets UpNodes \cup {Sender}$
    \EndIf
\EndFunction

\Function{ReceiveReadyAcknowledge}{Sender}
    \State $Sender$ is removed from $Acknowledge$ in $Merge()$
\EndFunction

\State

\Function{Recovery}{}
    \State $State \gets Election$
    \State Stop Work
    \State $Counter \gets Counter + 1$
    \State $GroupID \gets (Me,Counter)$
    \State $Coordinator \gets Me$
    \State $UpNodes \gets {Me}$
    \State $State \gets Reorganization$
    \State $State \gets Normal$
\EndFunction

\end{algorithmic}

In a distributed system information cannot be instantaneously spread throughout the system.
A process can only make local observations.
In this work, we attempt to model what a process will observe as a result of omission failure.
Therefore, it is important that observations that a process makes hold to the Markov property.
In the original algorithm, there are several portions that, when projected to the leader's observation, do not meet the Markov property.
The following sections state the portions of the algorithm where the observation of the leader process does not yield the probability of next transition.

Leader selection is performed a priori-- only process 0 may become a group coordinator.
Only process 0 may become the leader of a multiprocess group.
This simplification was applied because the configuration of the system with a larger number of processes depended on the configuration of the other processes.
Without this simplification, the state of the rest of the system would not have the memoryless property.
The state of the processes that are not in the observers group would change each round.
As a consequence the state of the rest of the system and the likelihood of forming a specific group size would change each step if other processes could become leader.

DIAGRAM HERE

The changes added a third message to completing an election -- a ready acknowledge message.
This message is sent by a member after receiving the ready message from the coordinator.
This allows the coordinator to be certain of the member's status before the next round.
Without the ready acknowledgment, the member may not receive the ready message and the coordinator will observe the member is a part of the group.
As a consequence of that uncertainty, the probability that a member remains in a group in the first round after an election has a different probability than each subsequent round.
By adding the extra message, the observation of the coordinator of the state, must be the state of the member of the group.
The sequence presented in figure XXX is not possible and as a consequence, the probability a member remains in a group in the first round after an election is a fixed value.

DIAGRAM HERE

Members cannot leave a group without the leader's permission.
Members do not suspect the coordinator has failed, only the coordinator may suspect the members.
For the purpose of starting an election, an Are You There message and it's negative response are considered equivalent to a Are You Coordinator message and a positive response.
On receipt of the negative response, the member will immediately recover and become a leader.
This assumption relies on Are You Coordinator and Are You There messages being sent at roughly the same time.

DIAGRAM HERE

This change leads to a live-lock situation in a crash failure, where the group's leader crashes and does not return and as a consequence the remaining members are trapped in a group without a leader.
For the purpose of this work, we have disregarded these live lock scenarios.
However, the live-lock could be avoided if the member can detect that it has not received an "Are You Coordinator" message in a round.
When the process fails to receive the message, the coordinator must have also removed the member from their group since they could not have received a "Are You Coordinator" response message.
Integrating this component is an area of future work.

\section{Formal Modeling}

\subsection{Assumptions}

All participating peers were assumed to be on the same schedule; all peers began executing the model simultaneously.
Synchronization was accomplished using Choi's work \cite{DCS}.
It was also assumed the clocks are synchronized. 
If the network has faulted, process clocks would not drift noticeably from their last synchronization.
As an alternative, a production system would likely use GPS time synchronization to obtain certain power system readings \cite{PHASORREADINGS}.

\subsection{Constructing The Markov Chain}


\section{Conclusion}



