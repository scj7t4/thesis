% Overview of the information flow properties and applications:

% THIS IS A DUMP OF THE CONTENTS OF THE PAPER FROM 4-25-2016 -- update me with edits.

\chapter{Information Flow Analysis of Distributed Computing}


\section{METHODS}

A model created for a distributed system must have sufficient information to be accurate.
The current state of a distributed system is difficult to obtain because memory spaces are typically isolated.
Without exact synchronization, an accurate global snapshot of the system cannot be taken.
Instead of attempting to capture exact global snapshots, our approach relies on allowing an agent to reason about the state of the other agents in the system.
By doing so, an agent can construct a model.
We propose the following for the execution environment of the distributed system:

\begin{itemize}
    \item Each agent has some set of logical variables which it manipulates as its algorithms execute.
    \item Each agent belongs to a domain unique to the agent (agent $i$ is the only member of the logical domain $D_i$).
    \item No agent can directly access a variable outside its domain.
    \item The authenticity of any information transfer (using modal operator $I_{i,j}$) is always trusted. However, the trust operator ($T_{i,j}$) is used to describe a message that is lost in transit: in all logical formulas presented, the trust operator describes the omission of a message.
    \item Agents do not exhibit Byzantine failure, nor do they crash; only messages may be omitted.
\end{itemize}

If no information is passed between agents, they are MSDND secure (ignoring any sort of leakage from interactions in the physical world).
As information is passed, aspects of the agent's state are leaked.
However, depending on when messages are sent and which messages are lost, the agent can be left in doubt as to the state of the other agent.
The two armies problem is a well-known thought experiment about the ability of multiple parties to reach consensus when there are omission faults.

\section{TWO ARMIES PROBLEM}

First, we will show that information flow analysis can be used to determine what state information is deducible to a particular agent in a system.
We use the common two armies problem as a starting point.
Using the MSDND security model, we will formally show which portions of the system state are leaked to an agent through message exchanges.

In the two armies problem, two agents, which are generals of their respective armies, must cooperate to attack an enemy city.
However, the two armies are physically separated by the enemy city and must send messengers to coordinate their plan.
Any messenger the generals send can be captured by the enemy city, preventing their message from being delivered.
The attack will fail if the generals do not make an agreement on when to attack.
The generals must come to an agreement when their channel for communication, a messenger, is unreliable.

After one message has been sent to one of the two generals, the state of the intended recipient is MSDND secure to the sender.
Let $\varphi_0$ be an \ac{AP} indicating ``General A will attack at dawn.''

\begin{thm}
If no messages are exchanged, the state of the two generals is mutually MSDND secure. \label{thm:nomsg}
\end{thm}
\begin{prooftight}
If no messages are exchanged and no information is leaked from the physical world, the two generals have no way of determining the other's state.
\end{prooftight}

\begin{thm}
Once at least one messenger delivers a message to one of the generals, one of the generals is not MSDND secure.
\end{thm}
\begin{prooftight}
Let $\{ \varphi_i : i \in 1,2 ... n \}$ describe the state that a general has received $\varphi_{i-1}$.
\begin{case}
One messenger is sent by General A and arrives at General B.
\label{case:generalsn0}
\end{case}

If no confirmations are sent, then General A clearly cannot deduce if General B has received the message.
To General A, General B is MSDND secure because General A has no way to valuate $B_B \varphi_0$.
However, if B believes A's message, then A is not MSDND secure to B, because B believes that $\varphi_0$ is true:

%\begin{table}[H]
%\centering
%\small
\begin{msdndproof}
1. & $\varphi_0$ & General A decides to attack at dawn. \\
2. & $I_{B,A} \varphi_0$ & General A sends a messenger to B informing them of their army's intent. \\
3. & $B_{B}I_{B,A} \varphi_0 \wedge T_{B,A} \varphi_0$ & General B believes the message from general A. \\
4. & $B_{B} \varphi_0$ & By C1. \\
5. & $B_{B} \varphi_0 \rightarrow \varphi_1$ & General B knows the plan. \\
6. & $w \vDash V_{\varphi_0}^{B}(w)$ & $V_{\varphi_0}^{B}(w)$ always returns true. \\
\end{msdndproof}
Therefore, A is not MSDND secure to B. However, $V_{\varphi_1}^{A}(w) \not \in \mathbb{V}$ , so B is secure to A.

\begin{case}
Any number of messengers are sent and deliver their messages, alternating from General A or General B to the other general. \label{case:generalsnn}
\end{case}
As each messenger arrives, the receiving general will trust the integrity of the message and believe its contents, resulting in that general assigning value to $\varphi_i$.

\begin{msdndproof}
% Revise me to have some dots in me so the last step resolves n.
1. & $B_{B} \varphi_0$ & Continuing from Case \ref{case:generalsn0}. \\
2. & $B_{B} \varphi_0 \rightarrow \varphi_1$ & General B decides to follow A's plan. \\
3. & $I_{A,B} \varphi_1$ & General B sends a messenger to A informing them of their army's intent. \\
4. & $B_{A}I_{A,B} \varphi_1 \wedge T_{A,B} \varphi_1$ & General A believes the message from general B. \\
5. & $B_{A}\varphi_1$ & By C1. \\
6. & $B_{A}\varphi_1 \rightarrow \varphi_2$ & General A agrees. \\
...& & The same logical chain repeats. \\
7. & $w \vDash V_{\varphi_n}^{x}(w)$ & $V_{\varphi_n}^{x}(w)$ is always true. $x$ is $A$ or $B$ depending on the value of n. %
\end{msdndproof}
Therefore, either A or B is not MSDND secure to the other for $\varphi_n$.

In the case $n=1$, B is now unsure that A has received $\varphi_1$ and cannot deduce if $B_{A} \varphi_1$.
B is unsure of A's state and, as a consequence, A is MSDND secure to B.
However, B is not MSDND secure to A because $\varphi_1$ is known to A. By extension, for $i=2,4...n$, B is secure to A, but not A to B. For $i=3,5...n$, A is secure to B, but not B to A.
\end{prooftight}

\begin{thm}
If a messenger is captured and the message is not resent, both agents will be secure on the last successfully delivered message $\varphi_{n}$ or $\varphi_0$ if the first messenger is captured.
\label{thm:captured}
\end{thm}
\begin{prooftight}
    If a messenger carrying a reply is captured, original sender cannot determine which message was captured.
\begin{case}
One messenger is sent and captured by the enemy.
\end{case}
%Although, the expectation of a messenger might leak something?
If the messenger does not arrive, it is equivalent to the messenger never being sent. (Theorem \ref{thm:nomsg})

\begin{case}
If $n-1$ messengers successfully deliver their message, but messenger $n$ is captured, both are secure on $\varphi_{n}$.
\end{case}
Suppose General A sends $\varphi_{n-1}$ to B.
On the delivery of the message $\varphi_{n-1}$ to B, the value of $\varphi_{n}$ is secure in B to A, as A has no way of knowing if $\varphi_{n-1}$ was delivered, unless B sends $\varphi_{n}$ with a messenger.
When B does send $\varphi_{n}$, the messenger never arrives.
As a consequence, General A has no way of assigning value to $B_A \varphi_n$ ($V_{\varphi_n}^A \not \in \mathbb{V}$).
However, as before, $\varphi_{n-1}$ at A is not secure to B.
\end{prooftight}

\section{BYZANTINE GENERALS}

If General A is attempting to coordinate with multiple armies, the problem becomes more complex.
If we extend the messenger analogy to cover faulty generals (ones sending incorrect information or omitting messengers), the generals can reach consensus if, for every faulty general, there are three generals that work correctly.\cite{byzantine-generals}

\begin{thm}
In any message exchange that conforms to the constraints of the Byzantine Generals problem, all agents are MSDND insecure on the plan $\varphi$.
\end{thm}
\begin{prooftight}
Suppose agent $i$ decides to use plan $\varphi$ to attack and there is some set of Byzantine generals $T$ and some set of loyal generals $G$ ($i \in G$). If $|G| > 3|T|$, the algorithm executes successfully and $B_x \varphi : \forall x \in G$. Therefore, every general in $G$ can valuate $\varphi$ and the variable is insecure.
\end{prooftight}

However, in a general omission model, the level of loyalty from processes may be impractical to achieve.
If the required number of ``loyal'' processes are not available, the system cannot achieve consensus.

\section{ELECTION IN AN ANONYMOUS COMPLETE NETWORK}

Consider a version of the ``coin-flipping'' leader election expressed in BIT logic.
The algorithm is functionally identical, but contains additional guards in the conditionals as an expression of the agents' knowledge.
Additionally, note the construct of labeling each $\varphi$ that an agent receives is simply assigning labels to make the algorithm easier to understand.
The algorithm only considers the complete collection of $\varphi$s for execution and not the source of any $\varphi$ the algorithm uses.

Let $\psi_i$ be the state where an agent $i$ completes the algorithm as the leader, and $\gamma_i$ as the state where an agent $i$ has terminated the algorithm.
The algorithm terminates when $\{\gamma_i = T : \forall i \}$ is satisfied and exactly one $\psi_i$ is true ($\sum_i \psi_i = 1$).

\begin{algorithm}
\caption{Anonymous Coin Flipping Leader Election Expressed in BIT logic}
\label{alg:coinflip}
\begin{algorithmic}[1]
\small
\State $\varphi_i \gets random(T,F)$
\State $\psi_i \gets F$
\State Send $\varphi_i$ to every active neighbor ($\forall j \neq i\ : I_j,i \varphi_i$)
\State Receive $\varphi_j$ from every active neighbor 
\If {$\varphi_i \wedge (\forall j \neq i, w \vDash V_{\varphi_j}^i(w) : \neg \varphi_j)$}
	\State $\psi_i \gets T$
	\State $\gamma_i \gets T$
\ElsIf {$(\varphi_i \wedge \exists j \neq i, w \vDash V_{\varphi_j}^i( \varphi_j)(w) : \varphi_j) \vee (\forall k$ s.t. $w \vDash V_{\varphi_k}^i(w) : \neg \varphi_k$}
	\State $\varphi_i \gets random(T,F)$
	\State Go to next round
\ElsIf {$\exists j \neq i, w \vDash V_{\varphi_j}^i(w) : \varphi_j$}
	\State $\gamma_i \gets T$
\EndIf
\end{algorithmic}
\end{algorithm}

\begin{thm}
Algorithm \ref{alg:coinflip} may not terminate correctly unless there is detectable, perfect information transfer to all parties in the algorithm.
\end{thm}
\begin{prooftight}
Let $i$ be the agent that would correctly terminate as the leader. Let $j$ be a process that has selected $\varphi_j = F$.

\begin{msdndproof}
1. & $\varphi_i = T, \varphi_j = F$ & Initial conditions. \\
2a.& $I_{j,i} \varphi_i$ & $i$ sends $\varphi_i$ to $j$. \\
2b.& $I_{i,j} \varphi_j$ & $j$ sends $\varphi_j$ to $i$. \\
3a.& $\neg (B_j I_{j,i} \varphi_i \wedge T_{j,i} \varphi_j)$ & $j$ does not receive $\varphi_i$. \\
3b.& $B_i I_{i,j} \varphi_j \wedge T_{i,j} \varphi_i$ & $i$ receives $\varphi_j$. \\
4a.& $w \not \vDash V_{\varphi_i}{j}(w)$ & $j$ cannot valuate $\varphi_i$. \\
4b.& $B_i \varphi_j$ & By C1. \\
5a. & $\not \exists V_{\varphi_i}^j(w) \wedge \neg \varphi_j \rightarrow (\varphi \gets T)$ & $j$ Cannot determine that $i$ can terminate and incorrectly changes $\varphi$. \\
5b. & $\varphi_i \wedge w \vDash V_{\varphi_j}^i(w) \wedge \neg \varphi_j \rightarrow \psi_i$ & $i$ incorrectly terminates as the leader. \\
\end{msdndproof} Therefore, $i$ will terminate before $j$ decides to go passive.
\end{prooftight}
This result agrees with results from similar analysis\cite{anon-omission}.

\section{MODEL CONSTRUCTION FOR THE TWO ARMIES PROBLEM}
Suppose for the two armies problem, A and B wish determine their chance of success.
If neither A nor B have any idea of the probability their messenger is captured, they have no way of assessing their chances of a successful attack.
However, what can be determined if the two generals share some knowledge about the world?

Suppose that both generals know their messenger has the probability $q=(1-p)$ of being captured in transit.
When A sends a messenger to B, A will know there is a probability $p$ that the messenger arrives at B.
If the messenger does indeed arrive at B, B knows that A knows there is a $p$ chance their attack will succeed.
In this situation, both generals can construct identical models of the success the attack, given the initial message arrives.
It is also worth noting in this scenario, no additional messages can improve the outlook of the two generals.
Each additional message only confirms there is a probability $p$ the other general will also attack.

\subsection{Generalization}

The idea of converting a Kripke model to a Markov model is mentioned in \cite{kripkemarkov1} and \cite{kripkemarkov2}.
However, we could not locate a formal description of the method, so we present one here:
given a Kripke Model $K = <W,R,\mathbb{V}>$, where we obligate $K$ to contain sufficient state information to ensure the probability mapping is memoryless.
Let $f_{W} : W \rightarrow X$ be a bijective mapping from $W$ to $X$, where $X$ is the set of states for a Markov model.
Let $f_{R}$ be a mapping for a relationship $wRw' \in R$ to $\Pr(X_{i+1}=f_W(w') | X_{i}=f_W(w))$.
Therefore, for a Kripke-Markov model $P_{ij} = f_R(w,w')$ where $i=f_W(w)$ and $j=f_W(w')$.

For a process $i$, information about the system state is restricted to those \ac{AP}s in its domain, $D_i$.
Let $O : D \times X \rightarrow Y$ be a surjective mapping from the complete, hidden system state to an observable state for a domain.
With $O$, one can construct $B$, the observation probability distribution matrix for a \ac{HMM}:
\begin{equation}
 B_{ij} =
   \begin{cases}
    1, & \text{if } O(D_i, x_i) = y_j\\
    0, & \text{otherwise.} \\
  \end{cases}
\end{equation}

Intuitively, one or more worlds will look identical to a process $i$, since the information necessary to differentiate them is not in $D_i$.
As a side effect of $O$ being a deterministic, surjective mapping, a special case exists where the observed state has the Markov property:
\begin{equation}
    \Pr(Y_{k+1}=a | Y_{k}=b) = \sum_{\{x_i | O(D_x, x_i) = a\}} \Pr(X_{k+1}=x_i | X_{k}=x_j), \forall x_j \in \{x_j | O(D_x, x_j) = b\}.
\label{eq:hidden-to-chain}
\end{equation}

The observed probability of going from $Y_k$ to $Y_{k+1}$ is the probability that the hidden state $X_k$ transitions to a state $X_{k+1}$ such that the observation for $X_{k+1}$ is $Y_{k+1}$.
It must also hold that the observed probability does not depend on the hidden state: any $X_k$ that results in an observation $Y_k$ must have the same probability of transitioning to $Y_{k+1}$ as the other $X_{k}$s that yield the same observation.

\subsection{State Determination}

When an agent uses the information transfer operator ($I_{i,j}$) to pass information to another agent in the system, it intends for that agent to believe the passed wff.
When an agent distributes a wff to many agents with the information transfer operator, it leads to a set of beliefs about the beliefs of the receiving agents.
The set, $N_i$, is the set of beliefs agent $i$ can have if all the wff it passed to the other agents are believed.
For example, if an agent distributes a wff $\varphi$ to a set of agents $Ag$ ($i \not \in Ag$), then $N_i = \{ B_i B_j \varphi : \forall j \in Ag \}$.
Since the belief of each $B_j \varphi$ is outside of the domain $D_i$, the agent $i$ can only valuate a wff in $N_i$ which has been leaked to $i$.

Let the set $L_i$ be the subset of $N_i$ for which a valuation function exists in a domain $i$.
$L_i$ can be populated either by direct information transfer or information leakage from interactions with agents.
We can similarly define a set $M_i$ which is the subset of $N_i$ and superset of $L_i$.
The values in $M_i$ correspond to the beliefs of agent $i$ if $i$ had perfect knowledge of the beliefs other agents ($L_i \subseteq M_i \subseteq N_i$).

\begin{thm}
Each member of $L_i$ is MSDND insecure to $i$.
\end{thm}
\begin{prooftight}
Each wff in $L_i$ has a valuation function in the domain $i$.

\begin{msdndproof}
1. & $I_{j,i} \varphi$ & $i$ informs some agent $j$ of $\varphi$. \\
2. & $B_{j}I_{j,i} \varphi \wedge T_{j,i} \varphi$ & $j$ receives $\varphi$ and believes its authenticity. \\
3  & $B_j \varphi$ & By C1. \\
4. & $I_{i,j} \varphi_{ack}$ & $j$ acknowledges $B_j \varphi$. \\
5. & $B_{i}I_{j,i} \varphi_{ack} \wedge T_{i,j} \varphi_{ack}$ & $i$ receives $\varphi_{ack}$. \\
6. & $B_{i} \varphi_{ack}$ & By C1. \\
7. & $B_{i} \varphi_{ack} \rightarrow B_i B_j \varphi$ & Because $j$ acknowledged $\varphi$, $i$ believes $j$ believes $\varphi$. \\
8. & $w \vDash V_{B_i B_j \varphi}^{i}(w)$ & $i$ does believe $\varphi$. \\
9. & $w \vDash V_{B_j \varphi}^{j}(w)$ & Is always true. %
\end{msdndproof}
Therefore, $j \in L_i$, and $B_j \varphi$ is MSDND insecure to $i$.
\end{prooftight}

\begin{thm}
Each wff in $M_i$ and $N_i$ but not $L_i$ is MSDND secure to $i$.
\end{thm}
\begin{prooftight}
Each wff in $M_i$ and $N_i$ have no valuation in the domain $i$.
\begin{case}
The case where $j$ receives some wff $\varphi$ and is in $M_i$ but not $L_i$.
\end{case}
\begin{msdndproof}
1. & $I_{j,i} \varphi$ & $i$ informs some agent $j$ of $\varphi$. \\
2. & $B_{j}I_{j,i} \varphi \wedge T_{j,i} \varphi$ & $j$ receives $\varphi$ and believes its authenticity. \\
3  & $B_j \varphi$ & By C1. \\
4. & $I_{i,j} \varphi_{ack}$ & $j$ acknowledges $B_j \varphi$. \\
5. & $\neg(B_{i}I_{j,i} \varphi_{ack} \wedge T_{i,j} \varphi_{ack})$ & $i$ does not receive $\varphi_{ack}$. \\
6. & $w \neg \vDash V_{\varphi_ack}^{i}(w)$ & $i$ is uncertain if $B_j \varphi$. \\
7. & $w \vDash V_{\varphi}^{j}(w)$ & Is always true. %
\end{msdndproof}
Therefore, $j \in M_i$, and $B_j \varphi$ is MSDND secure to $i$.

\begin{case}
The case where $j$ does not receive some wff $\varphi$ and is in $N_i$ but not $M_i$.
\end{case}
\begin{msdndproof}
1. & $I_{j,i} \varphi$ & $i$ informs some agent $j$ of $\varphi$. \\
2. & $\neg(B_{j}I_{j,i} \varphi \wedge T_{j,i} \varphi)$ & $j$ does not receive $\varphi$. \\
3. & $w \not \vDash V_{B_j \varphi}^{i}(w)$ & $i$ is uncertain if $B_j \varphi$. \\
4. & $w \not \vDash V_{\varphi}^{j}(w)$ & $j$ is uncertain of $\varphi$. %
\end{msdndproof}
Therefore, $j \in N_i$, and $B_j \varphi$ is MSDND secure to $i$.
\end{prooftight}
We assumed any beliefs an agent held stem from information transfer from another agent.
Therefore, we can assert the beliefs in $L_i$ for any process $i$ must have a traceable history derived from a process having a valuation for the referenced wff that aligns with the belief process $i$ holds about the wff.
Furthermore, wff in $N_i$ but not $L_i$ do not have valuation in domain $D_i$ and cannot be used in an observation.
