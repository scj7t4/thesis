% Summary of the paper, and why the results presented matter.

\chapter{Conclusion}

We presented a useful framework for reasoning about creating models of distributed systems that tolerate omission failure.
The models and structures presented allow algorithm designers to design algorithms whose behavior can be modeled with a Markov chain which is invaluable for designing critical infrastructure systems that behave reliably during failure conditions.

To do this, applied existing information flow analysis techniques and applied them to a common distributed systems problem (the two armies problem) and show the information flow analysis was consistent with similar analysis.
We then extended the analysis to show how agents in a distributed system could use that analysis to determine what knowledge each process had.
As part of this we described belief sets created by distributing information to several other agents in the system.

Additionally, we defined how information being transferred between agents and the actions they take based on that information could be constrained to hold to the memorylessness property common to Markov chains.
Using this concept we demonstrated how a common leader election algorithm could be modified to use this memorylessness property, allowing it to be modeled online during changing conditions.

This work is particularly valuable for the analysis of critical infrastructure systems, where knowledge of their behavior during fault conditions is particularly important.
By allowing the ability for algorithms to determine what issues are likely to arise while they are operating, actions can be taken to protect the infrastructure from failure.
There are a wide range of possible applications, including actions either undertaken by human operators on site, or autonomous actions taken by the algorithms to harden themselves against failure.

We also presented a technique for hardening a real-time distributed cyber-physical system against network congestion.
The \ac{RED} queueing algorithm and an out-of-band version of explicit congestion notification (ECN) were used to signal an application of congestion.
Using this technique the application changed several of its characteristics to ready itself for the increased message delays caused by the congestion.

These techniques were demonstrated on the \ac{DGI}, a distributed control system for the \ac{FREEDM} smart-grid project.
In particular, this paper demonstrated the hardening techniques were effective in keeping the \ac{DGI} processes grouped together.
Additionally, it helped ensure the changes applied to the \ac{DGI} through cyber-coordinated actions did not destabilize the physical power network.

This technique will be important to create a robust, reliable \ac{CPS} for managing future smart-grids.
However, this technique could potentially be applied to any \ac{CPS} that could experience congestion on its network, as long as it has the flexibility to change its operating mode.
Potential applications can apply to both the cyber control network and the physically controlled process.
For example, in a \ac{VANET} system, the vehicles could react to congestion by increasing their following distance.
