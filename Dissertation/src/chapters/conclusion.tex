% Summary of the paper, and why the results presented matter.

\chapter{Conclusion}

This work presented a new approach for predicting the behavior of a real-time distributed system under omission failure conditions.
By using a continuous time Markov chain, a variety of insights can be gathered about the system, including observations such as how long a particular configuration will be stable, and the behavior of the system in the long run. 
The Markov results will be used  to make better real time schedules to better react to the network faults introducing in hardware-in-the-loop test-beds.
The primary concern are scenarios in which the cyber controller attempts to make physical components which are not connected in the physical network interact, and scenarios where a fault in the cyber network causes the paired events (where two physical controllers change to accomplish some transaction or exchange) to only be partially executed.
For example, in the DGI load balancing scheme, a node in a supply state injects a quantum of power into the physical network, but the node in the demand state does not change to accept it.
These errors, which are the primary focus of this work could cause instability if a sufficient number of these failed exchanges occur. In \cite{HARINI}, Choudhari et. al. show that failed transactions can create a scenario where the frequency of a power system could become unstable. 

In this work, we presented a technique for hardening a real-time distributed cyber-physical system against network congestion.
The \ac{RED} queueing algorithm and an out-of-band version of explicit congestion notification (ECN) were used to signal an application of congestion.
Using this technique the application changed several of its characteristics to ready itself for the increased message delays caused by the congestion.

These techniques were demonstrated on the \ac{DGI}, a distributed control system for the \ac{FREEDM} smart-grid project.
In particular, this paper demonstrated the hardening techniques were effective in keeping the \ac{DGI} processes grouped together.
Additionally, it helped ensure the changes applied to the \ac{DGI} through cyber-coordinated actions did not destabilize the physical power network.

This technique will be important to create a robust, reliable \ac{CPS} for managing future smart-grids.
However, this technique could potentially be applied to any \ac{CPS} that could experience congestion on its network, as long as it has the flexibility to change its operating mode.
Potential applications can apply to both the cyber control network and the physically controlled process.
For example, in a \ac{VANET} system, the vehicles could react to congestion by increasing their following distance.
