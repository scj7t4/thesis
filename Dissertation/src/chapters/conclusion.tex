% Summary of the paper, and why the results presented matter.

\chapter{Conclusion}

We presented a useful framework for reasoning about distributed systems that tolerate omission faults.
The models and structures presented in the framework allow algorithms to be designed with behaviors that can be modeled with a Markov chain.

To create the framework, existing information flow analysis techniques were applied to common distributed systems problems.
We showed the information flow-based analysis was consistent with previous work.
The analysis was extended to reason about a leader election algorithm.
We described belief sets created by distributing information to several other agents in the system and showed which portions were MSDND secure.

Additionally, we defined how information being transferred between agents and the actions they take based on that information could be modified to have the memorylessness property of Markov chains.
Using this concept we demonstrated how a common leader election algorithm could be modified to use this memorylessness property, allowing it to be modeled online during changing conditions.

Our work is particularly valuable for the analysis of critical infrastructure systems, where knowledge of their behavior during fault conditions is important.
By allowing the ability for algorithms to determine what issues are likely to arise while they are operating, actions can be taken to protect the infrastructure from failure.
There are a wide range of possible applications, including actions either undertaken by human operators on site, or autonomous actions taken by the algorithms to harden themselves against failure.
We presented a technique for hardening a real-time, distributed cyber-physical system against network congestion.
The \ac{RED} queueing algorithm and an out-of-band version of explicit congestion notification (ECN) were used to signal an application of congestion.
Using this technique, the application changed several of its characteristics to ready itself for the increased message delays caused by the congestion.

Techniques were demonstrated on the \ac{DGI}, a distributed control system for the \ac{FREEDM} smart-grid project.
We showed the hardening techniques were effective in keeping the \ac{DGI} processes grouped together.
Additionally, the changes applied to the \ac{DGI} through cyber-coordinated actions helped prevent potential destabilization of the physical power network.
Our techniques could be applied to any \ac{CPS} that could experience congestion on its network, as long as it has the flexibility to change its operating mode.
Potential applications can apply to both the cyber control network and the physically controlled process.
