% Describe related work in this area


\chapter{Related Work}

\section{Analysis of Distributed Systems}

\cite{markov-distributed} focused on examining the emergent behavior as a collection of processes in a grid computing scenario work to process a large dataset.
In their work, the authors focus on analysis derived from a discrete time Markov chain describing a single process completing a task.
This chain describes how the process in the grid based system goes through the steps of the process acquiring a task to do, working on that task and subsequently either completing the task or failing.
The created chains were ``absorbing'' chains, meaning that it had one or more states that the process could not leave once it had arrived in those states (task completed and task failed, for obvious reasons).
Their analysis used minimal s-t cuts to determine critical paths for the ideal operation of the system.
Using this analysis, they considered the Markov chain as a max-flow min-cut problem using the task complete or task failed absorbing states as sinks.
By identifying the ``critical transitions'' of the graph, the areas that would most greatly affect the performance of the system could be identified.

\cite{LEADERELECTIONEVAL} studies an Omega class failure detector using OmNet++\cite{OMNET}, a network simulation software package. Instead of omission failures, however, it considers crash failures. Each configuration goes through a predefined sequence of crash failures. OmNet++ is used to count the number of messages sent by each of three different leader election algorithms. Additionally, \cite{LEADERELECTIONEVAL} only considers the system to be in a complete and active state when all participants have consensus on a single leader.

\section{Physical Faults Caused by Cyber Entities in CPS}

Faults in \ac{CPS} can originate from many locations.
First and most obviously, the traditional physical system being augmented by the CPS is subject to its own failures, either from component failure or the actions of an attacker.
Secondly, the \ac{CPS} must employ sensors to detect the state of the physical components in the system.
Again, like the physical components, these sensors are subject to component failure or the actions of an attacker.
Lastly, if the \ac{CPS} communicates between entities using a communication network, the network can be disrupted by any number of issues, including DDoS, attackers, or congestion caused by other users in a shared network.

Several works have shown, that for a computer controlled smart grid, that failures originating at sensors or the communication network have the potential to cause the cyber entities controlling the physical networks to take incorrect actions.
Work has been undertaken to identify faulty sensing components in a network, but the identification of bad sensing equipment may not always be possible.
Additionally, it is not always possible to identify if the origin of an issue is a faulty sensor or an outside attacker.

If the cyber entity itself has been compromised, it could potentially exhibit Byzantine behavior, causing it to try and trick other components into bad actions, or it may try to disrupt the physical network directly.
Work has been done to try and identify when an entity in a cyber network is actively working to compromise the physical network by using the underlying physical invariants.
However, even if a cyber entity is trying to behave correctly, disruptions to the communication network or the sensors it uses can cause it to take actions similar to a process that is actively attempting to destabilize the system.
As before, these actions can be identified by using the underlying invariants of the physical network after they have occurred.
However, it would be ideal to avoid situations where a ``good'' entity is forced to act badly.

