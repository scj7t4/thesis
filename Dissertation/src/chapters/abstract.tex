
Cyber-physical systems (CPS) are an attractive option for future development of critical infrastructure systems.
By supplementing the traditional physical network with cyber control, the performance and reliability of the system can be increased.
In these networks, distributing the cyber control offers increased redundancy and availability during fault conditions.
However, there are very few works which study the effects of cyber faults on a  distributed cyber-physical system.
These are of a particular interest in the smart grid environment where outages and failures are very costly.
By examining the behavior of a distributed system under fault scenarios, the overall robustness of the system can be improved.
Designers can plan characteristics and responses to faults that allow the system to continue operating in difficult circumstances.

This work presents a techinque for modelling the availability of a cyberphysical system.
This is of high value because there is a limited amount of research that describes that models the availibility of distributed systems, and CPSs in general
This approach encodes the behavior of a dynamic reconfiguration system of a CPS as a continous-time Markov chain.
These models provide an estimate of when a cyberphysical system will be able to do work and when it is reconfiguring.
Generated models are presented and evaluted against data collected from actual runs of the system.
Using these models, we will be able to develop methods and gaurantees to quantify and protect operation of a CPS.
