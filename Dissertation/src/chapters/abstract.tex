
\ac{CPS} are an attractive option for future development of critical infrastructure systems.
By supplementing the traditional physical network with cyber control, the availability of the system can be improved.
In these networks, distributing the cyber control offers increased redundancy and availability during fault conditions.
However, there are very few works which study the effects of cyber faults on a distributed \ac{CPS}.
These are of a particular interest in the smart grid environment where outages and failures are very costly.
By examining the behavior of a distributed system under fault scenarios, the overall robustness of the system can be improved.
Designers can plan characteristics and responses to faults that allow the system to continue operating in difficult circumstances.

This work presents a technique for modeling the availability of a \ac{CPS}.
This is of high value because there is a limited amount of research that describes that models the availability of distributed systems, and \ac{CPS} in general
This approach encodes the behavior of a dynamic reconfiguration system of a CPS as a continuous-time Markov chain.
These models provide an estimate of when a \ac{CPS} will be able to do work and when it is reconfiguring.
Generated models are presented and evaluated against data collected from actual runs of the system.
Using these models, we will be able to develop methods and guarantees to quantify and protect operation of a CPS.
