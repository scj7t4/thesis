For a system with a heirarchical ordering of processes determining what memberships are allowed and their associated likelihood are listed in Table \ref{tab:membership-probs}.

\begin{table}
\centering
\caption{Possible Membership Changes in a Leader Election Algorithm}
\label{tab:membership-probs}
\begin{tabular}{l l l}
$\Pr(A \rightarrow A | A \in A)$ & & \\ \hline
$\Pr(B \rightarrow A | A \in A)$ & $\Pr(B \rightarrow B | B \in A)$ & \\
$\Pr(B \rightarrow A | B \in B)$ & $\Pr(B \rightarrow B | B \in B)$ & \\ \hline
$\Pr(C \rightarrow A | C \in A)$ & $\Pr(C \rightarrow A | C \in B)$ & $\Pr(C \rightarrow A | C \in C)$ \\
$\Pr(C \rightarrow B | C \in A)$ & $\Pr(C \rightarrow B | C \in B)$ & $\Pr(C \rightarrow B | C \in C)$ \\
$\Pr(C \rightarrow C | C \in A)$ & $\Pr(C \rightarrow C | C \in B)$ & $\Pr(C \rightarrow C | C \in C)$ \\ \hline
... & & \\
\end{tabular}
\end{table}

To fullfill the requirements of Equations \ref{eq:mod-req-a} and \ref{eq:mod-req-b}:

\begin{equation}
\begin{split}
\Pr(C \rightarrow B| C \in A) &= \Pr(C \rightarrow B) \\
\Pr(C \rightarrow B| C \in C) &= \Pr(C \rightarrow B) \\
\Pr(C \rightarrow A| C \in B) &= \Pr(C \rightarrow A) \\
\Pr(C \rightarrow A | C \in C) &= \Pr(C \rightarrow A) \\
\end{split}
\end{equation}
\begin{equation}
\begin{split}
\Pr(C \rightarrow C | C \in A) + \Pr(C \rightarrow A | C \in A) &= \Pr(C \rightarrow C) + \Pr(C \rightarrow A) \\
\Pr(C \rightarrow C | C \in C) + \Pr(C \rightarrow A | C \in C) &= \Pr(C \rightarrow C) + \Pr(C \rightarrow A) \\
\Pr(C \rightarrow  B | C \in B) + \Pr(C \rightarrow C | C \in B) &= \Pr(C \rightarrow C) + \Pr(C \rightarrow B) \\
\Pr(C \rightarrow B | C \in C) + \Pr(C \rightarrow C | C \in C) &= \Pr(C \rightarrow C) + \Pr(C \rightarrow B)
\end{split}
\end{equation}

Via subsitution:

\begin{equation}
\begin{split}
\Pr(C \rightarrow B) + \Pr(C \rightarrow C | C \in C) &= \Pr(C \rightarrow C) + \Pr(C \rightarrow B) \\
\Rightarrow \Pr(C \rightarrow C | C \in C) &= \Pr(C \rightarrow C).
\end{split}
\end{equation}

If process B can construct a Markov chain the probability of a process joining B's group must be independent of that processes' current state.
If that is the case, the ``Markov chain'' ceases to become a Markov chain since the probability of observing a state is the probability a set of processes join B's group.
For example, the probability that B observes C joining its group is:

\begin{equation}
\Pr(Y_{k+1}) = \Pr(C->B).
\end{equation}

The probability of C joining B's group is independent of C's state before the election.
In Equation \ref{eq:mod-req-b} every possible transition that is not of the form $\Pr(i \rightarrow i | i \in i)$ is constrained by an Equation \ref{eq:mod-req-a}.
Observing those conditions:

\begin{equation}
\Pr(Y_{k+1}) = \Pr(Y_{k+1} | Y_k)
\end{equation}

which is simply a probability distribution.
The presented algorithm does not fulfill this obligation for any process except the highest priority process and cannot generate Markov chain for those processes.

Every possible membership change must have a related Equation \ref{eq:mod-req-a}.