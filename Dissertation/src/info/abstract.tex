% Overview of project.
% What I accomplised, what my objectives were
New research about cyber-physical systems is rapidly changing the way we think about critical infrastructures such as the power grid.
Changing requirements for the generation, storage, and availability of power are all driving the development of the smart-grid.
Many smart-grid projects disperse power generation across a wide area and control devices with a distributed system. 
However, in a distributed system, the state of processes is hard to determine due to isolation of memory.
By using information flow security models, we reason about a process's beliefs of the system state in a distributed system.
Information flow analysis aided in the creation of Markov models for the expected behavior of a cyber controller in a smart-grid system using a communication network with omission faults.
The models were used as part of an analysis of the distributed system behavior when there are communication faults.
With insight gained from these models, existing congestion management techniques were extended to create a feedback mechanism, allowing the cyber-physical system to better react to issues in the communication network.
