%
% S&T Thesis Template
%

% @author Stephen Curtis Jackson <scj7t4@mst.edu>
% @author Dr. Kelly Homan

% User's E-mail List:
% S&T Thesis/Dissertation LaTeX Template User Group <latex-ogs-grp@mst.edu>

% Dissertation guidelines:
% http://grad.mst.edu/currentstudents/thesesdissertations/

\documentclass[times,12pt]{mstthesis}
\doublespacing

% Load all the requisite packages
% These files allow you to change the behavior of certain packages
% or the way LaTeX typesets things, grouped by what is being configured
% Configures: footnotes
%
% Makes footnotes a symbol instead of a number
\renewcommand{\thefootnote}{\fnsymbol{footnote}}
 % Configure footnote style
% Configures: hypenation
%
% Custom hypenation for words LaTeX gets wrong

%\hyphenation{local-ization diff-usion}

% \brokenpenalty is meant to eliminate hyphenation of words at the end of a page. However, it may introduce non-uniform pages
% https://groups.google.com/d/msg/comp.text.tex/pBSOMuQzOH0/innVfjc3ZG4J
%\brokenpenalty=10000
% Or
\brokenpenalty10000\relax
% having this causes page numbers to be incorrect in ToC. Not having it causes word breaks at end of page
% Another solution -- may not fix hyphenation at end of page. 
\raggedbottom
 % Custom hypenation options
% Configures: Paragraphs
%
% Options in this file influence how latex sets up paragraphs. These influence how many
% lines can be orphaned at the top or bottom of a page.

% \windowpenalty prevents the first line of a paragraph from being alone at the bottom of a page.
% It will either cause that line to move to the next page, or cause one line from the next page to join the lone line. 
\widowpenalty=10000
% \clubpenalty prevents the last line of a paragraph to be alone at the top of a page. At best it will force there to be two lines.
\clubpenalty=10000   
% For more info: http://en.wikibooks.org/wiki/LaTeX/Page_Layout#Widows_and_orphans
% MS&T Office of Grad Studies says ``Need to have at least 2 lines at the top and at the bottom of every page. No lone headers at bottom of page.
% see https://groups.google.com/d/msg/comp.text.tex/uN7tvjubk8o/kcUvevut0hEJ
 % Define the behavior of typesetting paragraphs

% The build script will process all the .tex files in the packages directory
% And generate this file as part of the build process.
% It will automatically include all the .tex files into final document at
% this point.
\input{packages/__packages.generated}

\makeindex
\makeglossary  


\begin{document}

\setlength{\parskip}{0pt}
\setlength{\parindent}{0.5in}


% Set to ms or phd
\begin{ThesisTitlePage}{phd}
% You will also need to edit main.tex to select between phd & ms
% Look for where this file is input

\author{\MakeUppercase{Stephen Curtis Jackson}}

\thesistitle{Models of Leader Elections and Their Applications}

\department{Computer Science}

\degree{Doctor of Philosophy}

% Committe
\ThesisAdviser{Dr. Bruce McMillin}
% If you have a second advisor, put their name in this command
% \cothadviser{} 
\memberone{Dr. One}
\membertwo{Dr. Two}
\memberthree{Dr. Three}
\memberfour{Dr. Four}


% Graduation date.  NOT your submission date! ...
\graddate{2016}

\end{ThesisTitlePage}

% Specifiy any copyright state you'd like in this file.

\copyrightyear{2016}
\ThesisCopyrightPage{true}


% If you are doing the publication option:
%\begin{ThesisPublicationOption}{ms}
%This dissertation has been prepared in the form of six papers.
This thesis has been prepared in the form of two papers.

\renewcommand{\theenumi}{Paper \arabic{enumi}}
\begin{enumerate}%[Paper 1]

%  http://tex.stackexchange.com/questions/38139/how-can-i-calculate-the-difference-of-2-counters-pageref
% \newcounter{tempor}
% \setcounter{tempor}{\pageref{paper:1_start}}
% \addtocounter{tempor}{1}
%\item Pages~\arabic{temp}--\pageref{paper:1_end} have been published as
\item Pages~\pageref{paper:1_start}--\pageref{paper:1_end} have been submitted as 
  \textit{Improving XCS Scalability with Automatic Problem Decomposition} to the Sixteenth International Workshop on
Learning Classifier Systems (2013) with Daniel R. Tauritz and Samuel A. Mulder.

\item Pages~\pageref{paper:2_start}--\pageref{paper:2_end} have been submitted as 
  \textit{Regression Testing for Model Transformations: A Multi-Objective Approach} to Symposium on Search-based Software Engineering (2013) with Marouane Kessentini and Daniel R. Tauritz.
\end{enumerate}
\renewcommand{\theenumi}{\arabic{enumi}}

%An earlier letter, 
%\textit{\href{http://dx.doi.org/10.1016/j.physb.2010.01.025}{Criterion for light localization in random %amplifying media}}, 
%has been published in Physica B \textbf{405}, 3012--3015 (2010) with Hui Cao and Alexey Yamilov. %This is reported in Paper 1.

\end{ThesisPublicationOption}


% Abstract
\begin{ThesisAbstract}
\begin{abstract}
Cyber-physical systems (CPS) are an attractive option for future development of
critical infrastructure systems. By supplementing the traditional physical
network with cyber control, the performance and reliability of the system
can be increased. In some of these networks, distributing the cyber control
offers increased redundancy and availability during fault conditions. However,
there are very few works which study the effects of cyber faults on a 
distributed cyber-physical system. These are of a particular interest in the smart grid
environment where outages and failures are very costly. By examining the
behavior of a distributed system under fault scenarios, the overall robustness
of the system can be improved by planning characteristics and responses to
faults that allow the system to continue operating in difficult circumstances.

This work examines the consequences of network unreliability on a core part of the
 Distributed Grid Intelligence (DGI) for the FREEDM (Future Renewable Electric 
Energy Delivery and Management) Project. By applying different rates of packet
loss in specific configurations to the communication stack of the software and
observe the behavior of a critical component (Group Management) under those
conditions. These components identify the amount of time spent in a group,
working, as a function of the network reliability. Given this one can parameterize
the amount of messages that are lost and thus the number of failed physical actuations.
Knowing this and the physical characteristics of the system, one can tune the amount
of time between reconfiguration in order to prevent the number of failed physical
changes from causing the system to become unstable. Using this, we can develop
methods and guarantees to protect the operation of a CPS.
\end{abstract}

\end{ThesisAbstract}

% Acknowledgements
\begin{ThesisAcknowledgment}
The author acknowledges the support of the Future Renewable
Electric Energy Delivery and Management Center,
a National Science Foundation supported Engineering Research
Center under grant NSF EEC-081212, and the United States Department of Education GAANN program.

\end{ThesisAcknowledgment}

% Table of contents and list of figures, tables etc.
\begin{ThesisFrontMatter}
\tableofcontents
\listoffigures
\listoftables
\newpage
\addcontentsline{toc}{chapter}{LIST OF ACRONYMS}
\chapter*{LIST OF ACRONYMS}
\singlespacing\setlength{\parskip}{0pt}
\begin{acronym}
\acro{AGS}{average group size}
\acro{AP}{atomic proposition}
\acro{AYC}{Are You Coordinator}
\acro{AYT}{Are You There}
\acro{BIT}{Belief, Information transfer, Trust}
\acro{CDF}{Cumulative Distribution Function}
\acro{CPS}{Cyber-physical systems} 
\acro{CTMC}{Continuous Time Markov Chain}
\acro{CV}{critical value}
\acro{DF}{degrees of freedom}
\acro{DGI}{Distributed Grid Intelligence}
\acro{DNP3}{Distributed Network Protocol 3}
\acro{DTMC}{Discrete Time Markov Chain}
\acro{DoS}{Denial of Service}
\acro{ECN}{Explicit Congestion Notification}
\acro{EWMA}{Exponentially Weighted Moving Average}
\acro{FIFO}{First In First Out}
\acro{FREEDM}{Future Renewable Electric Energy Delivery and Management}
\acro{HMM}{Hidden Markov Model}
\acro{IGT}{in group time}
\acro{IoT}{Internet of Things}
\acro{M2M}{Machine to Machine}
\acro{MSDND}{Multiple Security Domain Model Non-deducibility}
\acro{NS3}[NS-3]{Network Simulator 3}
\acro{RED}{Random Early Detection}
\acro{SCADA}{Supervisory Control And Data Acquisition}
\acro{SRC}{Sequenced Reliable Connection}
\acro{SUC}{Sequenced Unreliable Connection}
\acro{VANET}{Vehicular Ad Hoc Networks}
\acro{pdf}{probability distribution function}
\acro{wff}{Well-formed formula}
\end{acronym}%
\setlength{\parskip}{0pt}\doublespacing% ... added by koh ...
%
% Define the terms used in this document
\nomname{NOMENCLATURE}
\printnomenclature%
\end{ThesisFrontMatter}

\setlength{\textfloatsep}{72pt}
\setlength{\intextsep}{60pt}


% Thesis body.
\begin{ThesisBody}
\setlist{nosep}
\section{Introduction}
% Computer Society journal papers do something a tad strange with the very
% first section heading (almost always called "Introduction"). They place it
% ABOVE the main text! IEEEtran.cls currently does not do this for you.
% However, You can achieve this effect by making LaTeX jump through some
% hoops via something like:
%
%\ifCLASSOPTIONcompsoc
%  \noindent\raisebox{2\baselineskip}[0pt][0pt]%
%  {\parbox{\columnwidth}{\section{Introduction}\label{sec:introduction}%
%  \global\everypar=\everypar}}%
%  \vspace{-1\baselineskip}\vspace{-\parskip}\par
%\else
%  \section{Introduction}\label{sec:introduction}\par
%\fi
%
% Admittedly, this is a hack and may well be fragile, but seems to do the
% trick for me. Note the need to keep any \label that may be used right
% after \section in the above as the hack puts \section within a raised box.



% The very first letter is a 2 line initial drop letter followed
% by the rest of the first word in caps (small caps for compsoc).
% 
% form to use if the first word consists of a single letter:
% \IEEEPARstart{A}{demo} file is ....
% 
% form to use if you need the single drop letter followed by
% normal text (unknown if ever used by IEEE):
% \IEEEPARstart{A}{}demo file is ....
% 
% Some journals put the first two words in caps:
% \IEEEPARstart{T}{his demo} file is ....
% 
% Here we have the typical use of a "T" for an initial drop letter
% and "HIS" in caps to complete the first word.
\IEEEPARstart{F}{REEDM} (Future Renewable Electric Energy Delivery and Management) is a smart grid project focused on the future of the electrical grid.
Major proposed features of the FREEDM network include the solid state transformer, distributed local energy storage, and distributed local energy generation \cite{FREEDMMIGRATION}.
This vein of research emphasizes decentralizing the power grid, making it more reliable by distributing energy production resources.
Part of this design requires the system to operate in islanded mode, where portions of the distribution network are partitioned from each other.
The effects of these partitions are still not well understood.
This is particularly true in a distributed cyber-physical system, in which partitions may occur in both the cyber and physical domains.
Related work\cite{HARINI}\cite{TSG} has indicated that cyber faults can cause a physical system to apply unstable settings.

This work presents a distributed leader election algorithm and Markov model of that algorithm.
The presented algorithm maintains the Markov property for the observations of the leader despite omission failures
This approach to considering how a distributed system interacts during a fault condition allows for the creation of new techniques for managing a fault scenario in cyber-physical systems. 
This discussion presents an approach that utilizes Markov chain to model a system's grouping behavior.
These chains produce expectations of how long a system can be expected to stay in a particular state as well as how much time it will be able to spend coordinating and doing useful work over a period of time. 
Using these measures, the behavior of the control system for the physical devices can be adjusted to prevent faults.

PUT IN A LIST OF SECTIONS

%\IEEEpubidadjcol
% needed in second column of first page if using \IEEEpubid
\section{FREEDM DGI}

This study models the group management module of the FREEDM DGI.
The DGI is a smart grid operating system that organizes and coordinates power electronics.
It also negotiates contracts to deliver power to devices and regions that cannot effectively facilitate their own needs.
DGI leverages common distributed algorithms to control the power grid, making it an attractive target for modeling a distributed system.

The DGI software consists of a central component, known as the broker.
This broker is responsible for presenting a communication interface.
It also furnishes any common functionality the system's algorithms may need.
These algorithms, grouped into modules, work in concert to move power from areas of excess supply to excess demand.

DGI utilizes several modules to manage a distributed smart-grid system.
Group management, the focus of this work, implements a leader election algorithm to discover which processes are reachable within the cyber domain.
Other modules provide additional functionality, such as collecting global snapshots, negotiating the migrations, and giving commands to physical components.

DGI is a real-time system; certain actions (and reactions) involving power system components need to be completed within a pre-specified time-frame to keep the system stable.
It uses a round robin scheduler in which each module is given a predetermined window of execution which it may use to perform its duties.
When a module's time period expires, the next module in the line is allowed to execute. 
The start of each round is synchronized between systems.
Each DGI process will execute the same module at the same time.

\section{Group Management}
The DGI uses the leader election algorithm, ``Invitation Election Algorithm,'' written by Garcia-Molina \cite{INVITATIONELECTION}.
Originally published in 1982, this algorithm provides a robust election  procedure that allows for transient partitions.
Transient partitions are formed when a faulty link between two or more clusters of DGIs causes the groups to divide temporarily.
These transient partitions merge when the link becomes more reliable.
The election algorithm allows for failures that disconnect two distinct sub-networks.
These sub networks are fully connected, but connectivity between the two sub-networks is limited by an unreliable link.

Many election algorithms have been created. 
Each algorithm is designed to be well-suited to the circumstances it will deployed in.
Specialized algorithms exist for wireless sensor networks \cite{LE-WSN-1}\cite{LE-WSN-2}, and other special circumstances \cite{LE-SPECIALCIRCUMSTANCES-1}\cite{LE-SPECIALCIRCUMSTANCES-2}.
Work on leader elections has been incorporated into a variety of distributed frameworks: Isis \cite{ISISTOOLKIT}, Horus \cite{HORUSTOOLKIT}, Totem \cite{TOTEMTOOLKIT}, Transis \cite{TRANSISTOOLKIT}, and Spread \cite{SPREADTOOLKIT} all have methods for creating groups.
Despite this wide array of work, the fundamentals of leader election are similar across all implementations.
Processes arrive at a consensus of a single peer that coordinates the group.
Processes that fail are detected and removed from the group. 

The elected leader is responsible for making work assignments, and identifying and merging with other coordinators when they are found, as well as maintaining an up-to-date list of peers for the members of his group. 
Group members monitor the group leader by periodically checking if the group leader is still alive by sending a message. 
If the leader fails to respond, the querying nodes will enter a recovery state and operate alone until
they can identify another coordinator.
Therefore, a leader and each of the members maintain a set of processes which are currently reachable, a subset of all known processes in the system.

Leader election can also be classified as a failure detector \cite{LEADERELECTIONEVAL}.
Failure detectors are algorithms which detect the failure of processes within a system; they maintain a list of processes that they suspect have crashed.
This informal description gives the failure detector strong ties to the leader election process. 
The group management module maintains a list of suspected processes which can be determined from the set of all processes and the current membership.

The leader and the members have separate roles to play in the failure detection process.
Leaders use a periodic search to locate other leaders in order to merge groups.
This serves as a ping / response query for detecting failures within the system.
The member sends a query to its leader.
The member will only suspect the leader, and not the other processes in their group.

\section {Experimental Setup}

\subsection{Broker Architecture}
The DGI software used in this designed around a broker architectural specification.
Each core functionality of the system was implemented within a module that was provided access to core interfaces.
These interfaces provided functionality such as scheduling requests, message passing, and a framework to manipulate physical devices.
The Broker provided a common message passing interface that all modules could access.
This interface was then used to pass information between modules. 
For this purpose of this work, messages are sent as single UDP datagrams.
If a datagram is lost, it is not resent.
DGI expect an increasing sequence number on datagrams, which ensures message ordering.

\subsection{Algorithmic Changes}

The original Garcia-Molina algorithm has been modified so the observations of the coordinator process have the Markov property.
The full algorithm is presented below.
The rest of this section describes the changes that were made to the algorithm and shows how the combination of those changes allow the observations of the Coordinator to follow the Markov process.
The execution model of this algorithm assumes a real-time system using a round-robin scheduler.
All processes have their clock synchronized to a reasonable tolerance.
At a predetermined time and following predetermined intervals the algorithm executes at each process.
Using the synchronized clocks, all processes execute either $Check()$ or $Timeout()$ at the same time.
Processes can only form groups if their clocks are sufficiently in sync with another process' clock.

\begin{algorithmic}

\State $AllNodes \gets \{ 1, 2, ..., N \}$
\State $Coordinators \gets \emptyset$
\State $UpNodes \gets { Me }$
\State $State \gets Normal$
\State $Coordinator \gets Me$
\State $Responses \gets \emptyset$
\State $Counter \gets$ A random initial identifier
\State $GroupID \gets (Me,Counter)$

\State

\Function{Check}{}
    \State This function is called at the start of a round by the leader
    \If {$State = Normal$ and $Coordinator \gets Me$}
        \State $Responses \gets \emptyset$
        \State $TempSet \gets \emptyset$
        \For {$j = (AllNodes - \{Me\})$}
            \State $AreYouCoordinator(j)$
            \State $TempSet \gets TempSet \cup j$
        \EndFor
        \State Nodes which respond "Yes" to $AreYouCoordinator$ are put into the $Responses$ set.
        \State When an $AreYouThere$ response is "No" and this process is a coordinator, the querying process is put in the $Responses$ set.
        \State Wait for $Timeout(CheckTimeout)$, Nodes that do not respond are removed from UpNodes.
        \State $UpNodes \gets (TempSet-Responses) \cup {Me}$
        \If {$Responses = \emptyset$}
            \Return
        \EndIf
        \State $p \gets \max(Responses)$
        \If $Me > p$
            \State Wait time proportional to me-p
        \EndIf
        \Call{Merge}{Responses}
    \EndIf
    \State The next call to this is after Timeout(CheckTimeout)
\EndFunction

\State

\Function{Timeout}{}
    \State This function is called at the start of a round by the group members
    \If {$Coordinator = Me$}
        \Return
    \Else
        \Call{AreYouThere}{Coordinator,GroupID,Me}
        \If{Response is No}
            \Call{Recovery}{}
        \EndIf
    \EndIf
    \State The next call to this is after Timeout(TimeoutTimeout)
\EndFunction

\State

\Function{Merge}{Coordinators}
    \State This function invites all coordinators in Coordinators to join a group led by Me
    \State $State \gets Election$
    \State Stop work
    \State $Counter \gets Counter+1$
    \State $GroupID \gets (Me,Counter)$
    \State $Coordinator \gets Me$
    \State $TempSet \gets UpNodes - {Me}$
    \State $UpNodes \gets \emptyset$
    \For {$j \in Coordinators$}
        \Call{Invite}{j,Coordinator,GroupID}
    \EndFor
    \For {$j \in TempSet$}
        \Call{Invite}{j,Coordinator,GroupID}
    \EndFor
    \State Wait for $Timeout(InviteTimeout)$, Nodes that accept the invite are added to UpNodes
    \State $State \gets Reorganization$
    \For {$j \in UpNodes$}
        \Call{Ready}{j,Coordinator,GroupID,UpNodes}
    \EndFor
    \State $Acknowledge \gets UpNodes$
    \State Wait for $Timeout(ReadyTimeout)$, Nodes that do not acknowledge are removed from UpNodes
    \State $UpNodes \gets UpNodes - Acknowledge$
    \State $State \gets Normal$
\EndFunction

\State

\Function{ReceiveReady}{Sender,Leader, Identifier, Peers}
    \If {$State = Reorganization$ and $GroupID = Identifier$}
        \State $UpNodes \gets Peers$
        \State $State \gets Normal$
        \State Respond Ready Acknowledge 
    \EndIf
\EndFunction

\State

\Function{ReceiveAreYouCoordinator}{Sender}
    \If {$State = Normal$ and $Coordinator = Me$}
        \State Respond Yes
    \Else
        \State Respond No
    \EndIf
\EndFunction

\State

\Function{ReceiveAreYouThere}{Sender, Identifier}
    \If {$GroupID = Identifier$ and $Coordinator = Me$ and $Sender \in UpNodes$}
        \State Respond Yes
    \Else
        \State Respond No
        \State Add sender to $Responses$ set for $Check()$ if this process is a coordinator.
    \EndIf
\EndFunction

\State

\Function{ReceiveInvitation}{Sender,Leader,Identifier}
    \If {$State \neq Normal$}
        \Return
    \EndIf
    \If {$Sender \neq 0$}
        \Return
    \EndIf
    \State Stop Work
    \State $Temp \gets Coordinator$
    \State $TempSet \gets UpNodes$
    \State $State \gets Election$
    \State $Coordinator \gets Leader$
    \State $GroupID \gets Identifier$
    \If {$Temp = Me$}
        \State Forward invite to old group members
        \For $j \in TempSet$
            \State $Invite(j,Coordinator,GroupID)$
        \EndFor
    \EndIf
    \State $Accept(Coordinator,GroupID)$
    \State $State \gets Reorganization$
    \If {$Timeout(ReadyTimeout)$ expires before $Ready$ is received}
        \State $Recovery()$
    \EndIf
\EndFunction

\State

\Function{ReceiveAccept}{Sender,Leader,Identifier}
    \If {$State \gets Election$ and $GroupID = Identifier$ and $Coordinator = Leader$}
        \State $UpNodes \gets UpNodes \cup {Sender}$
    \EndIf
\EndFunction

\Function{ReceiveReadyAcknowledge}{Sender}
    \State $Sender$ is removed from $Acknowledge$ in $Merge()$
\EndFunction

\State

\Function{Recovery}{}
    \State $State \gets Election$
    \State Stop Work
    \State $Counter \gets Counter + 1$
    \State $GroupID \gets (Me,Counter)$
    \State $Coordinator \gets Me$
    \State $UpNodes \gets {Me}$
    \State $State \gets Reorganization$
    \State $State \gets Normal$
\EndFunction

\end{algorithmic}

In a distributed system information cannot be instantaneously spread throughout the system.
A process can only make local observations.
In this work, we attempt to model what a process will observe as a result of omission failure.
Therefore, it is important that observations that a process makes hold to the Markov property.
In the original algorithm, there are several portions that, when projected to the leader's observation, do not meet the Markov property.
The following sections state the portions of the algorithm where the observation of the leader process does not yield the probability of next transition.

Leader selection is performed a priori-- only process 0 may become a group coordinator.
Only process 0 may become the leader of a multiprocess group.
This simplification was applied because the configuration of the system with a larger number of processes depended on the configuration of the other processes.
Without this simplification, the state of the rest of the system would not have the memoryless property.
The state of the processes that are not in the observers group would change each round.
As a consequence the state of the rest of the system and the likelihood of forming a specific group size would change each step if other processes could become leader.

DIAGRAM HERE

The changes added a third message to completing an election -- a ready acknowledge message.
This message is sent by a member after receiving the ready message from the coordinator.
This allows the coordinator to be certain of the member's status before the next round.
Without the ready acknowledgment, the member may not receive the ready message and the coordinator will observe the member is a part of the group.
As a consequence of that uncertainty, the probability that a member remains in a group in the first round after an election has a different probability than each subsequent round.
By adding the extra message, the observation of the coordinator of the state, must be the state of the member of the group.
The sequence presented in figure XXX is not possible and as a consequence, the probability a member remains in a group in the first round after an election is a fixed value.

DIAGRAM HERE

Members cannot leave a group without the leader's permission.
Members do not suspect the coordinator has failed, only the coordinator may suspect the members.
For the purpose of starting an election, an Are You There message and it's negative response are considered equivalent to a Are You Coordinator message and a positive response.
On receipt of the negative response, the member will immediately recover and become a leader.
This assumption relies on Are You Coordinator and Are You There messages being sent at roughly the same time.

DIAGRAM HERE

This change leads to a live-lock situation in a crash failure, where the group's leader crashes and does not return and as a consequence the remaining members are trapped in a group without a leader.
For the purpose of this work, we have disregarded these live lock scenarios.
However, the live-lock could be avoided if the member can detect that it has not received an "Are You Coordinator" message in a round.
When the process fails to receive the message, the coordinator must have also removed the member from their group since they could not have received a "Are You Coordinator" response message.
Integrating this component is an area of future work.

\section{Formal Modeling}

\subsection{Assumptions}

All participating peers were assumed to be on the same schedule; all peers began executing the model simultaneously.
Synchronization was accomplished using Choi's work \cite{DCS}.
It was also assumed the clocks are synchronized. 
If the network has faulted, process clocks would not drift noticeably from their last synchronization.
As an alternative, a production system would likely use GPS time synchronization to obtain certain power system readings \cite{PHASORREADINGS}.

\subsection{Constructing The Markov Chain}


\section{Conclusion}




\end{ThesisBody}



%\begin{ThesisAppendix}{three}
%***%\chapter{Characteristic Length Scales}
\label{sec:lengths}
%This table lists lengths used in the dissertation.

\captionsetup[table]{list=no}
%\begin{center}
\begin{table} % for label and caption purposes
%\begin{tabular}{|l|l|l|} % table runs off page
%\begin{tabular}{|l|l|p{7cm}|} % center column too large
\caption{Length scales used in this dissertation}
\begin{tabular}{|l|p{4cm}|p{8.5cm}|}
%\begin{tabular*}{0.75\textwidth}{|l|l|l|}{@{\extracolsep{\fill}} | c | c | c | }
\hline Symbol & Name & Description \\ \hline
$\lambda$ & wavelength & Wavelength of incident light \\ 
$L$ &  system length & Length of waveguide along direction of propagation ($z$-axis) \\ 
$W$ & system width & Dimension of waveguide perpendicular to direction of propagation ($y$-axis) \\
$L_{\phi}$ & phase coherence length & Length over which phase remains coherent. Equivalent to $L_{inelastic}$\cite{1988_Stone,1986_Imry}. Applicable only to electron transport. \\
$L_D$ & path length & How far a particle (i.e. ray optics) travels in the media in ballistic and diffusive regimes \\
$\ell_{scat}$ & scattering length & Average distance between scattering events. Often referred to as $\ell_{mfp}$ (mean free path) or the inelastic length\cite{1984_John_prl}, or extinction length\cite{1999_van_Tiggelen}. \\ 
$\ell_{tmfp}$ & transport mean free path & Average distance over which phase and direction are randomized.  $\ell_{tmfp} = \frac{\ell_{scat}}{1-\langle cos\ \theta \rangle}$. Sometimes referred to as elastic mfp\cite{1991_John}. % page 34, col 2
Measured with respect to $L$. \\ 
$\xi$ & localization length & Probability of diffusive path forming loop is 1. $\xi~=~N~\ell_{tmfp}$. Measured with respect to $L$. \\ 
$\ell_{a,g}$ & ballistic absorption/gain length & Average distance over which intensity decreases by two/increases by two. \\
$\xi_{a,g}$ & absorption/gain length & How far, on average, a particle travels in the diffusive regime before being absorbed (or doubled), measured with respect to path length $L_D$ in the diffusive regime \\ 
$z_p$ & penetration depth & Applies to diffusive regime only. $z_p \approx \ell_{tmfp}$ \\ 
\hline
\end{tabular}

\ \\ 

\label{tabl:lengths}
\end{table}
%\end{center}
All length scales (except $\lambda$) are normalized by wavelength. % lengths can be converted to equivalent times by $\ell = c t$
%\captionsetup[table]{list=yes} % referenced in introduction
%\chapter{Transmission Derivation}

This is an expanded version of the analysis performed for Fig.~\ref{fig:energydistrib}

\begin{equation}
T = T(\omega, x_o) = \frac{(\exp(-\frac{L}{\xi}))^2}{ 
(\omega-\omega _o)^2 + (\exp(-\frac{L}{\xi})\frac{1}{2}\exp(\frac{\mid L-2 x_o \mid}{\xi}))^2 }
\label{fig:appendixtransmission}
\end{equation}
Where we have appoximated cosh() as $\frac{1}{2}\exp()$. 

From the behavior of transmission in media with defects and centers of localizaiton, we see two cases when on a resonant frequency:  either the defect is in the first half of the sample.%, (\ref{fig:cononicaldefectpositions}, plot 1).

\begin{equation}
{\cal E}(x) = \left\{ 
\begin{array}{l l}
  A \exp\left(\frac{2 (x-x_o)}{\xi}\right) & \quad 0 < x < x_o \\
  B \exp\left(-\frac{2 (x-x_o)}{\xi}\right) & \quad x_o < x < L\\
\end{array} \right.
\label{fig:left}
\end{equation}

or it is in the second half of the sample.%, (\ref{fig:cononicaldefectpositions}, plot 2).
\begin{equation}
{\cal E}(x) = \left\{ 
\begin{array}{l l}
 A \exp\left(-\frac{2 x}{\xi}\right) & \quad 0 < x < L-2 x_o \\
 B \exp\left(\frac{2 (x-x_o)}{\xi}\right) & \quad L-2 x_o < x < x_o \\
 C \exp\left(-\frac{2 (x-x_o)}{\xi}\right) & \quad x_o < x < L \\
\end{array} \right.
\label{fig:right}
\end{equation}

With either case, we see three distinct regions when the frequency is not on resonance but prior to the pure exponential decay regimes. %(\ref{fig:cononicaldefectpositions}, plot 3):
\begin{equation}
{\cal E}(x) = \left\{ 
\begin{array}{l l}
y_1 = A \exp\left(-\frac{2 x}{\xi}\right)   & \quad 0 < x < x_1  \\
y_2 = B \exp\left(\frac{2 (x-x_o)}{\xi}\right) & \quad x_1 < x < x_o  \\
y_3 = C \exp\left(-\frac{2 (x-x_o)}{\xi}\right) & \quad  x_o < x < L \\
\end{array} \right.
\label{fig:startingequations}
\end{equation}
Where $x_1$ is the turning point.

\begin{comment}
%% pictures were removed, causes a problem during compile (?)
\begin{figure}
\vskip -0.5cm
\centerline{
\scalebox{0.5}{\includegraphics{pictures/transmission_derivation_14_LR}}
\scalebox{0.5}{\includegraphics{pictures/transmission_derivation_34_LR}}
\scalebox{0.5}{\includegraphics{pictures/transmission_derivation_off_res_LR}}
}
\vskip -0.5cm
\caption{Log of energy versus position sketches for a center of localization in the first half (left), Eq.~\ref{fig:left}; and second half (center), Eq.~\ref{fig:right}. The turning point in the center plot is $L-2 x_o$. The right sketch is off-resonance but prior to pure exponential decay, which applies to any center of localization position, as described by Eq.~\ref{fig:startingequations}. The turning point in the right plot from the initial exponential decay to growth varies depending how far off resonance one is. We call this $x_1$}
\label{fig:cononicaldefectpositions}
\end{figure}
\end{comment}

From the off-resonance Eq.~\ref{fig:startingequations}, we apply boundry conditions to determine the coefficients in passive systems. We take gain into account and make corrections later.

At $x=0$ then $y_1=4$, giving $A=4$. Thus
\begin{equation}
\boxed{y_1 = 4 \exp\left(-\frac{2 x}{\xi}\right)}   \quad \quad \quad 0 < x < L-2 x_o 
\end{equation}

At $x=L$, $y_3=T$. Solve for C,
\begin{equation}
C=T \exp\left(\frac{2(L-x_o)}{\xi}\right)
\end{equation}
plug back in,
\begin{equation}
y_3 = T \exp\left(\frac{2(L-x_o)}{\xi}) \exp(-\frac{2 (x-x_o)}{\xi}\right)  \quad \quad \quad  x_o < x < L
\end{equation}
simplify to get
\begin{equation}
\boxed{y_3 = T \exp\left(-\frac{2(x-L)}{\xi}\right)}  \quad \quad \quad  x_o < x < L
\end{equation}
at $x_o$, $y_2 = y_3$
\begin{equation}
B \exp\left(\frac{2 (x_o-x_o)}{\xi}\right) = B = T \exp\left(-\frac{2(x_o-L)}{\xi}\right)
\end{equation}
plug in to $y_2$
\begin{equation}
y_2 = T \exp\left(-\frac{2(x_o-L)}{\xi}\right) \exp\left(\frac{2 (x-x_o)}{\xi}\right) \quad \quad \quad L-2 x_o < x < x_o 
\end{equation}
simplify to get
\begin{equation}
\boxed{y_2 = T \exp\left(\frac{2(x+L-2 x_o)}{\xi}\right)} \quad \quad \quad L-2 x_o < x < x_o 
\end{equation}

The turning point $x_1$ varies as a function of frequency. Boundry conditions on $x_1$ are that it should remain less than $x_o$ and that it be non-negative. To solve for $x_1$, see where $y_1=y_2$
\begin{equation}
4 \exp\left(-\frac{2 x_1}{\xi}\right) = T \exp\left(\frac{2(x_1+L-2 x_o)}{\xi}\right)
\end{equation}

\begin{equation}
\boxed{x_1(\omega) = -\frac{\xi}{4} \log(\frac{1}{4} T) - \frac{1}{2}L + x_o}
\end{equation}

Knowing $y_1,y_2$, and $y_3$ we can integrate to find the energy ${\cal E}$:
\begin{equation}
\begin{gathered}
{\cal E}(x,\omega) = \int _0 ^{x_1} 4 \exp\left(-\frac{2 x}{\xi}\right) dx +
    \int _{x_1} ^{x_o} T \exp\left(\frac{2(x+L-2 x_o)}{\xi}\right) dx + \\
    \int _{x_o} ^{L} T \exp\left(-\frac{2(x-L)}{\xi}\right)
\label{fig:Eintegral}
\end{gathered}
\end{equation}

Solve, reduce to get the energy as a function of x, frequency, and transmission:
\begin{equation}
\begin{gathered}
{\cal E}(x,\omega) = \frac{\xi}{2} \left( T \left( \exp\left(\frac{2(L-x_o)}{\xi}\right) - \exp\left(\frac{2(L-2 x_o+x_1)}{\xi}\right)+\right. \right.\\
\left.\left.\exp\left(\frac{2(L-x_o))}{\xi}\right) -1 \right) + 4 \left(1-\exp\left(-\frac{2 x_1}{\xi}\right)\right) \right)
\label{fig:appendixenergy}
\end{gathered}
\end{equation}

 % expansion of math in T/E paper
%***%%\chapter{Derivation of Transfer Matrices for Electric Field Propagation in Planar Quasi-1D Waveguide From Maxwell's Equations}
\chapter{Transfer Matrices for Electric Field Propagation}
\label{sec:appendix_derivation_transfer_matrices_quasi1d}

In the following derivation, the transfer matrix method\cite{1981_MacKinnon_scaling}\cite{1992_Pendry}\cite{2003_Kettemann} is developed from Maxwell's equations\cite{1999_Jackson}. Before starting, the assumptions necessary for the derivation are enumerated. 
\begin{itemize}
\item No leakage of electric field at edge ($y=0$, $y=W$) of waveguide (i.e. metallic boundaries). Gives boundary conditions of zero field at edges. Incident and output edges are open (no restrictions).
\item $\delta$-function scattering potentials, later reduced to a finite sum of Fourier components. Use of this scattering potential has generalized results
%   \item non-physical
%   \item infinite number of closed channels
\item No inelastic scattering: no energy loss due to scattering when passive, and phase remains coherent [scatterers only affect amplitude].
\item No noise (spontaneous emission). We are interested primarily in the AL/diffusion phenomenon. Also, experimentally noise can be suppressed. % CITE
%The transmission with gain will be slightly different compared to real output.
\item The gain mechanism is purely mathematical: no atomic level modeling is included. This is part of being mesoscopic regime: independent of atomic-based scattering mechanisms.
\item No input beam properties are assumed (can be plane wave, but that is not necessary).%, other than the capability to be selectively incident on a specific channel.
%\item By choosing planar quasi-1D geometry (and thus scalar waves), we implicitly assume that polarization will not significantly alter transport phenomena. Although planar geometry may be experimentally realizable, it is not as popular as 3D quasi-1D
\end{itemize}

%\section{From Maxwell equations to wave equation}
%Starting with Maxwell's equations, the wave equation for electromagnetic waves is derived to demonstrate that only a single polarization is being studied. Ampere's Law (with current $\vec{J}=0$ and $D=\epsilon_0 E$)\cite{1999_Jackson}, %Eq. I.1b, page 2
%\begin{equation}
%\vec{\nabla} \times \vec{{\cal H}} = \epsilon_0 \frac{\partial\vec{E}}{\partial t}
%\label{eq:amperes_law}
%\end{equation}
%Faraday's Law \cite{1999_Jackson}, %Eq. I.1a, page 2]
%\begin{equation}
%\vec{\nabla} \times \vec{E} = -\mu_0 \frac{\partial\vec{{\cal H}}}{\partial t}
%\label{eq:faradays_law}
%\end{equation}
%Taking the temporal derivative of Eq.~\ref{eq:amperes_law}
%\begin{equation}
%\vec{\nabla} \times \frac{\partial\vec{{\cal H}}}{\partial t} = \epsilon_0 \frac{\partial^2\vec{E}}{\partial t^2}
%\label{eq:temporal_derivative_of_amperes_law}
%\end{equation}
%Taking the curl of Eq.~\ref{eq:faradays_law}
%\begin{equation}
% \vec{\nabla} \times \vec{\nabla} \times \vec{E} = -\mu_0 \left(\vec{\nabla} \times \frac{\partial\vec{{\cal H}}}{\partial t} \right)
%\label{eq:curl_of_faradays_law}
%\end{equation}
%Substitute Eq.~\ref{eq:temporal_derivative_of_amperes_law} into Eq.~\ref{eq:curl_of_faradays_law}
%\begin{equation}
%  \vec{\nabla} \times \vec{\nabla} \times \vec{E} = -\mu_0 \epsilon_0 \frac{\partial^2\vec{E}}{\partial t^2}
%\end{equation}
%Vector identity: $\vec{\nabla} \times \vec{\nabla} \times \vec{E} = \vec{\nabla} (\vec{\nabla}\cdot \vec{E})-\nabla^2 \vec{E}$. Since we assume no source, we have the wave equation

The wave equation is derived from Maxwell's equations (not show). In the following, only $s$-polarized waves (for transverse-magnetic (TM) waves) are assumed incident: electric field oscillates perpendicular to the plane of the 2D waveguide.
%``For EM waves propagating in the $z,y$ plane, the $s$ ($E$ field parallel to the $x$ axis) and $p$ ($E$ field perpendicular to the $x$ axis) polarized waves can be described by two decoupled wave equations.'' 
\cite{1996_soukoulis_dis2d}
\begin{equation}
\nabla^2 E = \frac{1}{c^2} \frac{\partial^2\vec{E}}{\partial t^2}
\label{eq:general_wave_equ}
\end{equation}
where $\mu_0 \epsilon_0 = \frac{1}{c^2}$. 
%[See 1996 Soukoulis et al PRB v53 n13 pg 8340, section II] 
% Eq.~\ref{eq:general_wave_equ} ``is identical with the scalar wave equation.'' The $p$-polarized wave equation involves ${\cal H}$.

\section{Time Independent Wave Equation}
Assuming electric field variables are separable,
\begin{equation}
E(\vec{r},t) = E(\vec{r}) e^{i\omega t}
\label{eq:separation_of_variables_E_field}
\end{equation}
the field is simplified by also assuming monochromatic and continuous wave (CW). Substituting  Eq.~\ref{eq:separation_of_variables_E_field} into the right side of Eq.~\ref{eq:general_wave_equ}, time dependence can be canceled. 
\begin{equation}
\nabla^2 E(\vec{r}) = - \frac{\omega^2}{c^2} E(\vec{r})
\label{eq:wave_equation_electric_field}
\end{equation}
where $\frac{\omega}{c}=k$. Although the following results will appear to be ``time independent,'' the time dependence can be reintroduced by multiplying both sides by $e^{i\omega t}$. Effectively the same as assuming $t=0$. 


\section{Separation of Variables}
Convert from general $\vec{r}$ to two-dimensional Cartesian coordinates (since the transfer matrices for a planar quasi-1D waveguide are desired): $\vec{r} = z \hat{i}+y\hat{j}$. Let $W \equiv$ width and $L \equiv$ length of waveguide.
%\begin{equation}
%\vec{r} = z \hat{i}+y\hat{j}
%\end{equation}
%and the Laplacian is
%\begin{equation}
%\nabla^2 = \frac{\partial^2}{\partial z^2} + \frac{\partial^2}{\partial y^2}
%\label{eq:laplacian_cartesian}
%\end{equation}

%unjustified claim
The $z$ and $y$ components of the field are independent, the separation of variables applies spatially.
\begin{equation}
E(\vec{r}) = E(z,y) = \sum^\infty_{n=1} E_n(z) \chi_n(y)
\label{eq:spacialseparation}
\end{equation}
where the sum is over all channels. For $\delta$-function scatterers, there can be an infinite number of closed channels.

Now the wave equation (Eq.~\ref{eq:wave_equation_electric_field}) is 
\begin{equation}
\nabla^2 E(z,y) = - \frac{\omega^2}{c^2} E(z,y)
\label{eq:wave_equation_electric_field_cartesian}
\end{equation}
Apply Laplacian %(Eq.~\ref{eq:laplacian_cartesian}) 
and separation (Eq.~\ref{eq:spacialseparation})
\begin{equation}
\sum^\infty_{n=1} \left[ \frac{\partial^2E_n(z)}{\partial z^2} \chi_n(y) + E_n(z) \frac{\partial^2 \chi_n(y)}{\partial y^2} \right] = \\
- \frac{\omega^2}{c^2} \sum^\infty_{n=1} E_n(z) \chi_n(y)
\label{eq:summation_variable_separation_cartesian}
\end{equation}

\section{Perpendicular Component Solution}
The solution to the differential equation perpendicular to the direction of propagation is found from the auxiliary equation for each channel
\begin{equation}
\left(\frac{\partial^2}{\partial y^2} + k_{\bot n}^2 \right) \chi_n(y) = 0
\end{equation}
Boundary conditions for metallic waveguide: Electric field $E$ is zero at the boundaries, $\chi_n(0) = \chi_n(W) = 0$.
%\begin{equation}
%\chi_n(0) = \chi_n(W) = 0
%\end{equation}
The normalized solution is the familiar
\begin{equation}
\chi_n(y) = \sqrt{\frac{2}{W}} \sin (k_{\bot n} y)
\end{equation}
where 
%\begin{equation}
$k_{\bot n} \equiv \frac{n \pi}{W}$. 
%\label{eq:k_perpendicular}
%\end{equation}
As a check of normalization, for $m=n$
\begin{equation}
\int^W_0 \chi_n^2(y) dy = \frac{2}{W} \int^W_0 \sin^2 (k_{\bot n} y) = \frac{2}{W} \frac{1}{2} W = 1
\end{equation}
and if $m \neq n$, solutions are orthogonal
\begin{equation}
\int^W_0 \chi_n(y)\chi_m(y) dy = 0
\end{equation}
Thus, for general $n$ and $m$,
\begin{equation}
\int^W_0 \chi_n(y)\chi_m(y) dy = \delta_{n,m}
\label{eq:convert_to_kronecker}
\end{equation}

\section{Parallel Component Solution}
For the solution parallel to the direction of propagation of Eq.~\ref{eq:wave_equation_electric_field_cartesian}, the $z$-component starts with
\begin{equation}
\frac{\partial^2 E_n(z)}{\partial z^2} - k_{\bot n}^2 E_n(z) = - \frac{\omega^2}{c^2} E_n(z)
\end{equation}
Re-arrange and introduce a new variable
\begin{equation}
\frac{\partial^2 E_n(z)}{\partial z^2} + k_{\parallel n}^2 E_n(z) = 0
\label{eq:zcomponentdiffequ}
\end{equation}
where
\begin{equation}
k_{\parallel n}^2 \equiv \frac{\omega^2}{c^2} - k_{\bot n}^2
\label{eq:k_parallel}
\end{equation}
Note: $k_{\parallel n}^2$ can be positive (corresponding to open channels) or negative (closed channels). If negative, then $k_{\parallel n}$ is imaginary, denoted $k_{\parallel n} = i \kappa_{\parallel n}$ for $n > N_{open}$. Open channels propagate forward, with velocity decreasing as channel index increases. Closed channels decrease in amplitude exponentially.

%The summation in Eq.~\ref{eq:summation_variable_separation_cartesian} can be split into open and closed channels
%\begin{equation}
%\sum_{n=1}^\infty = \sum_{n=1}^{N_o} + \sum_{n=N_{o}+1}^\infty	
%\end{equation}

%\begin{tabular}{cc}
%$ n \leq N_o $ \quad \quad \quad & $ k_{\parallel n}^2 = \frac{\omega^2}{c^2} - \left(\frac{n \pi}{W}\right)^2 > 0 $ \\
%$ n > N_o $ \quad \quad \quad & $ k_{\parallel n}^2 < 0 $ \\
%\end{tabular}

Separate electric field components into left(-) and right (+) traveling plane waves (two solutions to the second order differential equation)
\begin{equation}
\begin{gathered}
\text{Open: \ }  E_n(z) = E_n^+ \exp(i k_{\parallel n} z) + E_n^- \exp(-i k_{\parallel n} z) \\
\text{Closed: \ }   E_n(z) = E_n^+ \exp(-\kappa_{\parallel n} z) + E_n^- \exp(\kappa_{\parallel n} z) 
\end{gathered}
\label{eq:Eleftandrightpropagating}
\end{equation}
where $i \kappa \equiv k$

\section{Waveguide With Scatterers}
Up to this point, an empty waveguide has been considered. For scattering, replace $\frac{\omega^2}{c^2}$ of the wave equation \ref{eq:wave_equation_electric_field_cartesian} with a spacial Sellmeier equation
\begin{equation}
\frac{\omega^2}{c^2} (1 + \alpha \delta(z-z_0,y-y_0))
\label{eq:scatterer}
\end{equation}
where $\delta(z-z_0,y-y_0) \equiv \delta(z-z_0) \delta(y-y_0)$ is the scattering potential and $\alpha$ is the scattering strength. $\alpha$ can be complex; then the real part is the strength and the imaginary component is gain or absorption.

%Note: this new term changes Eqs.~\ref{eq:amperes_law},\ref{eq:general_wave_equ}, and \ref{eq:wave_equation_electric_field} by introducing a non-unity refractive index.

To determine transport of light past a scattering potential, apply continuity of electric field $E$ and its derivative. The following carries out matching component-wise derivative.

Assuming the scattering potential is located at cross-section $z$ (inside the waveguide $0<z<L$), and the electric field just before or after the scatterer (at $z \pm \Delta$) is a sum of independent channel components.
\begin{equation}
E(z \pm \Delta, y) = \sum_{n=1}^\infty E_n(z \pm \Delta) \chi_n(y)
\end{equation}
Applying Eq.~\ref{eq:scatterer} to Eq.~\ref{eq:zcomponentdiffequ}, the wave equation becomes
\begin{equation}
\sum_{n=1}^\infty \left( E_n^{\prime\prime} \chi_n + k_{\parallel n}^2 E_n \chi_n + \alpha \frac{\omega^2}{c^2} \delta(z-z_0,y-y_0) E_n \chi_n \right) = 0
\label{eq:doubleprimeEz}
\end{equation}

Multiply Eq.~\ref{eq:doubleprimeEz} by $ \chi_m $ and $ \int_0^W dy $. By applying Eq.~\ref{eq:convert_to_kronecker} and letting $A_{m,n}(y_0)=\chi_m(y_0) \chi_n(y_0)$, 
\begin{equation}
\sum_{n=1}^\infty \left( E_n^{\prime\prime} \delta_{nm} + k_{\parallel n}^2 E_n \delta_{nm} + \alpha \frac{\omega^2}{c^2} E_n \delta(z_0) A_{nm}(y_0)  \right) = 0 
\end{equation}
Apply the summation over $n$, which eliminates the Kronecker deltas.
\begin{equation}
 E_m^{\prime\prime} + k_{\parallel m}^2 E_m + \alpha \frac{\omega^2}{c^2} E_n \delta(z-z_0) \sum_{n=1}^\infty A_{nm}(y_0) = 0
\end{equation}
Integrate over $z$ from $(z-\Delta)$ to $(z+\Delta)$ and let $\Delta\rightarrow0$.
\begin{equation}
\begin{gathered}
\int_{z_0-\Delta}^{z_0+\Delta} E_m^{\prime\prime}(z) dz + k_{\parallel m} ^2 \int_{z_0-\Delta}^{z_0+\Delta} 
E_m(z) dz +\\ \alpha \frac{\omega^2}{c^2} \sum_{n=1}^\infty A_{n,m}(y_0) \int_{z_0-\Delta}^{z_0+\Delta} \delta(z_0) E_n dz = 0
\end{gathered}
\end{equation}

To do the second term integration, assume that for small $\Delta$, $E(z) \approx E(z_0)$.  
\begin{equation}
E_m^{\prime}(z_0 + \Delta) - E_m^{\prime}(z_0 - \Delta) + k_{\parallel m}^2 E_m(z_0) 2 \Delta + \alpha \frac{\omega^2}{c^2} \sum_{n=1}^\infty A_{n,m}(y_0) E_n(z_0) = 0
\end{equation}

Since $\Delta \rightarrow 0$, then $2 \Delta$ is really small, so that term is dropped.

To conclude, for a given channel $m$, electric field and the field derivative on both sides of the scatterer must match
\begin{equation}
\begin{gathered}
%\boxed{
E_m(z_0+\Delta) = E_m(z_0-\Delta) \\
%}
%\\
%\boxed{
E_m^{\prime}(z_0+\Delta) = E_m^{\prime}(z_0-\Delta) - \alpha \frac{\omega^2}{c^2} \sum_{n=1}^\infty A_{n,m}(y_0) E_n(z_0)
%}
\end{gathered}
\end{equation}

Note that the $\delta$ function scatterer has been eliminated, and $A_{n,m}$ can form an array (the ``scattering matrix'').

\begin{equation}
\begin{gathered}
\left( \begin{array}{cc}
\hat{I} & 0 \\
-\alpha \frac{\omega^2}{c^2}A_{mn}(y_0) & \hat{I} \\
\end{array} \right)
\left( \begin{array}{c}
E_{1..N_{max}}(z_0-\Delta) \\
\frac{1}{\kappa_{\parallel 1..N_{max}}} E_{1..N_{max}}^{\prime}(z_0-\Delta) 
\end{array} \right)
=\\
\left( \begin{array}{c}
E_{1..N_{max}}(z_0+\Delta) \\
\frac{1}{\kappa_{\parallel 1..N_{max}}} E_{1..N_{max}}^{\prime}(z_0+\Delta) 
\end{array} \right)
\end{gathered}
\end{equation}
Due to the form of the matrix, the determinant is always unity (only the diagonal contributes non-zero terms) regardless of the elements in the lower left quadrant. Elements of the lower left quadrant are
\begin{equation}
 -\alpha \frac{\omega^2}{c^2} \frac{2}{W} \sin(k_{\perp m} y_0) \sin(k_{\perp n} y_0)
\end{equation}
Note that the scattering matrix is real unless $\alpha$ or $\omega$ are complex.

\section{Free Space Propagation of Open Channels}
For open channels ($n \leq N_o$), field $E_n$ and derivative of field $ \frac{1}{k_{\parallel n}}E_n^{\prime}$ are more convenient basis than ``left traveling'' $E_n^-(z)$ and ``right traveling'' $E_n^+(z)$. First, the connection between the two basis is found. Starting from Eq.~\ref{eq:Eleftandrightpropagating}, electric field $E(z)$ is the solution to a second order differential equation, so it has two solutions.
\begin{equation}
\begin{gathered}
E_n(z) = E_n^+ \exp(i k_{\parallel n} z) + E_n^- \exp(-i k_{\parallel n} z) \\
E_n^{\prime}(z) = i k_{\parallel n} E_n^+ \exp(i k_{\parallel n} z) - i k_{\parallel n}E_n^- \exp(-i k_{\parallel n} z)
\label{eq:Eleftandrightpropagating_again}
\end{gathered}
\end{equation}
Solving for left- and right-traveling wave components,
\begin{equation}
\begin{gathered}
E_n^+(z) = \frac{1}{2} \left( E_n(z)+\frac{1}{i} \frac{1}{k_{\parallel n}} E_n^{\prime}(z) \right) \exp(-i k_{\parallel n} z) \\ 
E_n^-(z) = \frac{1}{2} \left( E_n(z)-\frac{1}{i} \frac{1}{k_{\parallel n}} E_n^{\prime}(z) \right) \exp(i k_{\parallel n} z) 
\label{eq:Eleftright}
\end{gathered}
\end{equation}

To preemptively clear up notation confusion, in previous steps $\Delta$ was used to denote a small distance away from the scatterer. Here $\Delta z$ will be used to signify a not infinitesimal displacement in position along the $z$ axis. The field and derivative of field is translated over distance $\Delta z$ from the original position $z$. First, substitute the shift into Eq.~\ref{eq:Eleftandrightpropagating_again}
\begin{equation}
E_n(z+\Delta z) = E_n^+ \exp(i k_{\parallel n} (z+\Delta z)) + E_n^- \exp(-i k_{\parallel n} (z+\Delta z)) 
\end{equation}
Then substitute Eq.~\ref{eq:Eleftright}
\begin{equation}
\begin{gathered}
E_n(z+\Delta z) = \frac{1}{2} \left( E_n(z) + \frac{1}{i} \frac{1}{k_{\parallel n}} E_n^{\prime}(z) \right) \exp(i k_{\parallel n} z) + \\
\frac{1}{2} \left( E_n(z) - \frac{1}{i} \frac{1}{k_{\parallel n}} E_n^{\prime}(z) \right) \exp(-i k_{\parallel n} z) 
\end{gathered}
\end{equation}
Reducing leaves how to shift an electric field over distance $\Delta z$.
\begin{equation}
%\boxed{
E_n(z+\Delta z) =E_n(z) \cos(k_{\parallel n} \Delta z) + \frac{1}{k_{\parallel n}} E_n^{\prime} \sin(k_{\parallel n} \Delta z) 
%}
\label{eq:open_channel_field_transfer}
\end{equation}
Similarly,
\begin{equation}
\frac{1}{k_{\parallel n}} E_n^{\prime}(z+\Delta z) = i E_n^+ \exp(i k_{\parallel n} (z+\Delta z)) -i E_n^- \exp(-i k_{\parallel n} (z+\Delta z))
\end{equation}
Then substitute Eq.~\ref{eq:Eleftright}
\begin{equation}
\begin{gathered}
\frac{1}{k_{\parallel n}} E_n^{\prime}(z+\Delta z) =
\frac{i}{2} \left( E_n(z) + \frac{1}{i} \frac{1}{k_{\parallel n}} E_n^{\prime}(z) \right) \exp(i k_{\parallel n} z) -\\
\frac{i}{2} \left( E_n(z) - \frac{1}{i} \frac{1}{k_{\parallel n}} E_n^{\prime}(z) \right) \exp(-i k_{\parallel n} z) 
\end{gathered}
\end{equation}

\begin{equation}
%\boxed{
\frac{1}{k_{\parallel n}} E^{\prime}(z+\Delta z)=- E_n(z) \sin(k_{\parallel n} \Delta z) + \frac{1}{k_{\parallel n}} E_n^{\prime} \cos(k_{\parallel n} \Delta z) 
%}
\label{eq:open_channel_deriv_transfer}
\end{equation}

\section{Free-space Propagation of Closed Channels}
For closed channels ($n > N_o$), change of $i$ results in hyperbolic trig functions.
\begin{equation}
\begin{gathered}
E_n(z) = E_n^+ \exp(-\kappa_{\parallel n} z) + E_n^- \exp(\kappa_{\parallel n} z) \\
E_n^{\prime}(z) = -\kappa_{\parallel n} E_n^+ \exp(-\kappa_{\parallel n} z) + \kappa_{\parallel n}E_n^- \exp(\kappa_{\parallel n} z)
\end{gathered}
\end{equation}
Recalling that $k_{\parallel n} = i \kappa_{\parallel n}$, then
\begin{equation}
\begin{gathered}
E_n^+(z) = \frac{1}{2} \left( E_n(z)- \frac{1}{\kappa_{\parallel n}} E_n^{\prime}(z) \right) \exp(\kappa_{\parallel n} z) \\ 
E_n^-(z) = \frac{1}{2} \left( E_n(z)+ \frac{1}{\kappa_{\parallel n}} E_n^{\prime}(z) \right) \exp(-\kappa_{\parallel n} z) 
\end{gathered}
\end{equation}

Shifting the field by $\Delta z$
%\begin{equation}
%E_n(z+\Delta z) = E_n^+ \exp(-\kappa_{\parallel n}(z+\Delta z)) + E_n^- \exp(\kappa_{\parallel n}(z+\Delta z))
%\end{equation}
%\begin{equation}
%E_n(z+\Delta z) = \frac{1}{2} \left( E_n(z)- \frac{1}{\kappa_{\parallel n}} E_n^{\prime}(z) \right) \exp(-\kappa_{\parallel n} z) +  
%\frac{1}{2} \left( E_n(z)+ \frac{1}{\kappa_{\parallel n}} E_n^{\prime}(z) \right) \exp(\kappa_{\parallel n} z) 
%\end{equation}
\begin{equation}
%\boxed{
E_n(z+\Delta z) = E_n(z) \cosh(\kappa_{\parallel n}\Delta z) + \frac{1}{\kappa_{\parallel n}} E_n^{\prime}(z) \sinh(\kappa_{\parallel n}\Delta z)
%}
\end{equation}
and
%\begin{equation}
%\frac{1}{\kappa_{\parallel n}}E_n^{\prime}(z+\Delta z) = -E_n^+ \exp(-\kappa_{\parallel n}(z+\Delta z)) + E_n^- \exp(\kappa_{\parallel n}(z+\Delta z)) 
%\end{equation}
%\begin{equation}
%\begin{gathered}
%\frac{1}{\kappa_{\parallel n}}E_n^{\prime}(z+\Delta z) =-\frac{1}{2} \left( E_n(z)- \frac{1}{\kappa_{\parallel n}} E_n^{\prime}(z) \right) \exp(-\kappa_{\parallel n} z) + \\
%\frac{1}{2} \left( E_n(z)+ \frac{1}{\kappa_{\parallel n}} E_n^{\prime}(z) \right) \exp(\kappa_{\parallel n} z) 
%\end{gathered}
%\end{equation}
\begin{equation}
%\boxed{
\frac{1}{\kappa_{\parallel n}}E_n^{\prime}(z+\Delta z) =E_n(z) \sinh(\kappa_{\parallel n}\Delta z) + \frac{1}{\kappa_{\parallel n}} E_n^{\prime}(z) \cosh(\kappa_{\parallel n}\Delta z)
%}
\end{equation}

To summarize,
\begin{equation}
\begin{gathered}
E_n(z+\Delta z) = E_n(z) \cosh(\kappa_{\parallel n}\Delta z) + \frac{1}{\kappa_{\parallel n}} E_n^{\prime}(z) \sinh(\kappa_{\parallel n}\Delta z) \\
\frac{1}{\kappa_{\parallel n}}E_n^{\prime}(z+\Delta z) =E_n(z) \sinh(\kappa_{\parallel n}\Delta z) + \frac{1}{\kappa_{\parallel n}} E_n^{\prime}(z) \cosh(\kappa_{\parallel n}\Delta z)
\label{eq:closed_channel_transfer}
\end{gathered}
\end{equation}

From Eq.~\ref{eq:open_channel_field_transfer}, \ref{eq:open_channel_deriv_transfer}, and \ref{eq:closed_channel_transfer} the ``free space propagation matrix'' can be constructed. The array would be of rank $2 n_{max}$ ($n_{max}=N_o+N_c$). The determinant of this matrix is alway unity (regardless of argument) because terms can be factored into $\sin^2x +\cos^2=1$ for each channel. Thus, for both free and scattering matrices, the determinant is unity regardless of free space separation $\Delta z$ or real (passive) and complex (active media) dielectric values.

\begin{comment}
\begin{equation}
\left( \begin{array}{cccccccc}
\cos(k_{\parallel 1}\Delta z)   & 0 & 0 & 0 & \sin(k_{\parallel 1}\Delta z)   & 0 & 0 & 0 \\
0 & \cos(k_{\parallel N_0}\Delta z) & 0 & 0 & 0 & \sin(k_{\parallel N_0}\Delta z) & 0 & 0 \\

0 & 0 & \cosh(k_{\parallel N_0+1}\Delta z)   & 0 & 0 & 0 & \sinh(k_{\parallel n}\Delta z) & 0  \\
0 & 0 & 0 & \cosh(k_{\parallel n_{max}}\Delta z) & 0 & 0 & 0 & \sinh(k_{\parallel n}\Delta z) \\

-\sin(k_{\parallel n}\Delta z) & 0 & \cos(k_{\parallel n}\Delta z) & 0 \\
-\sin(k_{\parallel n}\Delta z) & 0 & \cos(k_{\parallel n}\Delta z) & 0 \\

0 & \sinh(k_{\parallel n}\Delta z) & 0 & \cosh(k_{\parallel n}\Delta z) \\
0 & \sinh(k_{\parallel n}\Delta z) & 0 & \cosh(k_{\parallel n}\Delta z) \\
\end{array} \right)
\end{equation}

\begin{equation}
\left( \begin{array}{cccc}
\cos(k_{\parallel 1..N_0}\Delta z)  & 0 & \sin(k_{\parallel 1}\Delta z) & 0 \\
0 & \cosh(k_{\parallel N_0+1..n_{max}}\Delta z) & 0 & \sinh(k_{\parallel N_0+1..n_{max}}\Delta z)  \\
-\sin(k_{\parallel 1..N_0}\Delta z) & 0 & \cos(k_{\parallel 1..N_0}\Delta z) & 0 \\
0 & \sinh(k_{\parallel N_0+1..n_{max}}\Delta z) & 0 & \cosh(k_{\parallel N_0+1..n_{max}}\Delta z) \\
\end{array} \right)
\left( \begin{array}{c}
E_1 \\
\vdots \\
E_N \\	
E_{N+1} \\
\vdots \\
E_{N_{max}} \\
\frac{1}{k_{\parallel 1}} E_1^{\prime} \\
\vdots \\
\frac{1}{k_{\parallel N}} E_N^{\prime} \\
\frac{1}{\kappa_{\parallel N+1}} E_{N+1}^{\prime} \\
\vdots \\
\frac{1}{\kappa_{\parallel N_{max}}} E_{N_{max}}^{\prime} 
\end{array} \right)
\end{equation}
Note that the derivative portion of the free space propagation is where ``energy loss'' occurs (in passive $N_{open}$ systems). 


\section{Boundary Condition}

\begin{equation}
v(z) = 
\left( \begin{array}{c}
E_1 \\
\vdots \\
E_N \\	
E_{N+1} \\
\vdots \\
E_{N_{max}} \\
\frac{1}{k_{\parallel 1}} E_1^{\prime} \\
\vdots \\
\frac{1}{k_{\parallel N}} E_N^{\prime} \\
\frac{1}{\kappa_{\parallel N+1}} E_{N+1}^{\prime} \\
\vdots \\
\frac{1}{\kappa_{\parallel N_{max}}} E_{N_{max}}^{\prime} 
\end{array} \right)
= \left( \begin{array}{c}
\vec{E}_{open}(z) \\
\vec{E}_{closed}(z) \\
\vec{D}_{open}(z) \\
\vec{D}_{open}(z) \\
\end{array} \right)
\end{equation}

\begin{equation}
\hat{M}(0,L) \vec{v}(0) = \vec{v}(L)
\end{equation}

\begin{equation}
 E(y,z,t) = \left\{
\begin{array}{l l}
E_{in} e^{i(\vec{r}\cdot \vec{k} - \omega t)} + E_{r} e^{-i(\vec{r}\cdot \vec{k} + \omega t)} & \quad \mbox{ if $x<0$} \\
E_t e^{i(\vec{r}\cdot \vec{k} - \omega t)}  & \quad \mbox{ if $x>L$} \\ \end{array} \right.
\end{equation}

The transmission coefficient is $T=\frac{|E_t|^2}{|E_{in}|^2}$.
\end{comment}


%***%\chapter{\texorpdfstring{Relation of $T/{\cal E}$}{T/E} to \texorpdfstring{$D(z)$}{D(z)}} %  From Self-consistent Theory of Anderson Localization
\label{sec:appendix_TE_Dz_relation}

% see also lab notebook SVN 20091230_ben_transmission_energy_derivations
% Note: a significant amount of content was barrowed from /svn/research/oned/TE\_paper

This is an expansion of Appendix section~\ref{app:Dz_derivation}. As in that section we assume a slab geometry. The $z$ coordinate normal to the slab is separated from the perpendicular component ${\bf \rho}$ as ${\bf r}=({\bf \rho},z)$. Again assuming no dependence on ${\bf \rho}$ allows us to give the ensemble-averaged diffusive flux $\langle\vec{J}(\vec{r},t)\rangle$ and the energy density $\langle {\cal W}(\vec{r},t)\rangle$ are related via \cite{1953_Morse}
\begin{equation}
\langle\vec{J}(\vec{r},t)\rangle=-D(\vec{r})\vec{\nabla}\langle {\cal W}(\vec{r},t)\rangle
\label{eq:Jflux_relation}
\end{equation}
The diffusion approximation amounts to $D(\vec{r})\equiv D_0=c\ell_{tmfp}/3$, where $c$ is the speed of light and $\ell_{tmfp}$ is the transport mean free path.

We consider a 3D random medium in a shape of a slab of thickness $L$, where we explicitly  separate the coordinate $z$ normal to the slab from the perpendicular component ${\bf \rho}$ as ${\bf r}=({\bf \rho},z)$. Under a CW plane-wave illumination at normal incidence, the dependence on ${\bf \rho}$ and $t$ can be neglected. 
\begin{equation}
\langle\vec{J}_z(z)\rangle=-D(z)\frac{d}{dz}\langle {\cal W}(z)\rangle
\end{equation}

Integration over $z$ gives
\begin{equation}
\int_z^L\frac{\langle J_z(z^\prime)\rangle dz^\prime}{D(z^\prime)}=-\langle {\cal W}(L)\rangle + \langle {\cal W}(z)\rangle
\label{eq:E1_relation}
\end{equation}
where the energy stored inside the random medium ${\cal E}$ is formally defined as
\begin{equation}
\langle {\cal E} \rangle =\int_0^L\langle {\cal W}(z)\rangle dz.
\label{eq:Energy_definition_relation}
\end{equation}
thus
\begin{equation}
\langle {\cal E} \rangle = \int_0^L \left( \langle {\cal W}(L)\rangle + \int_z^L\frac{\langle J_z(z^\prime)\rangle }{D(z^\prime)}dz^\prime\right) dz
\end{equation}
The remaining work is to factor out transmission $T$ in order to find the relation between $T/{\cal E}$ and $D(z)$. The energy density $\langle {\cal W}(L)\rangle$ at the right boundary can be expressed in terms of right- and left-propagating fluxes. From the definition of diffusive flux \cite{1953_Morse}
\begin{equation}
\langle J_{\pm}(z)\rangle = \frac{c}{4} \langle {\cal W}(z)\rangle \mp \frac{D_0}{2} \frac{d\langle {\cal W}(z)\rangle }{dz}
\label{eq:diffusive_flux_relation}
\end{equation}
where $ \langle J_{-}\rangle$ and $ \langle J_{+}\rangle $ are the fluxes propagating along negative and positive $z$-directions respectively. Since $\langle J_+(L)\rangle=J_0T$ and $\langle J_-(L)\rangle=0$, using Eqs.~\ref{eq:diffusive_flux_relation} to eliminate $D_0$ yields
\begin{equation}
\langle J_+(L)\rangle + \langle J_-(L)\rangle = 2 \frac{c}{4}\langle {\cal W}(L)\rangle
\end{equation}
Therefore $\langle {\cal W}(L)\rangle=2J_0T/c$ and the energy can be re-written as
\begin{equation}
\langle {\cal E} \rangle = \int_0^L \left( 2J_0T/c + \int_z^L\frac{\langle J_z(z^\prime)\rangle}{D(z^\prime)}dz^\prime\right) dz
\end{equation}
Next, we reduce $\langle J_z(z^\prime)\rangle$ to find an approximately equivalent transmission.

%the boundary conditions are from Eq.~\ref{eq:Jflux_conserv}
%\begin{equation}
%J_z(z=0)=-J_0R,\ \, J_z(z=L)=J_0T
%\label{eq:Jflux_bc}
%\end{equation}

In the CW regime when the energy density ${\cal W}(z)$ is stationary, $\partial \langle {\cal W}(z)\rangle/\partial t=0$, it follows from energy conservation condition for flux $\vec{J}$ and energy ${\cal W}$
\begin{equation}
\frac{\partial \langle {\cal W}(\vec{r},t)\rangle }{\partial t}+\vec{\nabla} \cdot \langle\vec{J}(\vec{r},t)\rangle=
\frac{c}{\ell_g}\langle {\cal W}(\vec{r},t)\rangle+J_0 \delta(z-z_p)
\label{eq:Jflux_conserv_relation}
\end{equation}
 that the $z$ component of flux is constant for $z>z_p\sim\ell$. The value of the constant can be obtained from the boundary condition at $z=L$ as
\begin{equation}
\langle J_z(z)\rangle=\left\{
\begin{array}{l l}
\langle J_z(L)\rangle\equiv J_0 \langle T \rangle ,&\quad z_p<z<L\\
\langle J_z(0)\rangle\equiv -J_0 \langle R \rangle,&\quad 0<z<z_p\\
\end{array} \right.
\label{eq:Jfluxz_const_relation}
\end{equation}
where $T$ ($R$) is the transmission (reflection) coefficient. As a check, by integrating Eq.~(\ref{eq:Jflux_conserv_relation}) over the entire system we obtain the standard (passive) flux conservation $\langle J_z(L)\rangle -\langle J_z(0)\rangle =J_0 \langle T \rangle-(-J_0 \langle R \rangle)=J_0(\langle T \rangle+\langle R \rangle)=J_0$. To take advantage of the fact that $\langle J_z(z)\rangle$ is piecewise constant, c.f. Eq.~(\ref{eq:Jfluxz_const_relation}), we have to neglect by $0<z<z_p$ contribution. Then a constant can be substituted for $J_z(z')$,
\begin{equation}
\langle {\cal E} \rangle = \int_0^L \left( 2J_0\langle T \rangle/c + \int_z^L\frac{J_0 \langle T \rangle }{D(z^\prime)}dz^\prime\right) dz
\end{equation}
This introduces an error $\propto z_p/L\sim\ell/L\ll 1$. Factoring $T$ from the integrands,
\begin{equation}
\langle {\cal E} \rangle =J_0\langle T \rangle\int_0^L \left( \int_z^L\frac{dz^\prime}{D(z^\prime)}+2/c\right) dz
\label{eq:E1a_relation}
\end{equation}
Note that the second term is of the same order $\sim \ell/L$ as the term omitted in arriving to the above expression. Hence, $2/c$ contribution has to be dropped as well.
\begin{equation}
\langle {\cal E} \rangle =J_0T\int_0^L \int_z^L \frac{1}{D(z^\prime)}dz^\prime dz
\end{equation}

Taking advantage of the system symmetry, $D(z)=D(L-z)$, the double integration can be further simplified as
\begin{eqnarray}
\displaystyle\int_{0}^{L}\int_{z}^{L}\displaystyle\frac{1}{D(z^\prime)}dz^\prime dz &=&\frac{1}{2}\displaystyle\int_{0}^{L}\int_{0}^{L}\displaystyle\frac{1}{D(z^\prime)}dz^\prime dz \nonumber\\
&=&\frac{L}{2}\int_{0}^{L}\displaystyle\frac{1}{D(z)}dz.
\label{eq:E3_relation}
\end{eqnarray}
After normalizing the integral so that it yields unity in the case when the wave interference effects are neglected, $D(z)=D_0\equiv c\ell/3$, for passive media
\begin{equation}
\frac{\langle T \rangle}{\langle {\cal E} \rangle}\simeq
\frac{1}{J_0}
\frac{2D_0}{L^2}
\left(
\displaystyle\frac{1}{L}\displaystyle\int_{0}^{L}\displaystyle\frac{D_0}{D(z)}dz
\right)^{-1},
\label{eq:TE_vs_D_relation}
\end{equation}
We note that in process of deriving Eq.~(\ref{eq:TE_vs_D_relation}), we dropped the terms on the order of $\sim\ell/L\ll 1$.

Dropping the localization corrections leaves
\begin{equation}
\frac{\langle T \rangle}{\langle {\cal E} \rangle}\simeq \frac{1}{J_0} \frac{2D_0}{L^2}
\label{eq:diffusion_te_only}
\end{equation}
Any deviation from Eq.~\ref{eq:diffusion_te_only} in passive diffusive media can be attributed to localization corrections.
 
%\end{ThesisAppendix}


\bibliographystyle{plain}
\newpage
\bibliography{bibliography}

\begin{Vita}
% Describe myself.
Stephen Curtis Jackson was born in Kansas City, Missouri.
He received his Bachelor's degrees in Computer Engineering and Computer Science from Missouri University of Science and Technology in December 2010.
Afterward, he joined the Computer Science Ph.D. Program in January 2011 under the Information Assurance GAANN fellowship with Dr. Bruce McMillin as his research advisor.
His research interests were in Markov models, cyber-physical systems, and distributed systems.
He was awarded his Ph.D. in Computer Science from the Missouri University of Science and Technology in July 2016.

\end{Vita}

%
%***%\newpage 
%***%\printglossary
% make sure the glossary shows up in table of contents:
%***%\addcontentsline{toc}{chapter}{Glossary}

%% \newpage
% \printindex
% \addcontentsline{toc}{chapter}{Index}


\end{document}
\endinput
