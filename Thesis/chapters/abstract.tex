\begin{abstract}
Cyber-physical systems (CPS) are an attractive option for future development of
critical infrastructure systems. By supplementing the traditional physical
network with cyber control, the performance and reliability of the system
can be increased. In some of these networks, distributing the cyber control
offers increased redundancy and availability during fault conditions. However,
there are very few works which study the effects of cyber faults on a 
distributed cyber-physical system. These are of a particular interest in the Smart Grid
environment where outages and failures are very costly. By examining the
behavior of a distributed system under fault scenarios, the overall robustness
of the system can be improved by planning characteristics and responses to
faults that allow the system to continue operating in difficult circumstances.

This work examines the consequences of network unreliability on a core part of the
 Distributed Grid Intelligence (DGI) for the FREEDM (Future Renewable Electric 
Energy Delivery and Management) Project. By applying different rates of packet
loss in specific configurations to the communication stack of the software and
observe the behavior of a critical component (Group Management) under those
conditions. These components identify the amount of time spent in a group,
working, as a function of the network reliability. Given this one can parameterize
the amount of messages that are lost and thus the number of failed physical actuations.
Knowing this and the physical characteristics of the system, one can tune the amount
of time between reconfiguration in order to prevent the number of failed physical
changes from causing the system to become unstable. Using this, we can develop
methods and guarantees to protect the operation of a CPS.
\end{abstract}
