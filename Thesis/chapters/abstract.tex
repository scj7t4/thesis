\begin{abstract}
Cyber-physical systems (CPS) can improve the reliability of our critical infrastructure systems. By augmenting physical distribution systems with digital control through computation and communication, one can improve the overall reliability of a network. However, a system designer must consider what effects system unreliability in the cyber-domain can have on the physical distribution system. In this work, we examine the consequences of network unreliability on a core part of a Distributed Grid Intelligence (DGI) for the FREEDM (Future Renewable Electric Energy Delivery and Management) Project. To do this, we apply different rates of packet loss in specific configurations to the communication stack of the software and observe the behavior of a critical component (Group Management) under those conditions. These components will allow us to identify the amount of time spent in a group, working, as a function of the network reliability.
\keywords{cyber-physical systems, critical infrastructure, reliability, leader election, stability}
\end{abstract}
