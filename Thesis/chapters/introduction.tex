\chapter{Introduction}
A robust cyber-physical system should be able to survive and adapt to communication network outages in both the physical and cyber domains. When one of these outages occurs, the physical or cyber components must take corrective action to allow the rest of the system to continue operating normally. Additionally, process may need to react to the state change of some other process. Managing and detecting when other processes have failed is commonly handled by a leader election algorithm and failure detector.

In the smart grid domain, leader elections are an attractive option for autonomously configuring cyber components so processes can coordinate together to manage the physical system. Algorithms such as \cite{LOADBALANCING} and \cite{INCREMENTALCONSENSUS} are distributed algorithms for managing power in a smart grid that rely on an assumption that a group of processes will be able to coordinate together. 

FREEDM (Future Renewable Electric Energy Delivery and Management) is a Smart Grid project focused on the future of the electrical grid. Major proposed features of the FREEDM network include the Solid State Transformer (SST), local energy storage, and local energy generation\cite{FREEDMMIGRATION}. This vein of research emphasizes decentralizing the power grid: making it more reliable by distributing energy production resources. Part of this design requires the system to operate in “islanded mode”, where portions of the distribution network are segmented from each other. However, there is a major shortage of work on the effects cyber outages have on CPSs \cite{CYBERRESEARCHCALL} \cite{SMARTGRIDBENEFITS}. However, research that has been done, such as \cite{HARINI} indicate that cyber faults can cause a physical system to apply unstable settings.

This work begins with observations on the effects of network unreliability on the group management module of the Distributed Grid Intelligence (DGI) used by the FREEDM smart-grid project. This system uses a broker system architecture to coordinate several software modules that form a control system for a smart power grid. These modules include: group management, which handles coordinating processes via leader election; state collection, a module which captures a global system state; and load balancing which uses the captured global state to bring the system to a stable state.

This work presents the initial steps to better understanding and planning for these faults. As part of an analysis of omission failures, we present an approach in modeling the grouping behavior of a system using Markov chains. These chains produce expectations of how long a system can be expected to stay in a particular state, or how much time it will be able to spend coordinating and doing useful work over a period of time. Using these measures, the behavior of the cyber system can be specified to prevent faults.

In Chapter 2, we outline the state of the art: other distributed frameworks and works similar to our own. Chapter 3 provides a background of the architecture of DGI and the properties of the created Markov models. Next, Chapter 4 shows the results that have been collected so far. The final chapter summarizes the work and research goals.
