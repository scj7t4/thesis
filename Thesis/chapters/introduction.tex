\chapter{Introduction}

In the smart grid domain, leader elections are an attractive option for
automonously configuring cyber components. Proposed algorithms such as
\cite{LOADBALANCING} and \cite{INCREMENTALCONSENSUS} are distributed
algorithms for managing power in a smart grid rely on an assumption that
a group of peers will be able to coordinate together. In a system where
100\% up time is not guaranteed, leader
elections are a promising method of establishing these groups.

A strong cyber-physical system should be able to survive and adapt to network
outages in both the physical and cyber domains. When one of these outages
occurs, the physical or cyber components must take corrective action to allow
the rest of the network to continue operating normally. Additionally, other
nodes may need to react to the state change of the failed node. In the realm
of computing, algorithms for managing and detecting when other nodes have
failed is a common distributed systems problem known as leader election.

This work observes the effects of network unreliability on the the group
management module of the Distributed Grid Intelligence (DGI) used by the
FREEDM smart-grid project. This system uses a broker system architecture to
coordinate several software modules that form a control system for a smart
power grid. These modules include: group management, which handles coordinating
nodes via leader election; state collection, a module which captures a global
system state; and load balancing which uses the captured global state to bring
the system to a stable state.

FREEDM (Future Renewable Electric Energy Delivery and Management) System
is a Smart Grid project focused on the future of the electrical grid. Major proposed features
of the FREEDM network include the Solid State Transformer, distributed local energy storage,
and distributed local energy generation\cite{FREEDMMIGRATION}. This vein of research emphasizes decentralizing the power grid: making it more reliable by distributing energy production resources. Part of this
design requires the system to operate in islanded mode, where portions of the distribution
network are segmented from each other. However, there is a major shortage of work
within the realm of the effects cyber outages have on CPSs
\cite{CYBERRESEARCHCALL} \cite{SMARTGRIDBENEFITS}. Addtitionally, research that
has been done such as \cite{HARINI} indicate that cyber faults can cause a physical
system to apply unstable settings.

This work presents the initial steps to better understanding and planning for these faults.
By taking a new approach to considering how a distributed system interacts during a fault condition,
new techniques for managing a fault scenario in a cyber-physical systems will be created. To do
this, we present an approach in modeling the grouping behavior of a system using Markov chains.
These chains produce expectations of how long a system can be expected to stay in a particular
state, or how much time it will be able to spend coordinating and doing useful work over a period
of time. Using these measures, the behavior of the control system for the physical devices
can be adjusted to prevent faults.
