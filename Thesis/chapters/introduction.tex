\chapter{Introduction}

%%%%%%%%%%%%%%%%%%%%%%%%%%%%%%%%%%%%%%%%%%%%%%%%%%%%%%%%%%%%%%%%%%%%%%%%%%%%%%
Historically, leader elections have had limited applications in critical
systems. However, in the smart grid domain, there is a great opportunity to
apply leader election algorithms in a directly beneficial way.
\cite{LOADBALANCING} presented a simple scheme for performing power
distribution and stabilization that relies on formed groups. Algorithms like
Zhang, et. al's Incremental Consensus Algorithm \cite{INCREMENTALCONSENSUS},
begin with the assumption that there is a group of nodes who coordinate to
distribute power. In a system where 100\% up time is not guaranteed, leader
elections are a promising method of establishing these groups.

A strong cyber-physical system should be able to survive and adapt to network
outages in both the physical and cyber domains. When one of these outages
occurs, the physical or cyber components must take corrective action to allow
the rest of the network to continue operating normally. Additionally, other
nodes may need to react to the state change of the failed node. In the realm
of computing, algorithms for managing and detecting when other nodes have
failed is a common distributed systems problem known as leader election.

This work observes the effects of network unreliability on the the group
management module of the Distributed Grid Intelligence (DGI) used by the
FREEDM smart-grid project. This system uses a broker system architecture to
coordinate several software modules that form a control system for a smart
power grid. These modules include: group management, which handles coordinating
nodes via leader election; state collection, a module which captures a global
system state; and load balancing which uses the captured global state to bring
the system to a stable state.

It is important for the designer of a cyber-physical system to consider what
effects the cyber components will have on the overall system. Failures in the
cyber domain can lead to critical instabilities which bring down the entire
system if not handled properly.  In fact, there is a major shortage of work
within the realm of the effects cyber outages have on CPSs
\cite{CYBERRESEARCHCALL} \cite{SMARTGRIDBENEFITS}. In this paper we present a
slice of what sort of analysis can be performed on a distributed cyber control
by subjecting the system to packet loss. The analysis focuses on quantifiable
changes in the amount of time a node of the system could spend participating in
energy management with other nodes.

This work is meant to be an investigation cyber control on a physical system in
the face of faults. While a considerable amount of work has been done on the
physical components of a smart grid, there is a much smaller body of work on what
effects distributed computer control could have on this grid. This work presents
the ground work for greater work involving testing the cyber system with both simulators
and physical hardware to understand what effects this control on the system could have.
A robust CPS should not degrade the performance of the system. For example, a partition
in the cyber network, but not in the physical ideally should not be able cause instability
for the system as a whole.

The culmination of this work will be the development of a model of a partitioning
environment with omission faults to predict the stability of groups for a given system
configuration. This information can be used to determine how long load balancing operations
can be run in an environment without the omitted messages causing the cyber control to
make changes which destabilize the system.
