\chapter{Results}

Initial results were collected using the experimental platform. As described in
the experimental procedure chapter, various system topologies were tested with
the described packet loss rates. Tests using the experimental platform were run
as many as 40 times. The collected results have been divided into two sections:
SRC and SUC, the two delivery protocols used during testing.

The first minute of each test in the experimental test is discarded to remove
any transients in the test. The result is that while the tests were run for
ten minutes, the maximum result is 9 minutes of in group time.

\section{SRC}

\subsection{Two Node Case}

\begin{figure}[!h]
\centering
\includegraphics{2NODE-SRC-100-SIZE.pdf}
\caption{Average size of formed groups for two node system with 100ms resend time}
\label{fig:MGS-2NODE-100}
\end{figure}

\begin{figure}[!h]
\centering
\includegraphics{2NODE-SRC-100-GROUP.pdf}
\caption{Average size of formed groups for two node system with 100ms resend time}
\label{fig:IGT-2NODE-100}
\end{figure}

The 100ms resend SRC test with two nodes can be considered a sort of a control.
These tests, pictured in Figures \ref{fig:MGS-2NODE-100} and
\ref{fig:IGT-2NODE-100}. This test highlights the excellent performance of the
SRC protocol, achieving the maximum in group time of 9 minutes with only 15\%
of datagrams arriving at the reciever. Figure \ref{fig:MGS-2NODE-100} shows
that when there is no chance of datagrams arriving, the maximum group size is
one since no elections can occur. As the reliability increases, more time is
spent in a group. Since the maximum group size is 2, it is directly related to
the in group time.

\begin{figure}[!h]
\centering
\includegraphics{2NODE-SRC-200-SIZE.pdf}
\caption{Average size of formed groups for two node system with 200ms resend time}
\label{fig:MGS-2NODE-200}
\end{figure}

\begin{figure}[!h]
\centering
\includegraphics{2NODE-SRC-200-GROUP.pdf}
\caption{Average size of formed groups for two node system with 200ms resend time}
\label{fig:IGT-2NODE-200}
\end{figure}

Figures \ref{fig:MGS-2NODE-200} and \ref{fig:IGT-2NODE-200} demonstrates that as the
rate at which lost datagrams are resent is decreased to resend every 200ms the
time in group falls off. This behavior is expected, since each exchange has a
time limit for each message to arrive and the number of attempts is reduced by
increasing the resend time.

\subsection{Transient Partition Case}

\begin{figure}[!h]
\centering
\includegraphics{TRANS-SRC-100-SIZE.pdf}
\caption{Average size of formed groups for the transient partition case with 100ms resend time}
\label{fig:MGS-TRANS-100}
\end{figure}

\begin{figure}[!h]
\centering
\includegraphics{TRANS-SRC-100-GROUP.pdf}
\caption{Average size of formed groups for the transient partition case with 100ms resend time}
\label{fig:IGT-TRANS-100}
\end{figure}

The transient partition case, shows a simple example where a network partition
seperates two groups of DGI. In the simplest case where the opposite side of
the partition is unreachabe, nodes will form a group with the other nodes on the
same side of the partition. In our tests, there are two nodes on each side of
the partition. In the experiment, the probability of a datagram crossing the
partition is increased as the experiment continues. The 100ms case is shown in
Figures \ref{fig:MGS-TRANS-100} and \ref{fig:IGT-TRANS-100}.

While messages cannot cross the partition, the DGIs stay in a group with the
nodes on the same side of the partition leading to an in group time of 9 minutes,
the maximum value. As packets begin to cross the partition (with the reliability
increasing), DGI instances on either side begin to attempt to complete elections
with the nodes on the opposite partition and the time in group begins to fall.
However during this time, the mean group size continues to increase, meaning
while the elections are decreasing the amount of time that the module spends in
state where it can actively do work, it typically does not fall into a state
where it is in a group by itself, which means that most of the lost in group
time comes from elections.


\begin{figure}[!h]
\centering
\includegraphics{TRANS-SRC-200-SIZE.pdf}
\caption{Average size of formed groups for the transient partition case with 200ms resend time}
\label{fig:MGS-TRANS-200}
\end{figure}

\begin{figure}[!h]
\centering
\includegraphics{TRANS-SRC-200-GROUP.pdf}
\caption{Average size of formed groups for the transient partition case with 200ms resend time}
\label{fig:IGT-TRANS-200}
\end{figure}

The 200ms case displays similiar behavior, with a wider valley due to the
limited number of datagrams. 
