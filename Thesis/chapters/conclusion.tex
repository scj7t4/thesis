\chapter{Conclusion}

This work presented a new approach for predicting the behavior of a real-time distributed system under omission failure conditions. By using a continuous time Markov chain, a variety of insights can be gathered about the system, including observations such as how long a particular configuration will be stable for, and the behavior of the system in the long run.  The Markov results will be used  to make better real time schedules to better react to the network faults we plan on introducing to our test beds. The primary concern are scenarios in which the cyber controller attempts to make physical components which are not connected in the physical network interact, and scenarios where a fault in the cyber network causes the paired events (where two physical controllers change to accomplish some transaction or exchange) to only be partially executed. For example, in the DGI load balancing scheme, a node in a supply state injects a quantum of power into the physical network, but the node in the demand state does not change to accept it. These errors, which are the primary focus of this work could cause instability if a sufficient number of these failed exchanges occur. In \cite{HARINI}, Choudhari et. al. show that failed transactions can create a scenario where the frequency of a power system could become unstable. 

Moving forward, we have identified these areas as targets for improving the research done and creating new contributions.

\section{Time between reconfigurations}

 The rate that the system should reconfigure then is a function of the maximum number of failed migrations that the system can take, the time it takes to write to the channel and the time it takes process messages. The amount of time in group can also be consideration for which algorithm to select based on the needed amount of time to perform its work. Group Management can be used as a critical component in a real-time distributed system to manage the number of lost messages and as a consequence, the number of failed migrations in a CPS. It is critical to understand how frequently nodes enter and exit the group based on lost messages and how many migrations fail as a consequence of those messages. This area is deficient because it is strongly coupled to the interactions with the physical component: we must understand how the cyber configuration and physical changes made by that configuration can affect the system, and establish when reconfigurations should occur to keep the system stable. To correct this, we hope to develop a mathematical relationship between the stability of the group, and the physical management functionality of the CPS.

\section{Correctness of an Installed Configuration}

 The work presented in this document is probabilistic: the results of a leader election are random and based only on responses arriving with in a specified period of time. Other factors can affect what configurations can be installed such as trust in the parties in the group, the underlying physical topology, and the reliability of the peers in that group. We hope to develop guarantees on the properties of a configuration that protect the physical topology and the members of the group. These guarantees would also allow processes to better police the configurations they are installed in, in order to protect the system from malicious nodes.

\section{Accuracy of The Model}

We recognize that the model presented thus far does not have ideal accuracy. We expect to be able to further refine the model and the algorithms and formulas for generating the model. There are some features of the behavior of the DGI which are not completely encapsulated in the model. Additionally, the way models are specified can be generalized to support more systems of similar design. As part of this work however, we must remember modeling distributed systems is extremely difficult and while the increase in accuracy is desirable it is not a critical goal.

\section{Scope of the Model}

The models presented in this work focus only on the leader election component of a dynamically configured CPS. Additional work would incorporate additional components of the DGI system into the models for a more complete picture of the behavior of the system during failures. We will consider the correctness of the incorporated algorithms, how omission failures can violate that correctness, and what restrictions we can place on the configuration and operation of DGIs in order to protect the entire system during failures. To do this, we will expand the analysis performed here to incorporate algorithms such as state collection and load balancing and define metrics to quantify their behavior during omission failures. This thrust will pair with the correctness and time between reconfigurations: different algorithms will have different amounts of failure that can be allowed before reconfiguration is necessary.

\section{Deliverables}

Therefore, moving forward, we will expand the models we have presented here to include more of the properties of the complete CPS. This model will allow us to better understand what effects the group behavior has on the CPS. Using this, we can establish invariants which allow us to ensure the correctness of a CPS by providing assertions which will not be broken during execution. Creating these invariants will allow us to improve the development of CPS es, especially in their dynamic configuration, which is an area with limited development. These invariants also allow us to create an assertion of correctness which can be validated, during runtime, to ensure the system maintains its stability. We will continue to create and validate models of the CPS against simulations and actual hardware. As we do so we will construct invariants that describe the correct behavior of the groups to ensure safe operation. In addition to the existing conference paper \cite{CRITIS2012}, we have prepared a journal paper with the new work in this document. Additionally, we expect produce at least 2 additional publications as part of the thrusts presented. 
