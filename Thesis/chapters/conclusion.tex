\chapter{Conclusion}

This work presented a new approach for predicting the behavior of a real-time distributed system under omission failure conditions. By using a continuous time Markov chain, a variety of insights can be gathered about the system, including observations such as how long a particular configuration will be stable for, and the behavior of the system in the long run.  The Markov results will be used  to make better real time schedules to better react to the network faults we plan on introducing to our test beds. The primary concern are scenarios in which the cyber controller attempts to make physical components which are not connected in the physical network interact, and scenarios where a fault in the cyber network causes the paired events (where two physical controllers change to accomplish some transaction or exchange) to only be partially executed. For example, in the DGI load balancing scheme, a node in a supply state injects a quantum of power into the physical network, but the node in the demand state does not change to accept it. These errors, which is the primary focus of this work could cause instability if a sufficient number of these failed exchanges occur. In \cite{HARINI}, Choudhari et. al. show that failed transactions can create a scenario where the frequency of a power system could become unstable. 

Moving forward, we have identfied these areas as targets for improving the research done and creating new contributions.

\subsection{Consequences of Bad Configurations}

A major issue, and an area of limited research, is what effect a bad configuration can have on a CPS. For example, in the vein of research presented in this work, we consider omission failures and what affect they have on the configuration of peers in the system. Omission failures can lead to nodes failing to act, or taking actions that are not compensated for by another peer. This work examines what affect these failures have on the leader election algorithm, we have yet to consider what affect it has on algorithms such as load-balancing. Future work in this vein will focus on two components:

\begin{itemize}
\item Time between reconfigurations - The rate that the system should reconfigure then is a function of the maximum number of failed migrations that the system can take, the time it takes to write to the channel and the time it takes process messages. The amount of time in group can also be consideration for which algorithm to select based on the needed amount of time to perform its work. Group Management can be used as a critical component in a real-time distributed system to manage the number of lost messages, and as a consequence the number of failed migrations in a CPS. It is critical to understand how frequently nodes enter and exit the group based on lost messages and how many migrations fail as a consequence of those messages. This area is deficent because it is strongly coupled to the interactions with the physical component: we must understand how the cyber configuration and physical changes made by that configuration can affect the system, and establish when reconfigurations should occur to keep the system stable. To correct this, we hope to develop a mathematical relationship between the stability of the group, and the physical management functionality of the CPS.

\item Correctness of an Installed Configuration - The work presented in this document is probablistic: the results of a leader election are random and based only on responses arriving with in a specified period of time. Other factors can affect what configurations can be installed such as trust in the parties in the group, the underlying physical topology, and the reliability of the peers in that group. We hope to develop gaurntees on the properties of a configuration that protect the physical topology and the members of the group. These gaurntees would also allow processes to better police the configurations they are installed in, in order to protect the system from malicious nodes.
\end{itemize}

\subsection{Accuracy of The Model}



\subsection{Scope of the Model}

\subsection{Deliverables}

Therefore, moving forward, we will expand the models we have presented here to include more of the properties of the complete CPS. This model will allow us to better understand what effects the group behavior has on the CPS. Using this, we can establish invariants which allow us to establish the correctness of a CPS by providing assertions which will not be broken during execution. Creating these invariants will allow us to improve the development of CPSes, especially in their dynamic configuration, which is an area with limited development. These invariants also allow us to create an assertion of correctness which can be validated, during runtime, to ensure the system maintains its stability. We will continue to create an validate models of the CPS against simulations and actual hardware. As we do so we will construct invariants that describe the correct behavior of the groups of the system that ensure safe operation. 
