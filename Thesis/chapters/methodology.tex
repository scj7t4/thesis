\chapter{Experimental Design}

Tests were the system were completed by applying network settings and then running the nodes in the prescribed configuration for ten minutes (using the UNIX timeout command). At this point the test was terminated and the group management system appends statistics to an output file. New settings were applied and the next test was begun.

\section{Tools Used, Systems Used}

The application of settings and the initiation of tests was completed using a custom script written in Python. This script used a library, Fabric \cite{FABRIC}, to start runs of the system by the secure shell (SSH). This was run on one of the machine and monitored the I/O of all nodes to ensure everything was behaving correctly.

Our experimental software also provided for ``bussing,'' where a group of edges would
have the same reliability and were iterated together, and ``fixing,'' which allowed for edges that would not change reliability across any of the runs.

All tests were run on four Pentium 4 3GHz machines with 1GB of RAM and Hyper-threading. Tests were run on an ArchLinux install using a real-time kernel, however, the snapshot of the FREEDM software used to run the tests does not feature a real-time scheduler.

The testing software was responsible for initializing instances, allowing them to run and then terminate after a fixed time limit. Additionally it provided an iterative object which generated network settings which were copied to the target machines before each test began.

Each node recorded its own state information, which was appended to a log file at termination of the run. This data was then coupled with the experimental procedure data to create the tables and charts in the results.

For each run of the system, the first 60 seconds of the system were not logged to filter out transients. This leads to a maximum recordable in-group time of nine minutes.
\section{Tests Performed}

Our experiments considered two configurations of the system which can be considered highly characteristic of most other scenarios. The first, a two node configuration was intended to observe a slice of the behavior of the system when two nodes (a leader and a group member) struggle to communicate with one another.

The second configuration was a four node configuration with a transient partition, where the nodes were divided into pairs. Each pair of nodes could reliably communicate with each other, but reliable communication across pairs was not guaranteed. We would vary the reliability of the connection between the pairs and observe the effects on the system.

For both tests, we ran the system using both our sequenced reliable protocol as well as our sequenced unreliable protocol. Additionally, we varied the amount of time between resends for both protocols. A full list of the tests we performed are listed in Table \ref{TableTests}.

In each test, we recorded the number of elections which began, the number that completed successfully, the amount of time spent working on elections, the amount of time spent in a group, and the mean group size. Using these metrics, we hoped to capture a good representation of what kind effects network problems could have on the stability of the groups formed.

\begin{table}[!t]
% increase table row spacing, adjust to taste
\renewcommand{\arraystretch}{1.3}
\caption{Tests Performed}
\label{TableTests}
\centering
% Some packages, such as MDW tools, offer better commands for making tables
% than the plain LaTeX2e tabular which is used here.
\begin{tabular}{c c c c c}
\hline
Test No. & Test Type & Protocol & Resend Time & Window Size \\ \hline \hline
1        & 2 Node    & SRC      & 200ms       & N/A \\ \hline
2        & 2 Node    & SUC      & 200ms       & 8 \\ \hline
3        & 2 Node    & SRC      & 100ms       & N/A \\ \hline
4        & 2 Node    & SUC      & 100ms       & 8 \\ \hline
5        & Transient & SRC      & 200ms       & N/A \\ \hline
6        & Transient & SUC      & 200ms       & 8 \\ \hline
7        & Transient & SRC      & 100ms       & N/A \\ \hline
8       & Transient & SUC      & 100ms       & 8 \\ \hline
\end{tabular}
\end{table}
