\section{Background}

\subsection{Group Management}

The DGI uses the leader election algorithm, ``Invitation Election Algorithm,'' written by Garcia-Molina \cite{INVITATIONELECTION}.
This algorithm provides a robust election procedure which allows for transient partitions.
Transient partitions are formed when a faulty link between two or more clusters of processes causes the groups to divide temporarily.
These transient partitions merge when the link becomes more reliable.

Many election algorithms have been created.
Specialized algorithms exist for wireless sensor networks \cite{LE-WSN-1}\cite{LE-WSN-2} and other special circumstances\cite{LE-SPECIALCIRCUMSTANCES-1}\cite{LE-SPECIALCIRCUMSTANCES-2}.
Work on leader elections has been incorporated into a variety of distributed frameworks: Isis \cite{ISISTOOLKIT}, Horus \cite{HORUSTOOLKIT}, Totem \cite{TOTEMTOOLKIT}, Transis \cite{TRANSISTOOLKIT}, and Spread \cite{SPREADTOOLKIT} all have methods for creating groups.
Despite this wide array of work, the fundamentals of leader election are similar.
Processes arrive at a consensus of a single peer that coordinates the group.
Processes that fail are detected and removed from the group.

The elected leader is responsible for making work assignments, identifying and merging with other coordinators when they are found, and maintaining an up-to-date list of peers.
Group members monitor the group leader by periodically checking if the group leader is still alive by sending a message.
If the leader fails to respond, the querying peers will enter a recovery state and operate alone until they can identify another coordinator.
Therefore, a leader and each of the members maintain a set of processes that are currently reachable, a subset of all known processes in the system.

Using a leader election algorithm allows the FREEDM system to autonomously reconfigure rapidly in the event of a fault.
Cyber-components are tightly coupled with the physical components, and fault reaction is not limited to the cyber domain.
Processes automatically react to crash-stop failures, network issues, and power system faults.
The automatic reconfiguration allows processes to react immediately to issues, faster than a human operator, without relying on a central configuration point.
However, it is important the configuration leader election supplies is one where the system can do viable work without causing physical faults\cite{HARINI}.

\subsection{Markov Chain}

%When Markov chains work to model a system. Something about order of Markov chain. State what a Markov chain and the properties of the Markov chain. State what states mean. state what transition probabilies mean. State what ergodicity is. State what stationarity is. State how we generate the sequence of states.

A Markov chain is a collection of states and probabilistic transitions between those states.
States in a Markov chain are mutually exclusive.
In a Markov chain, when a system is some state $i$ it has some probability of transitioning to some other state $j$ at the next time-step.
A Markov chain is a first order chain if the probability of transitioning from $i$ to $j$ does not depend of the history of transitions that lead to state $i$.
First order chains are described as having a memoryless or Markov property.
This formalizes the independence of the next state from the history of previous states.
The Markov property describes a Markov chain as a sequence of random variables $X_{1}, X_{2}, X_{3}...$ and states the value of $X_{n+1}$ only depends $X_{n}$: \cite{MARKOV3}

\[ \Pr(X_{n+1}=x\mid X_1=x_1, X_2=x_2, \ldots, X_n=x_n) = \Pr(X_{n+1}=x\mid X_n=x_n) \]

An ergodic Markov chain is a chain where it is possible, in some finite number of steps, to go from any state to any other state.
A stationary Markov chain is one where the transition probabilities do not change over time.
In a stationary Markov chain, the $n$th visit to a state is indistinguishable from the $n+1$th visit to a state.

A Markov chain with $m$ states can be represented by a $m\times m$ matrix.
For simplicity when creating the model, matrices in this work are 1-indexed.
In a matrix $P$, the value of $P_{ij}$ represents the probability of the transition from $i$ to $j$.
It should be obvious that the sum of each row in the matrix is equal to one:

\[ \sum_{i=1}^{m} P_{ij} = 1 \]

A useful companion to the transition matrix is a state-distribution vector.
This $m$-dimensional vector (indexed from 1) describes the probability of observing a system state at a given moment.
The $i$th entry in the vector corresponds to the probability of observing state $i$.
For example, if an observation always observes a two state system in state $i=1$ the state-distribution would be noted as $[1.0 \quad 0]$.

A Markov chain is a suitable model for a memoryless random process with a finite number states which is observed at fixed time intervals.
By utilizing a Markov chain, a variety of statistical analyses can be performed on these modeled system.
For example, a Markov chain with the stationary and ergodic properties can be analyzed for its steady state probabilities.
The steady state describes the probability a random observation of a running process will observe some state $i$.
The steady state can by found via a system of equations: \cite{MARKOV3}

\[  0\leq\pi_j\leq1.0 \]
\[	\sum_{j = 1}^{m}\pi_j = 1.0 \]
\[	\pi_j = \sum_{i=1}^{m} \pi_i p_{ij} \]

where $\pi$ is a vector with a length equal to the number of states in the Markov chain.
In the following sections, the computation of the steady state will be noted as $Steady()$.
A Markov chain can also be used to predict what state a process will be in at some point in the the future.
Given a initial state and a number of time-steps a matrix operation will yield the likelihood of the process being in each state after the time interval has passed.
The mean passage time, a measure of how many time-steps will pass before a process returns or arrives to some state, can also be calculated.

We model a leader election algorithm with a closed form representation of the behavior of the algorithm.
This closed form representation is a profile Markov chain (noted as $P$).
The profile Markov chain is validated against a chain generated from execution of the algorithm.
The chain constructed from sampled data is known as a test chain (noted as $T$).
