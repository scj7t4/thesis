In this work we have examined an application of leader election algorithms in a cyber-physical system. We showed that while there are definite benefits and uses for leader election algorithms in cyber-physical systems, they generate a flurry of new problems for system designers to deal with. We have shown that network instability can cause disruptions to the amount of service the cyber system can provide. In our analysis we questioned what the effect of transient partitions and link failures would be on the physical system, especially when the two networks are isomorphic. Our work shows that with the selection of an appropriate protocol under certain failure models, a good quality of service can be achieved in general. The transient partition case, by contrast, creates problems with group stability and is an issue to be investigated further with respect to its effect on CPSs.
