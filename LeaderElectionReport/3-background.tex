\section{Introduction}

The DGI uses the leader election algorithm, ``Invitation Election Algorithm,'' written by Garcia-Molina\cite{INVITATIONELECTION}.
This algorithm provides a robust election procedure which allows for transient partitions.
Transient partitions are formed when a faulty link inside a group of processes causes the group to divide temporarily.
These transient partitions merge when the link becomes more reliable.

Many election algorithms have been created.
Specialized algorithms exist for wireless sensor networks\cite{LE-WSN-1}\cite{LE-WSN-2} and other special circumstances\cite{LE-SPECIALCIRCUMSTANCES-1}\cite{LE-SPECIALCIRCUMSTANCES-2}.
Work on leader elections has been incorporated into a variety of distributed frameworks: Isis\cite{ISISTOOLKIT}, Horus\cite{HORUSTOOLKIT}, Totem\cite{TOTEMTOOLKIT}, Transis\cite{TRANSISTOOLKIT}, and Spread\cite{SPREADTOOLKIT} all have methods for creating groups.
Despite this wide array of work, the fundamentals of leader election are similar.
Processes arrive at a consensus of a single peer that coordinates the group.
Processes that fail are detected and removed from the group.

The elected leader is responsible for making work assignments, identifying and merging with other coordinators when they are found, and maintaining an up-to-date list of peers.
Group members monitor the group leader by periodically checking if the group leader is still alive by sending a message.
If the leader fails to respond, the querying peers will enter a recovery state and operate alone until they can identify another coordinator.
Therefore, a leader and each of the members maintain a set of currently reachable processes, a subset of all known processes in the system.

Using a leader election algorithm allows the FREEDM system to autonomously reconfigure rapidly in the event of a failure.
Cyber-components are tightly coupled with the physical components, and reaction to faults is not limited to faults originating in the cyber domain.
Processes automatically react to crash-stop failures, network issues, and power system faults.
The automatic reconfiguration allows processes to react immediately to issues, faster than a human operator, without relying on a central configuration point.
The configuration a leader election supplies has two important characteristics.
First, the configuration should allow the system to perform ``work''.
In the case of the DGI, this work is categorized as a grouping of DGI's where, with the given configuration, any Power/Energy management algorithms running in that group can sucessfully coordinate resources.
Secondly, the leader election algorithm should avoid configurations where the system does ``bad'' work.
In these configurations, the algorithms may attempt to coordinate resources but physical or cyber network failures cause the work performed to contribute to the destabilization of the system.
This destabilization can lead to physical faults such as voltage collapse and blackouts\cite{HARINI}.

