%\specialhead{Abstract}
%\textbf{Abstract}
%\addcontentsline{toc}{lof}{Abstra}
Passive quasi-one-dimensional random media exhibit one of the three regimes of transport -- ballistic, diffusive, or Anderson localization -- depending on system size. The ballistic and diffusion approximations assumes particle transport, whereas Anderson Localization occurs when wave self-interference effects are dominant. When the system contains absorption or gain, then how the regimes can be characterized becomes unclear. By investigating theoretically and numerically the ratio of transmission to energy in a random medium in one dimension, we show this parameter can be used to characterize localization in random media with gain. 

Non-conservative media implies a second dimension for the transport parameter space, namely gain/absorption. By studying the relations between the transport mean free path, the localization length, and the gain or absorption lengths, we enumerate fifteen regimes of wave propagation through quasi-one-dimensional non-conservative random media. Next a criterion characterizing the transition from diffusion to Anderson localization is developed for random media with gain or absorption. The position-dependent diffusion coefficient, which is closely related to the ratio of transmission to energy stored in the system, is investigated using numerical models.
