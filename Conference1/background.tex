% BACKGROUND
% - What technology is the project based on?
%   - Describe the concept of the FREEDM smart-grid
%   - Introduce the idea of using GM to coordinate power resources
%   - Introduce LB as the algorithm that acctually applies the transactions and migrates power
%       - Describe as a flow control algorithm
%   - RED/ECN as a network management technique
%       - RED tries to maintain an everage queue size for a packet queue in a network device.
%       - Packets arriving after the queue is at a certain threshold may be randomly dropped or flagged to signal to the sender that queue is filling
%       - This probability is governed by things.
%       - At a hard limit packets are droped at rate x.
%       - RED also has a gentle mode where the hard rate has a second probability rate up to 2X max threshold.
%       - Packets are always dropped when the queue is full
%       - Used with ECN - a technique for managing congestion. RED can also set an ECN bit in the TCP header instead of dropping.
%       - ECN is typically limited to TCP applications.
%       - Our work tries to apply it to a UDP application to show its usefulness in a CPS.
%   - DGI Theory
%       - Real-time power management
%       - One and done UDP packet transmission with algorithm design that tolerates omission failures.
%       - Load-balancing basic theory.
%       - GM basic theory.
%   - Motivate Problems further
%       - Problem 1 - Groups being unable to form prevents the smart grid from accomplishing anything. Cite previous work as examples of exploration in this area.
%       - Problem 2 - However the configuration is value because:
%           - It detects resources that are no longer reachable or may be difficult to reach
%           - We care about this because related work indicates that k messages (failed migrations) are bad
%           - Diagram a failed migration
%           - Physical networks can handle some predetermined number of k based on their characteristics before they crash

\section{Background}

\subsection{DGI}

The DGI uses the leader election algorithm, ``Invitation Election Algorithm,'' written by Garcia-Molina\cite{INVITATIONELECTION}.
This algorithm provides a robust election procedure which allows for transient partitions.
Transient partitions are formed when a faulty link inside a group of processes causes the group to divide temporarily.
These transient partitions merge when the link becomes more reliable.

The elected leader is responsible for making work assignments, identifying and merging with other coordinators when they are found, and maintaining an up-to-date list of peers.
Group members monitor the group leader by periodically checking if the group leader is still alive by sending a message.
If the leader fails to respond, the querying peers will enter a recovery state and operate alone until they can identify another coordinator.
Therefore, a leader and each of the members maintain a set of currently reachable processes, a subset of all known processes in the system.

Using a leader election algorithm allows the FREEDM system to autonomously reconfigure rapidly in the event of a failure.
Cyber-components are tightly coupled with the physical components, and reaction to faults is not limited to faults originating in the cyber domain.
Processes automatically react to crash-stop failures, network issues, and power system faults.
The automatic reconfiguration allows processes to react immediately to issues, faster than a human operator, without relying on a central configuration point.
However, it is important the configuration a leader election supplies is one where the system can do viable work without causing physical faults like voltage collapse or blackouts\cite{HARINI}.

In this work we utilize the load balancing algorithm from CITE RAVI.
The load balancing algorithm manages power resources by using a sequence of migrations.
In each migration, a sequence of message exchanges identify processes whose power resources are not sufficient to meet their local demand and other processes supply them with power by utilizing a shared bus.
To do this, first processes that cannot meet their demand announce their need to all other processes.
Processes with resources that exceed their demand offer their power to processes that announced their need.
The processes perform a three-way handshake.
At the end of the handshake, the two processes have issued commands to their attached resources to supply power from the shared bus and to draw power from the shared bus.

The DGI executes these modules using a round-robin real-time schedule.
Processes synchronize their clocks and execute modules semi-synchronously.
Each time the load balancing module is scheduled to execute it performs multiple migrations during it's execution phase.
This schedule is depicted in Figure X.
In the figure, each time the load balancing module runs it has the opportunity to complete a fixed number of migrations during its execution window.
The schedule for the DGI is decided before the process is started and does not change when the DGI is running.
All DGI processes that can potentially group together use the same schedule.

The DGI algorithms can tolerate packet loss and is implemented using UDP to pass messages between DGI processes.
Effects of packet loss on the DGI's group management module have been explored in CRITIS and JOURNAL.
The load balancing algorithm can tolerate some message loss, but lost messages can cause migrations to only partially complete, which can cause instability in the physical network.
A failed migration is diagrammed in Figure X.

\subsection{Random Early Detection}
The RED queueing algorithm is a popular queueing algorithm for network devices.
It uses a probabilistic model and an exponentially weighted moving average (EWMA) to determine if the average queue size exceeds predefined values.
These values are used to identify potential congestion and manage it.
This is accomplished by determining the average size of the queue, and then probabilistically dropping packets to maintain the size of the queue.
In RED, when the average queue size $avg$ exceeds a minimum threshold ($min_{th})$), but is less than a maximum threshold ($max_{th}$), new packets arriving at the queue may be ``marked''.
The probability that a packet is marked is based on the following relation between $p_{b}$ and $p_{a}$ where $p_{a}$ is the final probability a packet will be marked.

\[ p_{b} = max_p (avg - min_{th}) / (max_{th}-min_{th}) \]
\[ p_{a} = p_{b} / (1-count * p_b) \]

Where $max_p$ is the maximum probability that a packet will be marked when the queue size is between $min_{th}$ and $max_{th}$ and $count$ is the number of packets since the last marked packet.
With this approach, $p_{b}$ varies linearly with the average queue size, and the $p_{a}$ is a function of that probability and the time since the last packet was marked.
If $avg$ is greater than $max_{th}$, in the gentle variation of the algorithm, the probability of marking increases from $max_p$ to 1 as the average queue size approaches $2*max_{th}$
In this work, we follow the recommendation of the authors and use the gentle variation.
In the event that the queue fills completely, the RED queue operates as a drop-tail queue.

In a simple implementation of the RED algorithm, marked packets are dropped.
For a TCP application, the result of the dropped packets causes the slow-start congestion control strategy to reduce the rate that packets are sent.
A more advanced implementation, using ECN, sets specific bits in the TCP header to indicate congestion.
By using ECN, TCP connections can reduce their transmission rate without re-transmitting packets.

UDP applications have not typically utilized ECN.
Although the ECN standard has flags in the IPv4 header, access to the IPv4 header is not possible on most system, most notably linux.
Furthermore, there is not a "one size fits all" solution to congestion in UDP algorithms.
However, for the DGI and a class of similiar real-time processes, congestion notification has great potential.
If processes can adjust the amount of traffic they send based on the anticipated congestion (by disabling features, for example), they can decrease the effects of that congestion.

