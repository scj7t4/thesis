% BACKGROUND
% - What technology is the project based on?
%   - Describe the concept of the FREEDM smart-grid
%   - Introduce the idea of using GM to coordinate power resources
%   - Introduce LB as the algorithm that acctually applies the transactions and migrates power
%       - Describe as a flow control algorithm
%   - Motivate Problems further
%       - Problem 1 - Groups being unable to form prevents the smart grid from accomplishing anything. Cite previous work as examples of exploration in this area.
%       - Problem 2 - However the configuration is value because:
%           - It detects resources that are no longer reachable or may be difficult to reach
%           - We care about this because related work indicates that k messages (failed migrations) are bad
%           - Diagram a failed migration
%           - Physical networks can handle some predetermined number of k based on their characteristics before they crash
%   - RED/ECN as a network management technique
%       - RED tries to maintain an everage queue size for a packet queue in a network device.
%       - Packets arriving after the queue is at a certain threshold may be randomly dropped or flagged to signal to the sender that queue is filling
%       - This probability is governed by things.
%       - At a hard limit packets are droped at rate x.
%       - RED also has a gentle mode where the hard rate has a second probability rate up to 2X max threshold.
%       - Packets are always dropped when the queue is full
%       - Used with ECN - a technique for managing congestion. RED can also set an ECN bit in the TCP header instead of dropping.
%       - ECN is typically limited to TCP applications.
%       - Our work tries to apply it to a UDP application to show its usefulness in a CPS.
%   - DGI Theory
%       - Real-time power management
%       - One and done UDP packet transmission with algorithm design that tolerates omission failures.
%       - Load-balancing basic theory.
%       - GM basic theory.

