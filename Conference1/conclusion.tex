% CONCLUSION
% Future work
% Summarize.

In this work, we presented a techinque for hardening a real-time distributed cyber-physical system against network congestion.
The RED queueing algorithm and an out-of-band version of explicit congestion notification (ECN) were used to signal an application of congestion.
Using this techinque the application changed several of its characteristics to ready itself for the increased message delays caused by the congestion.

These techniques were demonstrated on the DGI, a distributed control system for the FREEDM smart-grid project.
In particular, this paper demonstrated that the hardening techinques were effective in keeping the DGI processes grouped together.
Additionally, it helped ensure the changes applied to the DGI through cyber-coordinated actions did not destabilize the physical power network.

This techinque will be important to create a robust, reliable CPS for managing future smart-grids.
However, this techinque could potentially be applied to any CPS that could experience congestion on its network, as long as it has the flexibility to change its operating mode.
Potential applications can apply to both the cyber control network and the physically controlled process.
For example, in a VANET system, the vehicles could react to congestion by increasing their following distance.
