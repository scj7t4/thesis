% CONCLUSION
% Future work
% Summarize.

\section{Conclusion}
\label{sect:conclusion}

In this work, we presented a technique for hardening a real-time distributed cyber-physical system against network congestion.
The \ac{RED} queueing algorithm and an out-of-band version of explicit congestion notification (ECN) were used to signal an application of congestion.
Using this technique the application changed several of its characteristics to ready itself for the increased message delays caused by the congestion.

These techniques were demonstrated on the \ac{DGI}, a distributed control system for the \ac{FREEDM} smart-grid project.
In particular, this paper demonstrated the hardening techniques were effective in keeping the \ac{DGI} processes grouped together.
Additionally, it helped ensure the changes applied to the \ac{DGI} through cyber-coordinated actions did not destabilize the physical power network.

This technique will be important to create a robust, reliable \ac{CPS} for managing future smart-grids.
However, this technique could potentially be applied to any \ac{CPS} that could experience congestion on its network, as long as it has the flexibility to change its operating mode.
Potential applications can apply to both the cyber control network and the physically controlled process.
For example, in a \ac{VANET} system, the vehicles could react to congestion by increasing their following distance.
