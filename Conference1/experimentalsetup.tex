% EXPERIMENTAL SETUP
% - Setup
%   \ac{DGI}
%   \ac{NS3}
%   Boost
% - Describe the setup -- Network layout, \ac{DGI} placement at nodes, Random seeds?
% - Describe the quantites we are looking at? What the graphs will mean. Establish the control-- normal operation.
% - Network behavior during the experiments.

\section{Experimental Setup}

Experiments were run in a Network Simulator 3.23 test environment.
The simulation time replaced the wall clock time in the \ac{DGI} for the purpose of triggering real-time events.
As a result, the computation time on the \ac{DGI}s for processing and preparing messages was neglected.
However, to compensate for the lack of processing time, the synchronization of the \ac{DGI}s was instead distributed as a Gaussian distribution.
This was done to introduce realism to ensure that evens did not occur simultaneously as that is a physical impossibility.
Additionally, the real-time schedules used by the \ac{DGI} were adjusted to remove the processing time that was neglected in the simulation.

The \ac{DGI}s were place into a partitioned environment.
The test included 30 nodes.
Each of the nodes ran one \ac{DGI} process.
Two sets of 15 \ac{DGI} were each connect to a switch and each switch was in turn connected to the router.
This network is pictured in Figure X.
Node identifiers were randomly assigned to nodes in the simulation and used as the process identifier for the \ac{DGI}.

The links between the router and the switches had a \ac{RED} enabled queue placed on both network interfaces.
The \ac{RED} parameters for all queues were set identically.
A summary of \ac{RED} parameters are listed in Table X.
All links in the simulation were 100Mbps links with a 0.5ms delay.
RED was used in packet count mode to determine congestion.
ARP tables were populated before the simulation began.

\begin{table*}
\begin{tabular}{ | l | l | } \hline
Parameter & Value         \\ \hline
RED Queueing Mode & Packet\\ \hline 
RED Gentle Mode & True    \\ \hline
RED $Q_{w}$ & 0.002       \\ \hline
RED Wait Mode & True      \\ \hline
RED Min Threshold & 90    \\ \hline
RED Max Threshold & 130   \\ \hline
Maximum Queue Size & 1000 \\ \hline
RED Link Speed & 100 Mbps \\ \hline
RED Link Delay & 0.5 ms   \\ \hline
\end{tabular}
\caption{Summary of \ac{RED} parameters. Unspecified values default to the \ac{NS3} implementation default value}
\label{tab:red-parameters}
\end{table*}

To introduce traffic, a process attached to each of the switches attempted to send a high volume of messages to each other across the router.
Due to the bottleneck due to the properties of the network links, the greatest queueing effect occurred at the switch where the packets originated.

