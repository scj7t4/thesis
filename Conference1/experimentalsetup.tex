% EXPERIMENTAL SETUP
% - Setup
%   DGI
%   NS3
%   Boost
% - Describe the setup -- Network layout, DGI placement at nodes, Random seeds?
% - Describe the quantites we are looking at? What the graphs will mean. Establish the control-- normal operation.
% - Network behavior during the experiments.

\section{Experimental Setup}

Experiments were run in a NS3.23 test environment.
The simulation time replaced the wall clock time in the DGI for the purpose of triggering real-time events.
As a result, the computation time on the DGIs for processing and preparing messages was neglected.
However, to compensate for the lack of processing time, the synchronization of the DGIs was instead distributed as a Gaussian distribution.
This was done to introduce realism to ensure that evens did not occur simultaneously as that is a physical impossibility.
Additionally, the real-time schedules used by the DGI were adjusted to remove the processing time that was neglected in the simulation.

The DGIs were place into a partitioned environment.
The test included 30 DGI.
Two sets of 15 DGI were each connect to a switch and each switch was in turn connected to a router.
This network is pictured in Figure X.

The links between the router and the switches had a RED enabled queue placed on both network interfaces.
The RED parameters for all queues were set identically.
A summary of RED parameters are listed in Table X.
All links in the simulation were 100Mbps links with a 1ms delay.
RED was used in packet count mode to determine congestion.
ARP tables were populated before the simulation began.

To introduce traffic, a process attached to each of the switches attempted to send a high volume of messages to each other across the router.
Due to the bottleneck due to the properties of the network links, the greatest queueing effect occurred at the switch where the packets originated.

