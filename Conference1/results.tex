% RESULTS
%   - Show the unbounded Queue Graph and the mess it makes of LB and GM
%   - Show soft ECVN allows more work to be done, possibly at the cost of accumulating K.
%   - Shwo that HARD ecn allows the best operation -- work gets done and less K is accumlated.

% Show a control. 
% Normal operation. Highlight the bursty peaks from the \ac{DGI} algorithm running
% Point out the average queue size as a second line.
% State a relationship between the average queue size and the message delay.

% Introduce a sufficent amount of traffic to make the queue drain off slowly, recreating the previous figure
% State again that as the amount of other traffic increases the time to drain the queue starts to increase
% State that this does not affect normal operation because the \ac{DGI} has slack built into the schedule to ensure that a normal amount of background traffic does not cripple the \ac{DGI}.
% State that as the traffic increases further, eventually it will over flow the queue
% Show the unbounded queue growth for a drop tail queue.

% Introduce the \ac{RED} queue.
% Show that the \ac{RED} queue, without notifying the \ac{DGI} can do some management, but it isn't sufficent to keep the groups together.
% Show that based on the control, less work gets done and the groups change more frequently.

% Introduce the \ac{ECN} notification
% If the \ac{RED} queue allows \ac{DGI} traffic to pass (making the \ac{RED} queue more like a drop-tail for \ac{DGI} traffic) show that it improves on the previous scenario.
% Show that however, eventually the traffic reaches a threshold where the stratedgy is not sufficent to prevent issues.
% Introduce the \ac{DGI} reacting to the notifications.
% Show that when the \ac{DGI} goes into a maintain mode the average traffic drops.
% Show that this allows a large group to be maintained and the migrations to proceed as normal (with the time reduced schedule).

% Introduce HARD groupbreaks.
% Show that a hard group break greatly reduces traffic (even more than previously) & that this allows the \ac{DGI} to determine how to split.
% Demonstrate a worst-case scenario where the group break is not optimal.
% Show that a planned hard break allows that group to continue operating.

\section{Results}

Figures XXX show the normal operation of the system.
In this configuration, there is no congestion on the network. 
The \ac{DGI}s start, group together and then begin migrating power between processes.
Figure X plots the queue size over time at the busiest switch, Switch X.
The dotted line plots the \ac{EWMA} of the size of the queue.
From this experiment we establish the ``min threshold'' value used as a \ac{RED} queue parameter.
The traffic generated by each step of the group management algorithm is very bursty.
It should be obvious that the tightness of the clock synchronization in the group affect how large this peak is.
Figure Y plots the power level of the processes in the system.
Like SOMEPOWER, the level of the power at a process is the net sum of its power generation capability and load.
As power is shared on the network, processes with excess, converge toward zero net power.
Demand processes also converge toward zero net power.
The number of migrations completed is a good metric for how much work can be performed.
Figure Y represents NUMBER migrations between the 30 processes used in the control.

Figure X shows the queue size as the network traffic begins to increase.
The average queue size increases.
The \ac{DGI}s in these experiments use a schedule with slack under normal network congestion: there is un-need message delivery window.
This slack gives the network devices the opportunity to identify when the network congestion will go beyond the acceptable threshold.
Figure Y shows that when the traffic reaches an a sufficient level, a large unmanaged queue will fill completely.
Instead, congestion notifications signal when the congestion is becoming too high and attempts to manage other traffic by dropping packets.

Figure A shows an example of congestion affecting the physical network without \ac{ECN}.
As a result of the congestion in Figure A, processes leave the main group (Figure B).
Additionally power migrations are affected: migrations or lost, or the supply process is left uncertain of migrations completions.
Figure C plots the count of failed migrations over time.
In this scenario, only XXX migrations are completed compared to the control.

Figure Z shows an example of the \ac{ECN} algorithm notifying processes of the congestion.
Compared to the behavior in Figure A, the \ac{ECN} algorithm successfully prevents the group from dividing, and increases the number of migrations by reducing the number of attempted migrations each round.

Figure D shows an example of a more extreme congestion scenario.
In this scenario, the \ac{RED} algorithm shares a Hard \ac{ECN} notification.
This notification causes the \ac{DGI} to switch to a smaller fallback configuration.
Without this fallback configuration behavior, the system is greatly affected by the traffic.
Additionally, the hard notification provides additional benefits over the soft protection for high levels of traffic.
Figure E shows the same scenario without hard notifications and shows the affect of the of the congestion of the group status.
Table X summarizes the number of migrations and number of failed migrations for the three variations on the scenario.

Figures G, H, I show an extreme congestion scenario.
In this region beyond what the hard notification can protect against, the hard notifications instead allow the operation of the \ac{DGI} to degrade gracefully with congestion.
Compare the behavior in those figures to the behavior shown in Figures J, K, and L.
In Figures J,K,L, hard notifications are disabled, only soft notifications are used.
In those figures, the behavior of the system is adversely affected.
The summary of the affects is listed in Table X.
