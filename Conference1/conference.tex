% This is LLNCS.DEM the demonstration file of
% the LaTeX macro package from Springer-Verlag
% for Lecture Notes in Computer Science,
% version 2.4 for LaTeX2e as of 16. April 2010
%
\documentclass{llncs}

\usepackage{makeidx}  % allows for indexgeneration
\usepackage{comment} % multi-line comments
\usepackage{graphicx} % necessary for inclusion of .eps for figures
\usepackage{algpseudocode}
\usepackage{amsmath}
\usepackage{times}
\usepackage{setspace}
\usepackage{acronym}
\usepackage{caption}
\usepackage{adjustbox}

\graphicspath{ {./drawings/} {./plots/} }
\begin{document}
%
\frontmatter          % for the preliminaries
%
\pagestyle{headings}  % switches on printing of running heads
%\addtocmark{Hamiltonian Mechanics} % additional mark in the TOC
%
\mainmatter
\title{Application of Congestion Notifications in a Cyber-Physical System}
%
\titlerunning{Application of Congestion Notifications}  % abbreviated title (for running head)
%                                     also used for the TOC unless
%                                     \toctitle is used
%
\author{Stephen Jackson \and Dr. Bruce McMillin }
%
\authorrunning{Stephen Jackson et al.} % abbreviated author list (for running head)
%
%%%% list of authors for the TOC (use if author list has to be modified)
%\tocauthor{Ivar Ekeland, Roger Temam, Jeffrey Dean, David Grove,
%Craig Chambers, Kim B. Bruce, and Elisa Bertino}
%
\institute{Missouri University of Science \& Technology, Rolla, MO 65409, USA,\\
\email{\{scj7t4,ff\}@mst.edu}}

\maketitle              % typeset the title of the contribution

\newacro{FREEDM}{Future Renewable Electric Energy Delivery and Management}
\newacro{DGI}{Distributed Grid Intelligence}
\newacro{CPS}{Cyber-Physical System} 
\newacro{AYC}{Are You Coordinator}
\newacro{AYT}{Are You There}
\newacro{RED}{Random Early Detection}
\newacro{NS3}{Network Simulator 3}
\newacro{ECN}{Explicit Congestion Notification}
\newacro{VANET}{Vehicular Ad Hoc Networks}
\newacro{EWMA}{Exponentially Weighted Moving Average}


\begin{abstract}

\section{Abstract}

This work presents a new technique for protecting a cyber-physical, real-time, distributed system during network congestion. 
New applications of existing congestion avoidance and detection methods are used to ensure that the system can still perform useful work, without causing physical instability, during network congestion.
By notifying processes of congestion, the process adjusts its behavior to continue to effectively manage power resources during the network congestion.
We demonstrate with these techniques that the availability of the distributed system is improved through various scenarios.

\keywords{smart-grid, cyber-physical systems, real-time systems, distributed systems, network congestion}
\end{abstract}

\chapter{Introduction}

In the smart grid domain, leader elections are an attractive option for
automonously configuring cyber components. Proposed algorithms such as
\cite{LOADBALANCING} and \cite{INCREMENTALCONSENSUS} are distributed
algorithms for managing power in a smart grid rely on an assumption that
a group of peers will be able to coordinate together. In a system where
100\% up time is not guaranteed, leader
elections are a promising method of establishing these groups.

A strong cyber-physical system should be able to survive and adapt to network
outages in both the physical and cyber domains. When one of these outages
occurs, the physical or cyber components must take corrective action to allow
the rest of the network to continue operating normally. Additionally, other
nodes may need to react to the state change of the failed node. In the realm
of computing, algorithms for managing and detecting when other nodes have
failed is a common distributed systems problem known as leader election.

This work observes the effects of network unreliability on the the group
management module of the Distributed Grid Intelligence (DGI) used by the
FREEDM smart-grid project. This system uses a broker system architecture to
coordinate several software modules that form a control system for a smart
power grid. These modules include: group management, which handles coordinating
nodes via leader election; state collection, a module which captures a global
system state; and load balancing which uses the captured global state to bring
the system to a stable state.

FREEDM (Future Renewable Electric Energy Delivery and Management) System
is a Smart Grid project focused on the future of the electrical grid. Major proposed features
of the FREEDM network include the Solid State Transformer, distributed local energy storage,
and distributed local energy generation\cite{FREEDMMIGRATION}. This vein of research emphasizes decentralizing the power grid: making it more reliable by distributing energy production resources. Part of this
design requires the system to operate in islanded mode, where portions of the distribution
network are segmented from each other. However, there is a major shortage of work
within the realm of the effects cyber outages have on CPSs
\cite{CYBERRESEARCHCALL} \cite{SMARTGRIDBENEFITS}. Addtitionally, research that
has been done such as \cite{HARINI} indicate that cyber faults can cause a physical
system to apply unstable settings.

This work presents the initial steps to better understanding and planning for these faults.
By taking a new approach to considering how a distributed system interacts during a fault condition,
new techniques for managing a fault scenario in a cyber-physical systems will be created. To do
this, we present an approach in modeling the grouping behavior of a system using Markov chains.
These chains produce expectations of how long a system can be expected to stay in a particular
state, or how much time it will be able to spend coordinating and doing useful work over a period
of time. Using these measures, the behavior of the control system for the physical devices
can be adjusted to prevent faults.

% BACKGROUND
% - What technology is the project based on?
%   - Describe the concept of the FREEDM smart-grid
%   - Introduce the idea of using GM to coordinate power resources
%   - Introduce LB as the algorithm that acctually applies the transactions and migrates power
%       - Describe as a flow control algorithm
%   - RED/ECN as a network management technique
%       - RED tries to maintain an everage queue size for a packet queue in a network device.
%       - Packets arriving after the queue is at a certain threshold may be randomly dropped or flagged to signal to the sender that queue is filling
%       - This probability is governed by things.
%       - At a hard limit packets are droped at rate x.
%       - RED also has a gentle mode where the hard rate has a second probability rate up to 2X max threshold.
%       - Packets are always dropped when the queue is full
%       - Used with ECN - a technique for managing congestion. RED can also set an ECN bit in the TCP header instead of dropping.
%       - ECN is typically limited to TCP applications.
%       - Our work tries to apply it to a UDP application to show its usefulness in a CPS.
%   - DGI Theory
%       - Real-time power management
%       - One and done UDP packet transmission with algorithm design that tolerates omission failures.
%       - Load-balancing basic theory.
%       - GM basic theory.
%   - Motivate Problems further
%       - Problem 1 - Groups being unable to form prevents the smart grid from accomplishing anything. Cite previous work as examples of exploration in this area.
%       - Problem 2 - However the configuration is value because:
%           - It detects resources that are no longer reachable or may be difficult to reach
%           - We care about this because related work indicates that k messages (failed migrations) are bad
%           - Diagram a failed migration
%           - Physical networks can handle some predetermined number of k based on their characteristics before they crash

\section{Background}

\subsection{DGI}

The DGI uses the leader election algorithm, ``Invitation Election Algorithm,'' written by Garcia-Molina\cite{INVITATIONELECTION}.
This algorithm provides a robust election procedure which allows for transient partitions.
Transient partitions are formed when a faulty link inside a group of processes causes the group to divide temporarily.
These transient partitions merge when the link becomes more reliable.

The elected leader is responsible for making work assignments, identifying and merging with other coordinators when they are found, and maintaining an up-to-date list of peers.
Group members monitor the group leader by periodically checking if the group leader is still alive by sending a message.
If the leader fails to respond, the querying peers will enter a recovery state and operate alone until they can identify another coordinator.
Therefore, a leader and each of the members maintain a set of currently reachable processes, a subset of all known processes in the system.

Using a leader election algorithm allows the FREEDM system to autonomously reconfigure rapidly in the event of a failure.
Cyber-components are tightly coupled with the physical components, and reaction to faults is not limited to faults originating in the cyber domain.
Processes automatically react to crash-stop failures, network issues, and power system faults.
The automatic reconfiguration allows processes to react immediately to issues, faster than a human operator, without relying on a central configuration point.
However, it is important the configuration a leader election supplies is one where the system can do viable work without causing physical faults like voltage collapse or blackouts\cite{HARINI}.

In this work we utilize the load balancing algorithm from CITE RAVI.
The load balancing algorithm manages power resources by using a sequence of migrations.
In each migration, a sequence of message exchanges identify processes whose power resources are not sufficient to meet their local demand and other processes supply them with power by utilizing a shared bus.
To do this, first processes that cannot meet their demand announce their need to all other processes.
Processes with resources that exceed their demand offer their power to processes that announced their need.
The processes perform a three-way handshake.
At the end of the handshake, the two processes have issued commands to their attached resources to supply power from the shared bus and to draw power from the shared bus.

The DGI executes these modules using a round-robin real-time schedule.
Processes synchronize their clocks and execute modules semi-synchronously.
Each time the load balancing module is scheduled to execute it performs multiple migrations during it's execution phase.
This schedule is depicted in Figure \ref{fig:normal-schedule}.
In the figure, each time the load balancing module runs it has the opportunity to complete a fixed number of migrations during its execution window.
The schedule for the DGI is decided before the process is started and does not change when the DGI is running.
All DGI processes that can potentially group together use the same schedule.

\begin{figure}
\includegraphics[width=\linewidth]{NormalDGISchedule}
\caption{Example of DGI schedule. Processes attempt a fixed number of migrations each round.} \label{fig:normal-schdule}
\end{figure}

The DGI algorithms can tolerate packet loss and is implemented using UDP to pass messages between DGI processes.
Effects of packet loss on the DGI's group management module have been explored in CRITIS and JOURNAL.
The load balancing algorithm can tolerate some message loss, but lost messages can cause migrations to only partially complete, which can cause instability in the physical network.
A failed migration is diagrammed in Figure X.

\subsection{Random Early Detection}
The RED queueing algorithm is a popular queueing algorithm for network devices.
It uses a probabilistic model and an exponentially weighted moving average (EWMA) to determine if the average queue size exceeds predefined values.
These values are used to identify potential congestion and manage it.
This is accomplished by determining the average size of the queue, and then probabilistically dropping packets to maintain the size of the queue.
In RED, when the average queue size $avg$ exceeds a minimum threshold ($min_{th})$), but is less than a maximum threshold ($max_{th}$), new packets arriving at the queue may be ``marked''.
The probability that a packet is marked is based on the following relation between $p_{b}$ and $p_{a}$ where $p_{a}$ is the final probability a packet will be marked.

\begin{equation}
p_{b} = max_p (avg - min_{th}) / (max_{th}-min_{th})
\end{equation}
\begin{equation}
p_{a} = p_{b} / (1-count * p_b)
\end{equation}

Where $max_p$ is the maximum probability that a packet will be marked when the queue size is between $min_{th}$ and $max_{th}$ and $count$ is the number of packets since the last marked packet.
With this approach, $p_{b}$ varies linearly with the average queue size, and the $p_{a}$ is a function of that probability and the time since the last packet was marked.
If $avg$ is greater than $max_{th}$, in the gentle variation of the algorithm, the probability of marking increases from $max_p$ to 1 as the average queue size approaches $2*max_{th}$
In this work, we follow the recommendation of the authors and use the gentle variation.
In the event that the queue fills completely, the RED queue operates as a drop-tail queue.

In a simple implementation of the RED algorithm, marked packets are dropped.
For a TCP application, the result of the dropped packets causes the slow-start congestion control strategy to reduce the rate that packets are sent.
A more advanced implementation, using ECN, sets specific bits in the TCP header to indicate congestion.
By using ECN, TCP connections can reduce their transmission rate without re-transmitting packets.

UDP applications have not typically utilized ECN.
Although the ECN standard has flags in the IPv4 header, access to the IPv4 header is not possible on most system, most notably linux.
Furthermore, there is not a "one size fits all" solution to congestion in UDP algorithms.
However, for the DGI and a class of similiar real-time processes, congestion notification has great potential.
If processes can adjust the amount of traffic they send based on the anticipated congestion (by disabling features, for example), they can decrease the effects of that congestion.


% APPLICATION
% - Usage Theory
%       - Instead of dropping packets, the network device with \ac{RED} will send messages to an application announcing congestion
%       - These announcements allow the \ac{CPS} to adjust its behavior to allow better operation during the network fault.
%       - Network layout and design
%       - What are the goals of the successful operation of the \ac{CPS} algorithm
%       - Balance the amount of K that accumlates, while maximizing the amount of work done.
%       - Describe the type of traffic we are accounting for
% - What happens on the receipt of a Soft \ac{ECN} Message & Motivate this approach
% - What happens on the receipt of a hard \ac{ECN} message & Motivate this approach
% - Justify why these approaches are good for the \ac{CPS}.
% - Make sure there's mathy stuff about why this is better.
% - Justification using the models developed in the last paper--
%   - The congestion would normally cause \ac{RED} to drop packets at a certain rate, we're entering a contract with the router to change behavior rather than having an omission rate.
% - Tune \ac{RED} parameters based on Journal paper?

\section{Application}
\label{sect:application}

\subsection{Usage Theory}
When the \ac{RED} algorithm identifies congestion it must notify senders of congestion.
Since the \ac{ECN} fields in IPv4 are not available to applications running on the system, the notifications are multicast onto the source interface.
Additionally, since this approach is non-standard and most UDP applications would not understand the notification, we have opted to create an application that runs on the network device.
This application is responsible for generating the multicast message.
It also keeps a register of hosts running applications that support reacting to the \ac{ECN} notification.

%There are several reasons for this approach.
%First, related work has shown an \ac{ECN} strategy without some other queue management scheme is not sufficient to prevent congestion.
%By allowing real-time applications that decrease the number of messages for congestion special priority in the \ac{RED} algorithm, we allow those applications to continue operating during congestion.
%Additionally, in later sections, we demonstrate this strategy is effective for managing congestion.

When the \ac{RED} algorithm detects congestion, it sends a multicast beacon to a group of interfaces informing of the level of congestion.
For similarity with the \ac{RED} algorithm and the \ac{NS3} implementation, this notification is classified as either ``soft'' or ``hard.''
A soft notification is an indication the congestion in the network is approaching a level where real-time processes can expect message delays that may affect their normal operation.
A hard notification indicates the congestion has reached a level where messages may be subject to both delay and loss.
The notifications are rate limited so they do not flood the network.

\subsection{Group Management}

The group management module's execution schedule is broken into several periods of message generation and response windows.
Because the schedule of the \ac{DGI} triggers the execution of group management modules approximately simultaneously, the traffic generated by modules is bursty.
The number of messages sent is $O(n^2)$ (where n is the number of processes in the system), in a brief window, which is dependent on how well the clocks are synchronized in the system.
The duration of the response window is dependent on the amount of time it takes for messages to propagate to the hardest-to-reach process the \ac{DGI} hopes to group with.
Additionally, to contend with congestion, an additional slack must be added to allow the \ac{RED} algorithm to detect congestion before it reaches a critical level.

Figure \ref{fig:queue-types} depicts typical queueing behavior for a network device serving \ac{DGI} processes under different circumstances.
Because the traffic generated by \ac{DGI} modules is very bursty, the queue experiences a phenomena where the bursty traffic mixed with a steady background traffic causes the queue to fill.
With no background traffic, the impulse queues a large number of messages, but those messages are distributed in a timely manner.
When the background traffic is introduced, the queue takes longer to empty.
At a critical threshold, the queue does not empty completely before the next burst is generated by the \ac{DGI}.
In this scenario, the queue completely fills and no messages can be distributed.
The \ac{RED} algorithm and \ac{ECN} are used to delay or prevent the queue from reaching this critical threshold.

\begin{figure}
\includegraphics[width=\linewidth]{QueueStacked}
\caption{
Example of network queueing during \ac{DGI} operation. \ac{DGI} modules are semi-synchronous, and create bursty traffic on the network.
When there is no other traffic on the network (solid line), the bursty traffic causes a large number of packets to queue quickly, but the queue empties at a similar rate.
With background traffic (dashed line), the bursty traffic causes a large number of packets to be queued suddenly. More packets arrive continuously, causing the queue to drain off more slowly.
When the background traffic reaches a certain threshold (dotted line), the queue does not empty before the next burst occurs. When this happens, messages will not be delivered in time, and the queue will completely fill.
}
\label{fig:queue-types}
\end{figure}

For this work, the algorithm from \cite{JOURNALANON} was used.
This algorithm has a higher message complexity when in a group than the Garcia-Molina algorithm it is based on.
However, it does possess a desirable memoryless property that makes it easy to analyze.
This work uses an improved version of the algorithm which removes the restrictions in \cite{JOURNALANON} where only one process could become the leader.

\subsubsection{Soft \ac{ECN}}

A soft \ac{ECN} message indicates the network has reached a level of congestion where the router suspects processes will not be able to meet their real time requirements.
The soft \ac{ECN} message encourages the \ac{DGI} processes to reduce the number of messages they send to reduce the amount of congestion they contribute to the network, and to allow for reliable distribution techniques to have additional time to deliver messages (since fewer messages are being sent).
In the case of potential congestion, the group management module can reduce its traffic bursts by disabling elections during the congestion.
When the elections are disabled, messages for group management are only sent to members of the group.
Processes do not seek out better or other leaders to merge with.
As a consequence, the message complexity for processes responding to the congestion notification reduces from $O(n^2)$ to $O(n)$.

\subsubsection{Hard \ac{ECN}}

In a hard \ac{ECN} scenario, the router will have determined congestion has reached a threshold where the real-time processes will soon not be able to meet their deadlines.
In this scenario, the real-time process will likely split its group.
In an uncontrolled situation, the split will be random.
It is therefore desirable when this level of traffic is reached to split the group.
Splitting the group reduces the number of messages sent across the router for modules with $O(n^2)$ (where $n$ is the number of processes in the original group) message complexity.
For larger groups, splitting them provides a significant savings in the number of messages that must be queued by the router, especially since the traffic is very bursty.

\begin{figure}
\includegraphics[width=\linewidth]{NetworkLayout}
\caption{Example of process organization used in this paper. Two groups of processes are connected by a router.} \label{fig:network-layout}
\end{figure}

Suppose a network like one depicted in Figure \ref{fig:network-layout}, where processes are divided by a router.
In Figure \ref{fig:network-layout}, there are $n$ processes on one side of the network and $m$ on the other.
In normal operation the omission-modelable algorithm has an $O(n^2)$ message complexity.
In Soft \ac{ECN} maintenance mode, the reduced number of messages reduces the complexity to $O(n)$ by disabling elections.

During elections (and with each group update) the leader distributes a fallback configuration that will coordinate the division of the groups during intense congestion.
When the \ac{ECN} notification is received the processes will halt all current group management operations and enter a splitting mode where they switch to the fallback configuration.
The leader of the group distributes a fallback notification to ensure all processes in the group apply their new configuration. 
The complexity of distributing the notification is linear $O(n)$ and processes that already received the notification will have halted their communication.
This approach will ideally avoid the burst/drain phenomena from figure \ref{fig:queue-types}.

The design of the fallback configuration can be created to optimize various factors.
These factors include cyber considerations, such as the likely network path the processes in the group will use to communicate.
By selecting the group around the network resources, the group can be selected to minimize the amount of traffic that crosses the congested links in the future.
Additionally, considerations from the physical network can be considered.
Fallback groups can be created to ensure they can continue to facilitate the needs of the members.
This can take into the consideration the distribution of supply and demand processes in the current group.
By having a good mix of process types in the fallback group the potential for work can remain high.

\subsection{Cyber-Physical System}

For a real-time \ac{CPS}, message delays could affect coordinated actions.
As result, these actions may not happen at the correct moments or at all.
Since the two-army problem prevents any process from being entirely certain a coordinated action will happen in concert, problems arising from delay or omission of messages is of particular interest.
In particular, we are interested in the scenario from \cite{HARINI}, where only half of a power migration is performed.
Other power management algorithms could have similar effects on the power system based on this idea of a process performing an action that is not compensated for by other processes.

\subsubsection{Soft \ac{ECN}}

In a soft congestion notification mode, the process being informed of the congestion can reduce its affect on the congestion by changing how often it generates bursty traffic.
Processes running the load balancing algorithm make several traffic bursts when they exchange state information and prepare migrations.
As shown before, if the interval between these bursts is not sufficient for the queue to drain before the next burst occurs, then critical, overwhelming congestion occurs.
Since the schedule of the \ac{DGI} is fixed at run-time processes cannot simply extend the duration of the load balancing execution phase.
However, on notification from the leader, the process can, instead, adjust the number of migrations to increase the message delivery interval.
This notification to reduce the schedule originates from the coordinator as part of the message exchange necessary for the process to remain in the group.
Every process in the group must receive this message to participate in load balancing, ensuring all processes remain on the same real-time schedule.
Using this approach, the amount of traffic generated is unchanged but the time period a process waits for the messages to be distributed is increased.

\subsubsection{Hard \ac{ECN}}

When the \ac{DGI} process receives a hard congestion notification, the processes switch to a predetermined fallback configuration.
This configuration creates a cyber partition.
By partitioning the network, the number of messages sent by applications with $O(n^2)$ message complexity can be reduced significantly.
Each migration of load balancing algorithm begins with an $O(n^2)$ message burst and so benefits from the reduced group size created by the partition.

Suppose there is a network like the in Figure \ref{fig:network-layout} with $n$ processes on one half and $m$ on the other.
The number of messages sent across the router for the undivided group is of the order $2mn$ as the $n$ processes on side A send a message to the $m$ on side B and vice-versa.
Let $i_{1}$ and $j_{1}$ be the number of processes from side A and side B (respectively) in the first group created by the partition.
Let $i_{2}$ and $j_{2}$ be the number of processes in the second group created by the partition under the same circumstances of $i_1$ and $j_1$.
The number of messages sent that pass through the router, is then 

\begin{equation}
2 i_{1} j_{1} + 2 i_{2} j_{2}
\end{equation}

For an arbitrary group division, the following can be observed.
Suppose $i_{1}$ and $j_{2}$ are the cardinality of two arbitrarily chosen sets of processes from side A and side B respectively.
Following the same cut requirements as before:

\begin{equation}
i_2 = n - i_1
\end{equation}
\begin{equation}
j_2 = m - j_1
\end{equation}

The the number of messages that must pass through the router for this cut is:

\begin{equation}
2 i_{1} j_{1} + 2 (n-i_{1}) (m-j_{1})
\end{equation}
\begin{equation}
= 2 i_{1} j_{1} + 2 (nm - mi_{1} - nj_{1} + i_{1}j_{1})
\end{equation}
\begin{equation}
= 2 (2 i_{1} j_{1} + nm - mi_{1} - nj_{1})
\end{equation}

The value is maximized when $i_1$ and $j_1$ are $\frac{n}{2}$ and $\frac{m}{2}$:

\begin{equation}
2( 2 \frac{mn}{4} + mn - \frac{mn}{2} - \frac{mn}{2})
\end{equation}
\begin{equation}
= mn
\end{equation}

Which is a reduction of half as many messages.
For systems with a large number of participating processes this represents a significant reduction in the number of messages sent across the router.
As a consequence, this further extends the delivery window for processes sending messages.

In the best case scenario, some cut will have a single process opposite a large number of processes.
Consider cut where one process is selected from side A and $m-1$ are selected from side B.
The cut will also create a second group of $n-1$ processes from side A and one process from side B.

\begin{equation}
2(m-1) + 2(n-1)
\end{equation}

Which represents a reduction in message complexity from the original $2mn$.

\subsection{Relation To Omission Model}

The synchronization of clocks in the environment is assumed to be normally distributed around a true time value provided by the simulation.
The shape of the curve created by plotting the queue resembles that of the \ac{CDF} of the normal distribution, noted $F(x)$.
A simple description of the traffic behavior can then be described in terms of that curve.
First, observe that when the queue hits a specific threshold, even if the queue is drained at an optimal rate, the $n$th queued packed will not be delivered in time:

\begin{equation}
Qsize - min(Qsize, (DequeueRate * \Delta t)) \geq 0
\label{eq:origin}
\end{equation}

Where $\Delta t$ is the deadline for the message to be delivered.
If the size of the queue exceeds the number of messages that can be delivered before $\Delta t$ passes, some messages will not be delivered.
The size of the queue during the message bursts created by the DGI depends on the message complexity of the algorithm, the number of messages already in the queue, the other traffic on the network, and any replies that also have to be delivered in that interval.
Therefore, let $c$ represent the rate that traffic is generated by other processes.
Let $init_q$ represent the number of messages in the queue at the start of a burst. 
Let $init_m$ represent the number of messages sent in the beginning of the burst.
Let $resp$ represent the number of messages sent in response to the burst that must still be delivered before $\Delta t$ passes.
We can then express $Qsize$ as two parts:

\begin{equation}
Qsize = Burst + Obligations
\end{equation}

Where $Burst$ takes the form of the \ac{CDF} for the normal distribution:

\begin{equation}
Burst = init_m * F(x)  
\end{equation}

\begin{equation}
Obligations = c * \Delta t + init_q + resp
\end{equation}

From this we can derive the equation:

\begin{equation}
F(x) \geq \frac{DequeueRate * \Delta t - c * \Delta t - init_q - resp}{init_m}
\label{eq:prob-est}
\end{equation}

Where, from Equation \ref{eq:origin}, $DequeueRate * \Delta t$ is less than or equal to the number of messages in the queue. 
Solving for $F(x)$ gives a worst case estimate of the omission rate for a specific algorithmic or network circumstance.
$DequeueRate$ is affected by the amount of traffic in the system. 
It should be obvious a greater amount of background traffic corresponds to a greater average queue size.
From an relationship between the background traffic and the average queue size and the results presented in \cite{JOURNALANON}, Equation \ref{eq:prob-est} can be used to select the ECN parameters.

% EXPERIMENTAL SETUP
% - Setup
%   \ac{DGI}
%   \ac{NS3}
%   Boost
% - Describe the setup -- Network layout, \ac{DGI} placement at nodes, Random seeds?
% - Describe the quantites we are looking at? What the graphs will mean. Establish the control-- normal operation.
% - Network behavior during the experiments.

\section{Experimental Setup}
\label{sect:experimentalsetup}

Experiments were run in a Network Simulator 3.23\cite{NS3} test environment.
The simulation time replaced the wall clock time in the \ac{DGI} for the purpose of triggering real-time events.
As a result, the computation time on the \ac{DGI}s for processing and preparing messages was neglected.
However, to compensate for the lack of processing time, the synchronization of the \ac{DGI}s was instead randomly distributed as a normal distribution.
This was done to introduce realism to ensure events did not occur simultaneously.
Additionally, the real-time schedules used by the \ac{DGI} were adjusted to remove the processing time that was neglected in the simulation.

The \ac{DGI}s were placed into a partitioned environment.
The test included 30 nodes.
Each of the nodes ran one \ac{DGI} process.
Two sets of 15 \ac{DGI} were each connect to a switch and each switch was in turn connected to the router.
This network is pictured in Figure \ref{fig:network-layout}.
Node identifiers were randomly assigned to nodes in the simulation and used as the process identifier for the \ac{DGI}.

The links between the router and the switches had a \ac{RED} enabled queue placed on both network interfaces.
The \ac{RED} parameters for all queues were set identically.
A summary of \ac{RED} parameters are listed in Table \ref{tab:red-parameters}.
All links in the simulation were 100Mbps links with a 0.5ms delay.
RED was used in packet count mode to determine congestion.
ARP tables were populated before the simulation began.

\ac{RED} parameters were selected using Equation \ref{eq:prob-est} and \cite{JOURNAL}.
The relationship between the background traffic and the average queue size was estimated through runs of the \ac{NS3} simulation.
Figure \ref{fig:plotm} demonstrates the observed relationship between the total background traffic and the maximum average queue size for that level of traffic.
Additionally, the $DequeueRate$ was collected from a run of the simulation without traffic, and was found to be $713.08$ packets/second.
Therefore, from Equation \ref{eq:prob-est}, assuming $init_q=0, resp=225, init_m=225$ and $\Delta t=1$, the maximum traffic rate with no omissions is $263.0$ packets/second.
The number of packets for the $resp$ and $init_m$ were selected from the worst case of the algorithm in \cite{JOURNAL}.
Based on the traffic parameters in Table \ref{tab:red-parameters}, $263.0$ packets/second corresponds to 1.077 Mbps of traffic generated at one switch and 2.1545 Mbps traffic overall.
From the polynomial estimate in Figure \ref{fig:plotm}, the maximum average queue size for that level of traffic is $94.715$, estimated as $90$ for the \ac{RED} Min Threshold in Table \ref{tab:red-parameters}.
RED Max Threshold is computed using a similar technique, but using the message complexity for the Load Balancing algorithm, since it maintains its complexity during Soft ECN mode.

\begin{figure}
\centering
\includegraphics[width=0.65\linewidth]{m-max-average-queue.pdf}
\caption{Plot of the maximum observed average queue size as a function of the overall background traffic. The polynomial estimate is $y=22.70x^2-44.74x+85.72$}
\label{fig:plotm}
\end{figure}

\begin{table}
\begin{center}
\begin{tabular}{ | l | l | } \hline
Parameter & Value         \\ \hline
RED Queueing Mode & Packet\\ \hline 
RED Gentle Mode & True    \\ \hline
RED $Q_{w}$ & 0.002       \\ \hline
RED Wait Mode & True      \\ \hline
RED Min Threshold & 90    \\ \hline
RED Max Threshold & 130   \\ \hline
Maximum Queue Size & 1000 \\ \hline
RED Link Speed & 100 Mbps \\ \hline
RED Link Delay & 0.5 ms   \\ \hline
Clock Distribution $\sigma$ & 0.005 \\ \hline
Traffic Packet Size & 512 Bytes \\ \hline
\end{tabular}
\end{center}
\caption{Summary of \ac{RED} parameters. Unspecified values default to the \ac{NS3} implementation default value}
\label{tab:red-parameters}
\end{table}

To introduce traffic, processes attached to each of the switches attempted to send a high volume of messages to each other across the router.
The number of packets sent per second was a function of the data rate and the size of the packets sent.
In each simulation, half of the traffic originated from each switch.
Due to the bottleneck due to the properties of the network links, the greatest queueing effect occurred at the switches.

\chapter{Results}

\section{Initial Results}

Initial data was collected from a non-real time version of the DGI code.
For each selected message arrival chance, as many as forty tests were run.
The collected results from the tests are divided into several target scenarios as well as the protocol used.

The first minute of each test in the experimental test is discarded so that any transients in the test could be removed.
The tests were run for ten minutes, however the maximum result was 9 minutes of in group time.
These graphs first appeared in \cite{CRITIS2012}.

\subsection{Sequenced Reliable Connection}

\subsubsection{Two Node Case}

\begin{figure}
\centering
\begin{minipage}{0.45\textwidth}
    \centering
    \includegraphics[width=\textwidth]{2NODE-SRC-100-GROUP.pdf}
    \caption{Time in-group over a 10 minute run for a two node system with a 100ms resend time}
    \label{fig:IGT-SRC-2NODE-100}
\end{minipage}%
\qquad
\begin{minipage}{0.45\textwidth}
    \centering
    \includegraphics[width=\textwidth]{2NODE-SRC-200-GROUP.pdf}
    \caption{Time in-group over a 10 minute run for a two node system with a 200ms resend time}
    \label{fig:IGT-SRC-2NODE-200}
\end{minipage}
\end{figure}

The 100ms resend SRC test with two nodes can be considered a type of control in this study.
These tests, pictured in Figure \ref{fig:IGT-SRC-2NODE-100}, highlight the performance of the SRC protocol.
The maximum in group time of 9 minutes was achieved with only 15\% of datagrams arriving at the receiver. 

Figure \ref{fig:IGT-SRC-2NODE-200} demonstrates that as the rate at which lost datagrams were re-sent was decreased to 200ms, the in-group time decreased.
This behavior was expected.
Each exchange had a time limit for each message to arrive and the number of attempts was reduced by increasing the resend time.

\subsubsection{Transient Partition Case}

\begin{figure}
\centering
\begin{minipage}{0.45\textwidth}
    \centering
    \includegraphics[width=\textwidth]{TRANS-SRC-100-SIZE.pdf}
    \caption{Average size of formed groups for the transient partition case with a 100ms resend time}
    \label{fig:MGS-SRC-TRANS-100}
\end{minipage}%
\qquad
\begin{minipage}{0.45\textwidth}
    \centering
    \includegraphics[width=\textwidth]{TRANS-SRC-100-GROUP.pdf}
    \caption{Time in-group over a 10 minute run for the transient partition case with a 100ms resend time}
    \label{fig:IGT-SRC-TRANS-100}
\end{minipage}
\end{figure}

The transient partition case is a simple example in which a network partition separates two groups of DGI processes. In the simplest case where the opposite side of the partition is unreachable, nodes will form a group with the other nodes on the same side of the partition.
In this study, two processes were present on each side of the partition.
In the experiment, the probability of a datagram crossing the partition was increased as the experiment continued.
The 100ms case is shown in Figures \ref{fig:MGS-SRC-TRANS-100} and \ref{fig:IGT-SRC-TRANS-100}.

While messages cannot cross the partition, the DGIs stay in a group with the nodes on the same side of the partition, leading to an in-group time of 9 minutes (the maximum value possible).
As packets began to cross the partition (as the reliability increases), DGI instances on either side attempted to complete elections with the nodes on the opposite partition and the in group time began to decrease.
During this time, however, the mean group size continued to increase.
Thus, while the elections were decreasing the amount of time that the module spent in a state where it can actively do work, it typically did not fall into a state where it was in a group by itself. 
As result, most of the lost in group time comes from elections.

\begin{figure}
\centering
\begin{minipage}{0.45\textwidth}
    \centering
    \includegraphics[width=\textwidth]{TRANS-SRC-200-SIZE.pdf}
    \caption{Average size of formed groups for the transient partition case with a 200ms resend time}
    \label{fig:MGS-SRC-TRANS-200}
\end{minipage}%
\qquad
\begin{minipage}{0.45\textwidth}
    \centering
    \includegraphics[width=\textwidth]{TRANS-SRC-200-GROUP.pdf}
    \caption{Time in-group over a 10 minute run for the transient partition case with a 200ms resend time}
    \label{fig:IGT-SRC-TRANS-200}
\end{minipage}
\end{figure}

The 200ms case (Illustrated in Figures \ref{fig:MGS-SRC-TRANS-200} and \ref{fig:IGT-SRC-TRANS-200}) suggests a similar behavior to Figures \ref{fig:MGS-SRC-TRANS-100} and \ref{fig:IGT-SRC-TRANS-100}, with a wider valley produced by the reduced number of datagrams.
The mean group size dips below 2 in Figure \ref{fig:MGS-SRC-TRANS-200}, possibly because longer resend times allowed for a greater number race conditions between potential leaders.
Discussion of these race conditions is shown and discussed during the SUC section since it was more prevalent in those experiments.

\subsection{Sequenced Unreliable Connection}

\subsubsection{Two Node Case}

The SUC protocol's experimental tests revealed an immediate problem.
There is a general increasing trend for the amount of time in-group shown in Figure \ref{fig:IGT-SUC-2NODE-100}.
There is a high amount of variance, however, for any particular trial.

\begin{figure}
\centering
\begin{minipage}{0.45\textwidth}
    \centering
    \includegraphics[width=\textwidth]{2NODE-SUC-100-GROUP.pdf}
    \caption{Time in group over a 10 minute run for two node system with 100ms resend time}
    \label{fig:IGT-SUC-2NODE-100}
\end{minipage}%
\qquad
\begin{minipage}{0.45\textwidth}
    \centering
    \includegraphics[width=\textwidth]{2NODE-SUC-200-GROUP.pdf}
    \caption{Time in group over a 10 minute run for two node system with 200ms resend time}
    \label{fig:IGT-SUC-2NODE-200}
\end{minipage}
\end{figure}

In the 200ms resend case (illustrated in Figure \ref{fig:IGT-SUC-2NODE-200}), a greater growth rate occured in the in group time as the reliability increased.
When an average was taken across all of the collected data points from the experiment, the average in group time is higher for the 200ms case than it was for the 100ms case (6.86 vs 6.09).
There large amount of variance in the collected in group time, however.
As a result, it is not possible to state with confidence that the there is a significant difference between the two cases.

\section{Markov Models}

There was high amount of variance in the collected data.
As a result it was difficult to make any sort of prediction about other configurations from the data.
Markov chains were employed to model the system.

\subsection{Initial Model Calibration}

The presented methodology of constructing the model was initially calibrated against the original two-node case.
This calibration used a non-real-time version of the DGI codebase.
The resulting Markov chain was processed using SharpE \cite{SHARPE}\cite{SHARPE2} made by Dr. Kishor Trivedi's group at Duke University, a popular tool for reliability analysis.
SharpE measured the reward collected in 600 seconds, minus the reward that was collected in the first 60 seconds. 
Discarding the reward from the first 60 seconds emulated the 60 seconds were discarded in the experimental runs.
The SharpE results are plotted along with the experimental results in Figures \ref{fig:COMPARE-SUC-2NODE-100} and \ref{fig:COMPARE-SUC-2NODE-200}.

\begin{figure}
\centering
\begin{minipage}{0.45\textwidth}
    \centering
    \includegraphics[width=1.0\textwidth]{2NODE-SUC-100-COMPARE.pdf}
    \caption{Comparison of in-group time as collected from the experimental platform and the simulator (1 tick offset between processes).}
    \label{fig:COMPARE-SUC-2NODE-100}
\end{minipage}%
\qquad
\begin{minipage}{0.45\textwidth}
    \centering
    \includegraphics[width=1.0\textwidth]{2NODE-SUC-200-COMPARE.pdf}
    \caption{Comparison of in-group time as collected from the experimental platform and the simulator (2 tick offset between processes).}
    \label{fig:COMPARE-SUC-2NODE-200}
\end{minipage}
\end{figure}

The race condition between processes during an election is a consideration in the original leader election algorithm, and is an additional factor here.
The simulator provided a parameter to allow the operator to select how closely synchronized the peers were.
This synchronization was the time difference between when each of them would search for leaders.
The exchange of messages, particularly during an election, had a tendency to synchronize nodes during elections.
Nodes could synchronize even if they did not initially begin in a synchronized state. 
The simulation results aligned best for the 100ms resend case with 1 tick (Approximately 100ms difference in synchronization between processes) and 2 ticks (Approximately 400ms) in the 200ms resend case.

Models fit to the non-real-time code in groups larger than two processes had a poor fit.
This is presumed to be a combination of several factors.
The major source of fault included the structure of the chain. 
Construction of the chain assumes that all processes enter the election state a roughly the same time. 
This was not typically true for more than two processes.
Additionally, the simulator could only assume that the synchronization between processes was fixed.
The coincidental synchronization that occurred in the two node case was suppressed by the increased number of peers.
Furthermore, an issue with SharpE was discovered that prevented the particular structure of the chains produced from being handled correctly.
The election states with only one outbound transition uncovered a bug in the SharpE software.
To circumvent that, issue, SharpE was replaced by a random-walker which generates exponentially distributed numbers and follows the paths of the chain.
Time in group data for models which SharpE cannot process were collected across several hundred trials.

\begin{figure}
\centering
\begin{minipage}{0.45\textwidth}
    \centering
    \includegraphics[width=1.0\textwidth]{TRANS-RT-SUC-128-COMPARE.pdf}
    \caption{Comparison of in-group time as collected from the experimental platform and the time in group from the equivalent Markov chain (128ms between resends).}
    \label{fig:COMPARE-SUC-TRANS-RT-128}
\end{minipage}%
\qquad
\begin{minipage}{0.45\textwidth}
    \centering
    \includegraphics[width=1.0\textwidth]{TRANS-RT-SUC-64-COMPARE.pdf}
    \caption{Comparison of in-group time as collected from the experimental platform and the time in group from the equivalent Markov chain (64ms between resends).}
    \label{fig:COMPARE-SUC-TRANS-RT-64}
\end{minipage}
\end{figure}

The structure of the Markov Chain assumed that processes enter the election state simultaneously.
This was an appropriate assumption for the real-time system, since the round-robin scheduler synchronized when processes ran their group management modules.
The simulator was set to assume that the synchronization between processes was very tight.
New experimental data was collected for the 4 node, transient partition case.
The collected data is overlaid with the results from the random walker in Figures \ref{fig:COMPARE-SUC-TRANS-RT-128} and \ref{fig:COMPARE-SUC-TRANS-RT-64}.

\begin{table}
% increase table row spacing, adjust to taste
\caption{Error and correlation of experimental data and Markov chain predictions}
\label{tab:STAT-DATA}
\centering
% Some packages, such as MDW tools, offer better commands for making tables
% than the plain LaTeX2e tabular which is used here.
\begin{tabular}{|c||c|c|c|}
\hline
Re-send & Correlation & Error \\ \hline
128 & 0.7656 & 11.61\% \\ \hline
64 & 0.8604 & 11.70\% \\ \hline
\end{tabular}
\end{table}

As a measure of the strength of the model, the correlation between the predicted value was compared.
The average error was also computed for each of the samples taken.
This information is presented in Table \ref{tab:STAT-DATA}.

% CONCLUSION
% Future work
% Summarize.

In this work, we presented a techinque for hardening a real-time distributed cyber-physical system against network congestion.
The RED queueing algorithm and an out-of-band version of explicit congestion notification (ECN) were used to signal an application of congestion.
Using this techinque the application changed several of its characteristics to ready itself for the increased message delays caused by the congestion.

These techniques were demonstrated on the DGI, a distributed control system for the FREEDM smart-grid project.
In particular, this paper demonstrated that the hardening techinques were effective in keeping the DGI processes grouped together.
Additionally, it helped ensure the changes applied to the DGI through cyber-coordinated actions did not destabilize the physical power network.

This techinque will be important to create a robust, reliable CPS for managing future smart-grids.
However, this techinque could potentially be applied to any CPS that could experience congestion on its network, as long as it has the flexibility to change its operating mode.
Potential applications can apply to both the cyber control network and the physically controlled process.
For example, in a VANET system, the vehicles could react to congestion by increasing their following distance.


\subsubsection*{Acknowledgments.}

The authors acknowledge the support of the Future Renewable
Electric Energy Delivery and Management Center,
a National Science Foundation supported Engineering Research
Center under grant NSF EEC-081212, and the United States Department of Education GAANN program.

%
% ---- Bibliography ----
%
\bibliographystyle{splncs03}
\bibliography{latex_bibliography}
%


\end{document}
